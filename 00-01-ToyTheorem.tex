\documentclass[12pt]{article}
\usepackage{pmmeta}
\pmcanonicalname{ToyTheorem}
\pmcreated{2013-03-22 13:55:35}
\pmmodified{2013-03-22 13:55:35}
\pmowner{matte}{1858}
\pmmodifier{matte}{1858}
\pmtitle{toy theorem}
\pmrecord{7}{34684}
\pmprivacy{1}
\pmauthor{matte}{1858}
\pmtype{Definition}
\pmcomment{trigger rebuild}
\pmclassification{msc}{00-01}

% this is the default PlanetMath preamble.  as your knowledge
% of TeX increases, you will probably want to edit this, but
% it should be fine as is for beginners.

% almost certainly you want these
\usepackage{amssymb}
\usepackage{amsmath}
\usepackage{amsfonts}

% used for TeXing text within eps files
%\usepackage{psfrag}
% need this for including graphics (\includegraphics)
%\usepackage{graphicx}
% for neatly defining theorems and propositions
%\usepackage{amsthm}
% making logically defined graphics
%%%\usepackage{xypic}

% there are many more packages, add them here as you need them

% define commands here

\newcommand{\sR}[0]{\mathbb{R}}
\newcommand{\sC}[0]{\mathbb{C}}
\newcommand{\sN}[0]{\mathbb{N}}
\newcommand{\sZ}[0]{\mathbb{Z}}

% The below lines should work as the command
% \renewcommand{\bibname}{References}
% without creating havoc when rendering an entry in 
% the page-image mode.
\makeatletter
\@ifundefined{bibname}{}{\renewcommand{\bibname}{References}}
\makeatother

\newcommand*{\norm}[1]{\lVert #1 \rVert}
\newcommand*{\abs}[1]{| #1 |}
\begin{document}
A \emph{toy theorem} is a simplified version of a more general theorem.
For instance, by introducing some simplifying assumptions in a theorem,
one obtains a toy theorem.

Usually, a toy theorem is used to illustrate the claim of a theorem. 
It can also be illustrative and insightful to study proofs of
a toy theorem derived from a non-trivial theorem. 
Toy theorems also have a great education value. 
After presenting a theorem (with, say, a highly non-trivial proof),
one can sometimes give some assurance that the theorem
really holds, by proving a toy version of the theorem. 

For instance, a toy theorem of the Brouwer fixed point theorem
is obtained by restricting the dimension to one.
In this case, the Brouwer fixed point theorem follows
almost immediately from the intermediate value theorem
(see \PMlinkname{this page}{BrouwerFixedPointInOneDimension}).
%%%%%
%%%%%
\end{document}
