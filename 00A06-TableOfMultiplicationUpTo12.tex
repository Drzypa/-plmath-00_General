\documentclass[12pt]{article}
\usepackage{pmmeta}
\pmcanonicalname{TableOfMultiplicationUpTo12}
\pmcreated{2013-03-22 16:36:14}
\pmmodified{2013-03-22 16:36:14}
\pmowner{PrimeFan}{13766}
\pmmodifier{PrimeFan}{13766}
\pmtitle{table of multiplication up to 12}
\pmrecord{6}{38799}
\pmprivacy{1}
\pmauthor{PrimeFan}{13766}
\pmtype{Data Structure}
\pmcomment{trigger rebuild}
\pmclassification{msc}{00A06}
\pmclassification{msc}{00A05}
\pmclassification{msc}{11B25}

\endmetadata

% this is the default PlanetMath preamble.  as your knowledge
% of TeX increases, you will probably want to edit this, but
% it should be fine as is for beginners.

% almost certainly you want these
\usepackage{amssymb}
\usepackage{amsmath}
\usepackage{amsfonts}

% used for TeXing text within eps files
%\usepackage{psfrag}
% need this for including graphics (\includegraphics)
%\usepackage{graphicx}
% for neatly defining theorems and propositions
%\usepackage{amsthm}
% making logically defined graphics
%%%\usepackage{xypic}

% there are many more packages, add them here as you need them

% define commands here

\begin{document}
Because of the commutative property of multiplication, it does not matter if the row or the column gives the first operand.

\begin{tabular}{|r|r|r|r|r|r|r|r|r|r|r|r|r|}
$\times$ & 1 & 2 & 3 & 4 & 5 & 6 & 7 & 8 & 9 & 10 & 11 & 12 \\
1 & 1 & 2 & 3 & 4 & 5 & 6 & 7 & 8 & 9 & 10 & 11 & 12 \\
2 & 2 & 4 & 6 & 8 & 10 & 12 & 14 & 16 & 18 & 20 & 22 & 24 \\
3 & 3 & 6 & 9 & 12 & 15 & 18 & 21 & 24 & 27 & 30 & 33 & 36 \\
4 & 4 & 8 & 12 & 16 & 20 & 24 & 28 & 32 & 36 & 40 & 44 & 48 \\
5 & 5 & 10 & 15 & 20 & 25 & 30 & 35 & 40 & 45 & 50 & 55 & 60 \\
6 & 6 & 12 & 18 & 24 & 30 & 36 & 42 & 48 & 54 & 60 & 66 & 72 \\
7 & 7 & 14 & 21 & 28 & 35 & 42 & 49 & 56 & 63 & 70 & 77 & 84 \\
8 & 8 & 16 & 24 & 32 & 40 & 48 & 56 & 64 & 72 & 80 & 88 & 96 \\
9 & 9 & 18 & 27 & 36 & 45 & 54 & 63 & 72 & 81 & 90 & 99 & 108 \\
10 & 10 & 20 & 30 & 40 & 50 & 60 & 70 & 80 & 90 & 100 & 110 & 120 \\
11 & 11 & 22 & 33 & 44 & 55 & 66 & 77 & 88 & 99 & 110 & 121 & 132 \\
12 & 12 & 24 & 36 & 48 & 60 & 72 & 84 & 96 & 108 & 120 & 132 & 144 \\
\end{tabular}

Obviously the longest northwest to southeast diagonal contains numbers of the form $n^2$.
%%%%%
%%%%%
\end{document}
