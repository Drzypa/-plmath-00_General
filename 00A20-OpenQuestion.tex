\documentclass[12pt]{article}
\usepackage{pmmeta}
\pmcanonicalname{OpenQuestion}
\pmcreated{2013-03-22 14:42:41}
\pmmodified{2013-03-22 14:42:41}
\pmowner{rspuzio}{6075}
\pmmodifier{rspuzio}{6075}
\pmtitle{open question}
\pmrecord{15}{36331}
\pmprivacy{1}
\pmauthor{rspuzio}{6075}
\pmtype{Definition}
\pmcomment{trigger rebuild}
\pmclassification{msc}{00A20}
\pmdefines{open problem}
\pmdefines{open}
\pmdefines{conjecture}

\endmetadata

% this is the default PlanetMath preamble.  as your knowledge
% of TeX increases, you will probably want to edit this, but
% it should be fine as is for beginners.

% almost certainly you want these
\usepackage{amssymb}
\usepackage{amsmath}
\usepackage{amsfonts}

% used for TeXing text within eps files
%\usepackage{psfrag}
% need this for including graphics (\includegraphics)
%\usepackage{graphicx}
% for neatly defining theorems and propositions
%\usepackage{amsthm}
% making logically defined graphics
%%%\usepackage{xypic}

% there are many more packages, add them here as you need them

% define commands here
\begin{document}
\PMlinkescapeword{between}
\PMlinkescapeword{force}
\PMlinkescapeword{term}
\PMlinkescapeword{terms}
\PMlinkescapeword{difference}
\PMlinkescapeword{even}

The adjective \emph{open} is used by mathematicians to a statement which has neither been proven to be true or to be false.  (In the light of incompleteness thorems, we should perhaps also add ``not been proven to be independent of the axioms'')

Examples:  

``It is an open question whether there are an infinite number of prime numbers all of whose digits are 1.'' means that it neither has been proven that there exist an infinity of such primes nor been proven that there are only a finite number of such primes.

``It is an open problem to determine the smallest number of lines which contain all points of the set $S$.'' means that the number in question has not yet been determined.

The term \emph{conjecture} refers to a statement which the speaker has reason to believe the statement is correct even though the speaker cannot prove the statement.  (Also, one should be warned about a common abuse of this term.  Even after a statement has been proven, sometimes it is still referred to as a conjecture by force of habit.)

The difference between the terms \emph{conjecture} and \emph{open question} is that the term \emph{open} is neutral --- saying that a statement is open does not connote that the speaker is voicing an opinion regarding the truth or falsity of the statement.

The attachment below gives some examples of famous open problems in mathematics.  \PMlinkexternal{Open Problem Garden}{http://garden.irmacs.sfu.ca/} is an external \PMlinkescapetext{link} that contains more examples of open problems in mathematics.
%%%%%
%%%%%
\end{document}
