\documentclass[12pt]{article}
\usepackage{pmmeta}
\pmcanonicalname{SumOfOddNumbers}
\pmcreated{2013-03-22 14:38:35}
\pmmodified{2013-03-22 14:38:35}
\pmowner{pahio}{2872}
\pmmodifier{pahio}{2872}
\pmtitle{sum of odd numbers}
\pmrecord{15}{36230}
\pmprivacy{1}
\pmauthor{pahio}{2872}
\pmtype{Example}
\pmcomment{trigger rebuild}
\pmclassification{msc}{00A05}
\pmclassification{msc}{11B25}
\pmrelated{NumberOdd}

% this is the default PlanetMath preamble.  as your knowledge
% of TeX increases, you will probably want to edit this, but
% it should be fine as is for beginners.

% almost certainly you want these
\usepackage{amssymb}
\usepackage{amsmath}
\usepackage{amsfonts}

% used for TeXing text within eps files
%\usepackage{psfrag}
% need this for including graphics (\includegraphics)
%\usepackage{graphicx}
% for neatly defining theorems and propositions
%\usepackage{amsthm}
% making logically defined graphics
%%%\usepackage{xypic}

% there are many more packages, add them here as you need them

% define commands here
\begin{document}
The sum of the first $n$ positive odd integers can be calculated by using the well-known \PMlinkescapetext{property} of the arithmetic progression, that the sum of its \PMlinkescapetext{terms} is equal to the arithmetic mean of the first and the last \PMlinkescapetext{term}, multiplied by the number of the \PMlinkescapetext{terms}:
 $$\underbrace{1+3+5+7+9+\cdots+(2n\!-\!1)}_{n} = 
     n\cdot\frac{1\!+\!(2n\!-\!1)}{2} = n^2$$
Thus, the sum of the first $n$ odd numbers is $n^2$ (this result has been proved first time in 1575 by Francesco Maurolico).

Below, the odd numbers have been set to form a triangle, each $n^{\rm{th}}$ row containing the next $n$ consecutive odd numbers.\, The arithmetic mean on the row is $n^2$ and the sum of its numbers is\, $n\cdot n^2 = n^3$.
\begin{eqnarray*}
\begin{array}{cccccccccccccccccc}
& & & & & & & & & 1 & & & & & & & &\\
& & & & & & & & 3 & & 5 & & & & & & &\\
& & & & & & & 7 & & 9 & & 11 & & & & & &\\
& & & & & & 13 & & 15 & & 17 & & 19 & & & & &\\
& & & & & 21 & & 23 & & 25 & & 27 & & 29 & & & &\\
& & & & 31 & & 33 & &35 & &37 & &39 & & 41 & & &\\
& & & & &\vdots & & & & \vdots & & & & \vdots& & & & \\
\end{array}
\end{eqnarray*}
%%%%%
%%%%%
\end{document}
