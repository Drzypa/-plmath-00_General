\documentclass[12pt]{article}
\usepackage{pmmeta}
\pmcanonicalname{QED}
\pmcreated{2013-03-22 12:40:14}
\pmmodified{2013-03-22 12:40:14}
\pmowner{mathwizard}{128}
\pmmodifier{mathwizard}{128}
\pmtitle{QED}
\pmrecord{8}{32945}
\pmprivacy{1}
\pmauthor{mathwizard}{128}
\pmtype{Definition}
\pmcomment{trigger rebuild}
\pmclassification{msc}{00A99}
\pmsynonym{Q.E.D}{QED}
\pmrelated{QEDInTheoreticalAndMathematicalPhysics}
\pmrelated{QCDOrQuantumChromodynamics}
\pmrelated{MathematicalFoundationsOfQuantumFieldTheories}
\pmrelated{QuantumOperatorAlgebrasInQuantumFieldTheories}
\pmrelated{GrassmanHopfAlgebrasAndTheirDualCoAlgebras}
\pmrelated{FoundationsOfQuantumFieldTheories}
\pmrelated{QuantumChromod}
\pmdefines{Halmos symbol}
\pmdefines{tombstone}
\pmdefines{Halmos tombstone}

\usepackage{amssymb}
\usepackage{amsmath}
\usepackage{amsfonts}

%\usepackage{psfrag}
%\usepackage{graphicx}
%%%\usepackage{xypic}
\begin{document}
The term ``QED'' is actually an abbreviation and stands for the Latin \emph{quod erat demonstrandum}, meaning ``which was to be demonstrated.''

QED typically is used to signify the end of a mathematical proof.  The symbol 

$$ \square $$

is often used in place of ``QED,'' and is called the ``tombstone'', ``Halmos symbol'' or ``Halmos tombstone'' after mathematician Paul Halmos (it can vary in width, however, and sometimes it is fully or partially shaded).  Halmos borrowed this symbol from magazines, where it was used to denote ``end of article''.
%%%%%
%%%%%
\end{document}
