\documentclass[12pt]{article}
\usepackage{pmmeta}
\pmcanonicalname{Introducing0thPower}
\pmcreated{2013-03-22 13:24:20}
\pmmodified{2013-03-22 13:24:20}
\pmowner{mathcam}{2727}
\pmmodifier{mathcam}{2727}
\pmtitle{introducing 0th power}
\pmrecord{8}{33948}
\pmprivacy{1}
\pmauthor{mathcam}{2727}
\pmtype{Topic}
\pmcomment{trigger rebuild}
\pmclassification{msc}{00A05}
\pmrelated{EmptyProduct}

\endmetadata

% this is the default PlanetMath preamble.  as your knowledge
% of TeX increases, you will probably want to edit this, but
% it should be fine as is for beginners.

% almost certainly you want these
\usepackage{amssymb}
\usepackage{amsmath}
\usepackage{amsfonts}

% used for TeXing text within eps files
%\usepackage{psfrag}
% need this for including graphics (\includegraphics)
%\usepackage{graphicx}
% for neatly defining theorems and propositions
%\usepackage{amsthm}
% making logically defined graphics
%%%\usepackage{xypic}

% there are many more packages, add them here as you need them

% define commands here
\begin{document}
Let $a$ be a number not equal to zero. Then for all $n \in \mathbb{N}$, we have that $a^n$ is the product of $a$ with itself $n$ \PMlinkescapetext{times}. Using the fact that the integer 1 is a multiplicative identity, ($a\cdot 1=a$ for any $a$), we can write
\begin{displaymath}
a^n \cdot 1=a^n=a^{n+0}=a^n\cdot a^0,
\end{displaymath}
where we have used the properties of exponents under multiplication.  Now, after canceling a factor of $a^n$ from both sides of the above equation, we derive that $a^0=1$ for any non-zero number.
%%%%%
%%%%%
\end{document}
