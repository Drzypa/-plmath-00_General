\documentclass[12pt]{article}
\usepackage{pmmeta}
\pmcanonicalname{MacOSCalculator}
\pmcreated{2013-03-22 16:39:26}
\pmmodified{2013-03-22 16:39:26}
\pmowner{PrimeFan}{13766}
\pmmodifier{PrimeFan}{13766}
\pmtitle{Mac OS Calculator}
\pmrecord{6}{38862}
\pmprivacy{1}
\pmauthor{PrimeFan}{13766}
\pmtype{Definition}
\pmcomment{trigger rebuild}
\pmclassification{msc}{00A05}
\pmclassification{msc}{01A07}
\pmdefines{Mac OS 9 Calculator}
\pmdefines{Mac OS X Calculator}

\endmetadata

% this is the default PlanetMath preamble.  as your knowledge
% of TeX increases, you will probably want to edit this, but
% it should be fine as is for beginners.

% almost certainly you want these
\usepackage{amssymb}
\usepackage{amsmath}
\usepackage{amsfonts}

% used for TeXing text within eps files
%\usepackage{psfrag}
% need this for including graphics (\includegraphics)
%\usepackage{graphicx}
% for neatly defining theorems and propositions
%\usepackage{amsthm}
% making logically defined graphics
%%%\usepackage{xypic}

% there are many more packages, add them here as you need them

% define commands here

\begin{document}
The {\em Mac OS Calculator} is a software calculator that comes bundled with the Apple Mac OS operating system. As late as Mac OS 9, the Calculator was a very basic calculator with only the arithmetic operations. In Mac OS X, the Calculator program was upgraded, with not just the addition of Scientific mode (a scientific calculator) but also Programmer mode with functions useful for computer programmers, such as Unicode character lookup by numerical value. The currently displayed value is not lost at a change of mode.

Like the Windows Calculator, for the Mac OS Calculator $0^0 = 1$.

\begin{thebibliography}{1}
\bibitem{ml} Maria Langer, {\it Mac OS 8: Visual Quickstart Guide} Berkeley: Peachpit Press (1997): 108
\end{thebibliography}
%%%%%
%%%%%
\end{document}
