\documentclass[12pt]{article}
\usepackage{pmmeta}
\pmcanonicalname{Expression}
\pmcreated{2013-03-22 17:50:11}
\pmmodified{2013-03-22 17:50:11}
\pmowner{Wkbj79}{1863}
\pmmodifier{Wkbj79}{1863}
\pmtitle{expression}
\pmrecord{7}{40307}
\pmprivacy{1}
\pmauthor{Wkbj79}{1863}
\pmtype{Definition}
\pmcomment{trigger rebuild}
\pmclassification{msc}{00-01}
\pmrelated{Equation}
\pmdefines{sum expression}

\usepackage{amssymb}
\usepackage{amsmath}
\usepackage{amsfonts}
\usepackage{pstricks}
\usepackage{psfrag}
\usepackage{graphicx}
\usepackage{amsthm}
%%\usepackage{xypic}

\begin{document}
An \emph{expression} is a symbol or \PMlinkescapetext{combination} of symbols used to denote a quantity or value.  Expressions consist of constants, variables, operations, operators, functions, and parentheses.\\

\textbf{Note 1.}  If an expression \PMlinkescapetext{contains} one or more operations to be performed in a certain \PMlinkname{order}{OrderOfOperations}, the expression may be named after the last (\PMlinkname{i.e.}{Ie} outermost) operation.  For example, the expression $\displaystyle a^2-5\sqrt{a}+\frac{2}{3a}$ is a {\em sum expression}.\\

\textbf{Note 2.}  An equation is a denoted equality of two expressions.

\begin{thebibliography}{9}
\bibitem{dict} \emph{Dictionary.com Unabridged} (version 1.1).  Accessed on February 22, 2008.  URL: \PMlinkexternal{http://dictionary.reference.com/browse/expression}{http://dictionary.reference.com/browse/expression}
\end{thebibliography}
%%%%%
%%%%%
\end{document}
