\documentclass[12pt]{article}
\usepackage{pmmeta}
\pmcanonicalname{OnLineEncyclopediaOfIntegerSequences}
\pmcreated{2013-03-22 15:47:36}
\pmmodified{2013-03-22 15:47:36}
\pmowner{CompositeFan}{12809}
\pmmodifier{CompositeFan}{12809}
\pmtitle{On-Line Encyclopedia of Integer Sequences}
\pmrecord{11}{37754}
\pmprivacy{1}
\pmauthor{CompositeFan}{12809}
\pmtype{Definition}
\pmcomment{trigger rebuild}
\pmclassification{msc}{00A20}
\pmsynonym{OEIS}{OnLineEncyclopediaOfIntegerSequences}
\pmsynonym{Online Encyclopedia of Integer Sequences}{OnLineEncyclopediaOfIntegerSequences}

\endmetadata

% this is the default PlanetMath preamble.  as your knowledge
% of TeX increases, you will probably want to edit this, but
% it should be fine as is for beginners.

% almost certainly you want these
\usepackage{amssymb}
\usepackage{amsmath}
\usepackage{amsfonts}

% used for TeXing text within eps files
%\usepackage{psfrag}
% need this for including graphics (\includegraphics)
%\usepackage{graphicx}
% for neatly defining theorems and propositions
%\usepackage{amsthm}
% making logically defined graphics
%%%\usepackage{xypic}

% there are many more packages, add them here as you need them

% define commands here
\begin{document}
\PMlinkescapeword{core}

An Internet database of sequences of integers, together with formulas, generating functions, computer programs, keywords and cross-references to other sequences. Sometimes referred to by its acronym, {\em OEIS}. It was started by Neil Sloane as an outgrowth of his printed catalogs of integer sequences.

Each sequence is assigned a unique identification number (the letter A followed by six digits in base 10) based on the date of its addition to the database, though most entries contain information on their lexicographic context. In 2010 the database had almost 200000 sequences.

The most important sequences are tagged with the keyword ``core." Sequences of fractions are split into two sequences (numerators and denominators), while important mathematical constants like e are split into their base 10 digits. Zero is sometimes used as a placeholder value for non-existent terms.

A few other important keywords include: ``frac" for most fractions, ``cofr" for continued fractions, ``eigen" for eigenvalues, ``mult" for multiplicative sequences, ``walk" for sequences pertaining to self-avoiding paths, etc.

The sequences included in the database range in subject from Sloane's love of combinatorics to number theory, geometry, calculus, cryptography, quantum physics, etc., as well as many subjects that would seem to be far removed from mathematics, like archaeology and psychiatry.

In 2010 there was an effort to move the whole database to a Wiki platform, but this ultimately abandoned in favor of a custom-designed platform with a more rigorous editorial control. However, the OEIS does have a Wiki for subsidiary pages.

\PMlinkexternal{The On-Line Encyclopedia of Integer Sequences}{http://oeis.org/}


\begin{thebibliography}{6}
\bibitem{ad} A. Del Arte, ``\PMlinkexternal{Mathematician reaches 100k milestone of online integer archive}{http://www.southend.wayne.edu/modules/news/article.php?storyid=553}" {\it The South End} (Detroit), 11 Nov., Wayne State University (2004)
\bibitem{vo} V. Origlio, ``L'Enciclopedia delle sequenze intere" {\it Biblioteche Oggi}, Jan.-Feb. (2006): 41 - 45
\end{thebibliography}
%%%%%
%%%%%
\end{document}
