\documentclass[12pt]{article}
\usepackage{pmmeta}
\pmcanonicalname{IndexOfTables}
\pmcreated{2013-03-22 18:07:31}
\pmmodified{2013-03-22 18:07:31}
\pmowner{PrimeFan}{13766}
\pmmodifier{PrimeFan}{13766}
\pmtitle{index of tables}
\pmrecord{8}{40675}
\pmprivacy{1}
\pmauthor{PrimeFan}{13766}
\pmtype{Topic}
\pmcomment{trigger rebuild}
\pmclassification{msc}{00A20}

\endmetadata

% this is the default PlanetMath preamble.  as your knowledge
% of TeX increases, you will probably want to edit this, but
% it should be fine as is for beginners.

% almost certainly you want these
\usepackage{amssymb}
\usepackage{amsmath}
\usepackage{amsfonts}

% used for TeXing text within eps files
%\usepackage{psfrag}
% need this for including graphics (\includegraphics)
%\usepackage{graphicx}
% for neatly defining theorems and propositions
%\usepackage{amsthm}
% making logically defined graphics
%%%\usepackage{xypic}

% there are many more packages, add them here as you need them

% define commands here

\begin{document}
Since early in the history of mathematics, tables have served to aid in various computational tasks. Even today, with the ready availability of powerful calculators, tables still serve an important pedagogical purpose. Tables are found in the main text of mathematical papers and books, to illustrate the concepts in question, and as the back matter of books.

\section{Elementary tables}

The most elementary tables in mathematics are those for the four basic operations: addition, subtraction, multiplication and division. In the old days, schoolchildren had to learn these up to 12. Nowadays teachers stop at 10.

\begin{itemize}
\item Table of addition up to 12
\item Table of subtraction up to 12
\item Table of multiplication up to 12
\item Table of division up to 12
\end{itemize}

Before the age of computers, tables of logarithms were of paramount importance to most scientists, as well as tables of square roots.

\section{Logical tables}

Truth tables can be made for some fairly complicated expressions, but the ones of most use to computer programming students are those for the basic expressions, a truth table for logical AND, a truth table for logical OR, a truth table for logical XOR, etc. These can be expressed as binary operations by simply replacing TRUEs with 1s and FALSEs with 0s.

\section{Combinatorics tables}

Some tables have garnered interest beyond just looking up values to become classics. This is perhaps truest of Pascal's triangle, which has many interesting properties besides being a convenient way to look up binomial coefficients without having to compute large factorials.

\section{Number theory tables}

Books about prime numbers can be reliably counted upon to list the first thousand positive prime numbers, or perhaps the first ten thousand, but certainly the first hundred. A book about prime numbers, or a book about number theory in general, might have as an appendix a table of integer factorizations for $0 < n < 1001$. A table of Mersenne primes generally gives the exponent rather than writing out the prime in base 10 since these numbers get very large very quickly. A book specifically on Mersenne primes might also have a table of factors of small Mersenne numbers. The Fermat numbers grow large even more quickly, and thus a table of factors of Fermat numbers is even less likely to write out Fermat numbers.

Similarly to a book about prime numbers, a book on the Fibonacci numbers might have a list of Fibonacci numbers (the appendices of Koshy's book on the subject are actually tables of the factorizations of the Fibonacci and Lucas numbers).

For functions with a very small range of possible output values, such as the M\"obius function $-2 < \mu(n) < 2$ or the Liouville function $| \lambda(n) | = 1$, it makes sense for a table to pair them up with their matching summatory functions. Thus we have a table of values of the M\"obius function and the Mertens function and a table of values of the Liouville function and its summatory function.

There are also tables using primitive roots and index to solve congruences in relation to specific roots and indices.\, A pair of tables concern solutions of Diophantine equations: table of integer contraharmonic means and table of integer harmonic means.

In the study of global fields, a table of some fundamental units and a table of class numbers of imaginary quadratic fields is a useful aid to the student.

Moving on to the complex plane, just a few complex multiplication tables would be sufficient, as it would be easier for the student to learn the formula for complex multiplication rather than to try to memorize some multidimensional table which would be too large even if limited to a small range.

\section{Geometry}

To illustrate the Pythagorean theorem, older books might have an appendix of the first primitive Pythagorean triplets. Tables of values of the sine, cosine and tangent functions would not be out of place even in a basic geometry book.

\section{Reference tables}

For reference purposes, there are tables where one can look up the specific value of a real number or a sequence of numbers. For example, an index of important irrational constants such as Borwein's dictionary of real numbers. For sequences of integers, there are the books by Neil Sloane, the {\it Handbook of Integer Sequences} and the {\it Encyclopedia of Integer Sequences}, both predecessors to the Online Encyclopedia of Integer Sequences.
%%%%%
%%%%%
\end{document}
