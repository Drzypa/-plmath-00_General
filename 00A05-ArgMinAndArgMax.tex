\documentclass[12pt]{article}
\usepackage{pmmeta}
\pmcanonicalname{ArgMinAndArgMax}
\pmcreated{2013-03-22 14:27:55}
\pmmodified{2013-03-22 14:27:55}
\pmowner{kshum}{5987}
\pmmodifier{kshum}{5987}
\pmtitle{arg min and arg max}
\pmrecord{11}{35986}
\pmprivacy{1}
\pmauthor{kshum}{5987}
\pmtype{Definition}
\pmcomment{trigger rebuild}
\pmclassification{msc}{00A05}
\pmdefines{argmin argmax}

\endmetadata

% this is the default PlanetMath preamble.  as your knowledge
% of TeX increases, you will probably want to edit this, but
% it should be fine as is for beginners.

% almost certainly you want these
\usepackage{amssymb}
\usepackage{amsmath}
%\usepackage{amsfonts}

% used for TeXing text within eps files
%\usepackage{psfrag}
% need this for including graphics (\includegraphics)
%\usepackage{graphicx}
% for neatly defining theorems and propositions
%\usepackage{amsthm}
% making logically defined graphics
%%%\usepackage{xypic}

% there are many more packages, add them here as you need them

% define commands here
\begin{document}
For a real-valued function $f$ with domain $S$, $\arg \min_{x \in S} f(x)$ is the set of elements in $S$ that achieve the global minimum in $S$,
\[
 {\arg \min}_{x \in S} f(x) = \{ x \in S :\, f(x) = \min_{y\in S} f(y) \}.
\]

$\arg \max_{x \in S} f(x)$ is the set of elements in $S$ that achieve the global maximum in $S$,
\[
 {\arg \max}_{x \in S} f(x) = \{ x \in S :\, f(x) = \max_{y\in S} f(y) \}.
\]



%%%%%
%%%%%
\end{document}
