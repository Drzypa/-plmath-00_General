\documentclass[12pt]{article}
\usepackage{pmmeta}
\pmcanonicalname{TriceSequence}
\pmcreated{2013-03-22 16:45:00}
\pmmodified{2013-03-22 16:45:00}
\pmowner{PrimeFan}{13766}
\pmmodifier{PrimeFan}{13766}
\pmtitle{Trice sequence}
\pmrecord{5}{38975}
\pmprivacy{1}
\pmauthor{PrimeFan}{13766}
\pmtype{Definition}
\pmcomment{trigger rebuild}
\pmclassification{msc}{00A99}

% this is the default PlanetMath preamble.  as your knowledge
% of TeX increases, you will probably want to edit this, but
% it should be fine as is for beginners.

% almost certainly you want these
\usepackage{amssymb}
\usepackage{amsmath}
\usepackage{amsfonts}

% used for TeXing text within eps files
%\usepackage{psfrag}
% need this for including graphics (\includegraphics)
%\usepackage{graphicx}
% for neatly defining theorems and propositions
%\usepackage{amsthm}
% making logically defined graphics
%%%\usepackage{xypic}

% there are many more packages, add them here as you need them

% define commands here

\begin{document}
The {\em Trice sequence} consists of the nine numbers 21, 36, 55, 60, 67, 68, 92, 93, 125 in ascending order. It was published in Clifford Pickover's {\it Mazes of the Mind} in 1992 and added to the On-Line Encyclopedia of Integer Sequences early on in its history (see A001491), but for years people couldn't figure out what the sequence was because most of them were searching for a mathematical explanation. The first three numbers are both triangular and hexagonal, but none of the following numbers are. According to Neil Sloane, ``this was a mystery for many years; even Clifford Pickover could not recall the explanation .'' It wasn't until 1998 that Derek Holt identified the sequence and let Sloane know that it consists of the opus numbers of Beethoven's nine symphonies.
%%%%%
%%%%%
\end{document}
