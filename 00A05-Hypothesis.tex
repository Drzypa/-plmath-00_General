\documentclass[12pt]{article}
\usepackage{pmmeta}
\pmcanonicalname{Hypothesis}
\pmcreated{2013-03-22 17:15:18}
\pmmodified{2013-03-22 17:15:18}
\pmowner{PrimeFan}{13766}
\pmmodifier{PrimeFan}{13766}
\pmtitle{hypothesis}
\pmrecord{6}{39590}
\pmprivacy{1}
\pmauthor{PrimeFan}{13766}
\pmtype{Definition}
\pmcomment{trigger rebuild}
\pmclassification{msc}{00A05}

\endmetadata

% this is the default PlanetMath preamble.  as your knowledge
% of TeX increases, you will probably want to edit this, but
% it should be fine as is for beginners.

% almost certainly you want these
\usepackage{amssymb}
\usepackage{amsmath}
\usepackage{amsfonts}

% used for TeXing text within eps files
%\usepackage{psfrag}
% need this for including graphics (\includegraphics)
%\usepackage{graphicx}
% for neatly defining theorems and propositions
%\usepackage{amsthm}
% making logically defined graphics
%%%\usepackage{xypic}

% there are many more packages, add them here as you need them

% define commands here

\begin{document}
\PMlinkescapeword{support}
\PMlinkescapeword{field}
\PMlinkescapeword{label}

In mathematics, a {\em hypothesis} is an unproven statement which is supported by all the available data and by many weaker results. An unproven mathematical statement is usually called a ``conjecture'', and while experimentation can sometimes produce millions of examples to support a conjecture, usually nothing short of a proof can convince experts in the field. But when a conjecture is supported not only but all the available data but also by numerous weaker results, it is upgraded in label to a hypothesis. The most famous conjecture in mathematics is the Riemann hypothesis, which despite many attempts at a proof, is supported by many related results. The convexity conjecture, on the other hand, is considered ``incompatible'' with the $n$-tuples conjecture  and more results appear to support the latter, thus neither is upgraded to hypothesis.

\begin{thebibliography}{1}
\bibitem{rc} R. Crandall \& C. Pomerance, {\it Prime Numbers: A Computational Perspective}, Springer, NY, 2001: 1.2.4
\end{thebibliography}
%%%%%
%%%%%
\end{document}
