\documentclass[12pt]{article}
\usepackage{pmmeta}
\pmcanonicalname{ComplexSystemsBiologyCSB}
\pmcreated{2013-03-22 18:11:43}
\pmmodified{2013-03-22 18:11:43}
\pmowner{bci1}{20947}
\pmmodifier{bci1}{20947}
\pmtitle{complex systems biology (CSB)}
\pmrecord{44}{40772}
\pmprivacy{1}
\pmauthor{bci1}{20947}
\pmtype{Topic}
\pmcomment{trigger rebuild}
\pmclassification{msc}{00A30}
\pmclassification{msc}{18A40}
\pmclassification{msc}{37F99}
\pmclassification{msc}{11Y16}
\pmclassification{msc}{18A05}
\pmclassification{msc}{03D15}
\pmsynonym{systems biology}{ComplexSystemsBiologyCSB}
\pmsynonym{CSB}{ComplexSystemsBiologyCSB}
\pmsynonym{abstract relational biology}{ComplexSystemsBiologyCSB}
%\pmkeywords{complex systems biology}
%\pmkeywords{Categorical Ontology of higher complexity levels}
\pmrelated{Category}
\pmrelated{SystemDefinitions}
\pmrelated{CategoryOfMRSystems3}
\pmrelated{CategoricalOntology}
\pmrelated{OrganismicSets2}
\pmrelated{FunctionalBiology}
\pmrelated{RosettaGroupoids}
\pmrelated{NaturalTransformationsOfOrganismicStructures}
\pmrelated{MolecularSetVariable}
\pmrelated{CategoryOfMolecularSets}
\pmrelated{SupercategoryOfVariableMolecularSets}
\pmrelated{Mat}
\pmdefines{categorical ontology of levels}
\pmdefines{complex system biology modeling and ontology}
\pmdefines{CSB}

% this is the default PlanetMath preamble. 
\usepackage{amssymb}
\usepackage{amsmath}
\usepackage{amsfonts}

% define commands here
\usepackage{amsmath, amssymb, amsfonts, amsthm, amscd, latexsym}
%%\usepackage{xypic}
\usepackage[mathscr]{eucal}

\setlength{\textwidth}{6.5in}
%\setlength{\textwidth}{16cm}
\setlength{\textheight}{9.0in}
%\setlength{\textheight}{24cm}

\hoffset=-.75in %%ps format
%\hoffset=-1.0in %%hp format
\voffset=-.4in

\theoremstyle{plain}
\newtheorem{lemma}{Lemma}[section]
\newtheorem{proposition}{Proposition}[section]
\newtheorem{theorem}{Theorem}[section]
\newtheorem{corollary}{Corollary}[section]

\theoremstyle{definition}
\newtheorem{definition}{Definition}[section]
\newtheorem{example}{Example}[section]
%\theoremstyle{remark}
\newtheorem{remark}{Remark}[section]
\newtheorem*{notation}{Notation}
\newtheorem*{claim}{Claim}

\renewcommand{\thefootnote}{\ensuremath{\fnsymbol{footnote%%@
}}}
\numberwithin{equation}{section}

\newcommand{\Ad}{{\rm Ad}}
\newcommand{\Aut}{{\rm Aut}}
\newcommand{\Cl}{{\rm Cl}}
\newcommand{\Co}{{\rm Co}}
\newcommand{\DES}{{\rm DES}}
\newcommand{\Diff}{{\rm Diff}}
\newcommand{\Dom}{{\rm Dom}}
\newcommand{\Hol}{{\rm Hol}}
\newcommand{\Mon}{{\rm Mon}}
\newcommand{\Hom}{{\rm Hom}}
\newcommand{\Ker}{{\rm Ker}}
\newcommand{\Ind}{{\rm Ind}}
\newcommand{\IM}{{\rm Im}}
\newcommand{\Is}{{\rm Is}}
\newcommand{\ID}{{\rm id}}
\newcommand{\GL}{{\rm GL}}
\newcommand{\Iso}{{\rm Iso}}
\newcommand{\Sem}{{\rm Sem}}
\newcommand{\St}{{\rm St}}
\newcommand{\Sym}{{\rm Sym}}
\newcommand{\SU}{{\rm SU}}
\newcommand{\Tor}{{\rm Tor}}
\newcommand{\U}{{\rm U}}

\newcommand{\A}{\mathcal A}
\newcommand{\Ce}{\mathcal C}
\newcommand{\D}{\mathcal D}
\newcommand{\E}{\mathcal E}
\newcommand{\F}{\mathcal F}
\newcommand{\G}{\mathcal G}
\newcommand{\Q}{\mathcal Q}
\newcommand{\R}{\mathcal R}
\newcommand{\cS}{\mathcal S}
\newcommand{\cU}{\mathcal U}
\newcommand{\W}{\mathcal W}

\newcommand{\bA}{\mathbb{A}}
\newcommand{\bB}{\mathbb{B}}
\newcommand{\bC}{\mathbb{C}}
\newcommand{\bD}{\mathbb{D}}
\newcommand{\bE}{\mathbb{E}}
\newcommand{\bF}{\mathbb{F}}
\newcommand{\bG}{\mathbb{G}}
\newcommand{\bK}{\mathbb{K}}
\newcommand{\bM}{\mathbb{M}}
\newcommand{\bN}{\mathbb{N}}
\newcommand{\bO}{\mathbb{O}}
\newcommand{\bP}{\mathbb{P}}
\newcommand{\bR}{\mathbb{R}}
\newcommand{\bV}{\mathbb{V}}
\newcommand{\bZ}{\mathbb{Z}}

\newcommand{\bfE}{\mathbf{E}}
\newcommand{\bfX}{\mathbf{X}}
\newcommand{\bfY}{\mathbf{Y}}
\newcommand{\bfZ}{\mathbf{Z}}

\renewcommand{\O}{\Omega}
\renewcommand{\o}{\omega}
\newcommand{\vp}{\varphi}
\newcommand{\vep}{\varepsilon}

\newcommand{\diag}{{\rm diag}}
\newcommand{\grp}{{\mathbb G}}
\newcommand{\dgrp}{{\mathbb D}}
\newcommand{\desp}{{\mathbb D^{\rm{es}}}}
\newcommand{\Geod}{{\rm Geod}}
\newcommand{\geod}{{\rm geod}}
\newcommand{\hgr}{{\mathbb H}}
\newcommand{\mgr}{{\mathbb M}}
\newcommand{\ob}{{\rm Ob}}
\newcommand{\obg}{{\rm Ob(\mathbb G)}}
\newcommand{\obgp}{{\rm Ob(\mathbb G')}}
\newcommand{\obh}{{\rm Ob(\mathbb H)}}
\newcommand{\Osmooth}{{\Omega^{\infty}(X,*)}}
\newcommand{\ghomotop}{{\rho_2^{\square}}}
\newcommand{\gcalp}{{\mathbb G(\mathcal P)}}

\newcommand{\rf}{{R_{\mathcal F}}}
\newcommand{\glob}{{\rm glob}}
\newcommand{\loc}{{\rm loc}}
\newcommand{\TOP}{{\rm TOP}}

\newcommand{\wti}{\widetilde}
\newcommand{\what}{\widehat}

\renewcommand{\a}{\alpha}
\newcommand{\be}{\beta}
\newcommand{\ga}{\gamma}
\newcommand{\Ga}{\Gamma}
\newcommand{\de}{\delta}
\newcommand{\del}{\partial}
\newcommand{\ka}{\kappa}
\newcommand{\si}{\sigma}
\newcommand{\ta}{\tau}
\newcommand{\med}{\medbreak}
\newcommand{\medn}{\medbreak \noindent}
\newcommand{\bign}{\bigbreak \noindent}
\newcommand{\lra}{{\longrightarrow}}
\newcommand{\ra}{{\rightarrow}}
\newcommand{\rat}{{\rightarrowtail}}
\newcommand{\oset}[1]{\overset {#1}{\ra}}
\newcommand{\osetl}[1]{\overset {#1}{\lra}}
\newcommand{\hr}{{\hookrightarrow}}

\begin{document}
\subsection{Introduction}

 {\em Complex systems biology ($CSB$)} is generally described as a non-reductionist, mathematical theory
of emergent living organisms or biosystems in terms of a network, graph or category of integrated interactions between their structural and functional components or subsystems. This is often abbreviated to 
\PMlinkexternal{systems biology}{http://planetphysics.org/?op=getobj&from=books&id=248} in entries that should be described in fact as {\em complex systems biology}. Notably, several mathematical physicists or mathematicians,
such as von Neumann believed that all complex systems can be ultimately `decomposed' or disassembled into their simpler, physical components, whereas others, such as Elsasser argued that the heterogeneous logical class of biosystems makes them irreducible to their physical components of logically homogeneous classes; the latter view was
also shared by Robert Rosen who also produced encoding and dynamic reasons for which reductionism would not work
for biosystems. Thus, it would seem that there is a fundamental, physical and mathematical controvercy regarding the
essential nature of Life. Resolution of this fundamental controvercy in terms of mathematics is denied by many biologists who argue that the dynamics and physiology of biosystems is not formalizable in either mathematical terms or physical theory. The key question: {\em ``What is Life?''} is also the title of the widely-read book published in 1945 by the famous quantum theoretician Erwin $Schr\"odinger$, Nobel laureate and inventor of the equation that carries his name. 

 To address this fundamental question of life, applied mathematicians, as well as mathematical and theoretical biologists have been developing for over a century precise mathematical models of biosystems or organisms of increasing sophistication and generality. One can select the birth of cybernetics, biocybernetics, and the application of category theory, as well as set theory, to biosystems as the starting point of $CSB$ which is currently the branch of inter-disciplinary science, between mathematics and biology, as well as sociology, that aims to define in precise, mathematical terms the nature of dynamic and organizational complexity both in living organisms and in societies. 
The famous topologist and Fields medalist Ren\'e Thom was one of the more recent contributors to this field from the
viewpoint of topology and Poincar\'e' s Qualitative Dynamics. Grothendieck is also said to have a keen interest in such complexity problems related to living organisms. Defining the main problems and approaches in $CSB$ remains 
a monumental task for multi-disciplinary teams of applied mathematicians, biologists, biochemists, physicists, biophysicists, sociologists, computer scientists, and so on. The underlying logical problems are also formidable
as most complexity problems do not have easy, or simple, Boolean logic solutions, 
and are not amenable to linear engineering analysis, either direct or reverse. 


\subsection{Complex systems biology} 
\begin{definition}
 The \emph{categorical ontology theory of levels} is often defined as the classification theory in ontology, or the theory of existence of items (or objects--defined in the mathematical or logical sense) by means of the mathematical theory of categories into three levels of dynamic systems pertaining to: the physical/chemical level, the biological level, and the psychological level (or human mind). Connections between the three levels of reality and their transformations are represented, respectively, by morphisms/functors and natural transformations defined for categories of molecular sets, 
\PMlinkname{categories of $(M,R)$-systems}{CategoryOfMRSystems3} and organismic supercategories.
\end{definition}

 From a categorical ontology theory of levels viewpoint, however, the term complex may appear to be misplaced because {\em systems with chaos}, or chaotic dynamics, are currently defined by physicists as {\em `complex systems'}, which may have placed a role in the emergence of living systems that are, in fact, {\em super-complex}. Therefore, the more appropriate classification of this relatively new area in mathematical or theoretical biology and Biophysics is 
super-complex systems biology, $s$-complex systems biology, or simply ``systems biology''--as a more general approach where the focus may be not on the super-complexity aspects of living systems but on computer modeling of physiological, or functional genomics, integration of physiological flows, signaling pathways or interactomics.  However, unlike the case of purely functional $(M,R)$-systems theory in abstract relational biology (ARB), complex systems biology (or systems biology) proponents are primarily concerned with the integration of data from a multitude of bioinformatics and genomic/proteomic/post-genomic (primarily structural) data; $CSB$ scientists also aim to study {\em interactomics} or {\em metabolomics} primarily through computer-based data analysis, and often Bayesian-based attempts at integration.  branches of mathematics that find applications in $CSB$ are, for example: computer modeling, colored graphs, graph-theoretical based approaches, biotopology, genetic, metabolic and signaling network theories, Bayesian models, biostatistics, correlation techniques, and less frequently: abstract algebra, group theory, groupoid and category theory modeling of cell-cell interactions and biodynamics.

%%%%%
%%%%%
\end{document}
