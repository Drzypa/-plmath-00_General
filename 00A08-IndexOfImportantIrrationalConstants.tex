\documentclass[12pt]{article}
\usepackage{pmmeta}
\pmcanonicalname{IndexOfImportantIrrationalConstants}
\pmcreated{2013-03-22 17:03:10}
\pmmodified{2013-03-22 17:03:10}
\pmowner{PrimeFan}{13766}
\pmmodifier{PrimeFan}{13766}
\pmtitle{index of important irrational constants}
\pmrecord{17}{39343}
\pmprivacy{1}
\pmauthor{PrimeFan}{13766}
\pmtype{Topic}
\pmcomment{trigger rebuild}
\pmclassification{msc}{00A08}

\endmetadata

% this is the default PlanetMath preamble.  as your knowledge
% of TeX increases, you will probably want to edit this, but
% it should be fine as is for beginners.

% almost certainly you want these
\usepackage{amssymb}
\usepackage{amsmath}
\usepackage{amsfonts}

% used for TeXing text within eps files
%\usepackage{psfrag}
% need this for including graphics (\includegraphics)
%\usepackage{graphicx}
% for neatly defining theorems and propositions
%\usepackage{amsthm}
% making logically defined graphics
%%%\usepackage{xypic}

% there are many more packages, add them here as you need them

% define commands here

\begin{document}
The following table lists some of the most important irrational constants in mathematics.

Of course importance is sometimes debatable. Hardly anyone disputes the importance of $\pi$ or $e$ (in fact, these are the only two constants in the OEIS to have the keyword ``core'' attached to them), but for other constants it is not quite clear cut. In general, if a given constant has a name (especially a name hyphenating two famous mathematicians' last names) I consider it important.

Irrationality is not always clear cut either, e.g., it might be a mistake to exclude the Euler-Mascheroni constant $\gamma$ from this list.

The constants are given to 20 decimal places.

\begin{tabular}{|r|l|}
0.1149420448532962007 & Kepler-Bouwkamp constant or polygon-inscribing constant \\
0.1234567891011121314 & Champernowne's constant $C_{10}$ \\
0.2078795763507619085 & $i^i$ (has no imaginary part) or $e^{\frac{-\pi}{2}}$ \\
0.2357111317192329313 & \PMlinkname{Copeland-Erd\H{o}s constant}{CopelandErdsConstant} \\
0.2614972128476427837 & Meissel-Mertens constant \\
0.3275822918721811159 & L\'evy's constant \\
0.4146825098511116602 & The prime constant $\rho$ \\
0.5926327182016361971 & Lehmer's constant \\
0.6079271018540266286 & ${6 \over {\pi^2}}$, the probability that a random integer is squarefree \\
0.6434105462883380261 & Cahen's constant \\
0.7642236535892206629 & Landau-Ramanujan constant \\
0.8346268416740731862 & Gauss's constant \\
0.8862269254527580136 & $\Gamma(\frac{3}{2}) = \frac{1}{2} \sqrt{\pi}$ \\
0.9159655941772190150 & Catalan's constant $K$ \\
1.2020569031595942853 & Ap\'ery's constant $\zeta(3)$ \\
1.2254167024651776451 & $\Gamma(\frac{3}{4})$ \\
1.3063778838630806904 & Mills' constant \\
1.3247179572447460260 & The plastic constant \\
1.4142135623730950488 & Square root of two $\sqrt{2}$ \\
1.4513692348833810502 & Ramanujan-Soldner constant \\
1.6066951524152917637 & Erd\H{o}s-Borwein constant \\
1.6180339887498948482 & The golden ratio $\phi$ \\
1.6449340668482264364 & $\zeta(2) = \frac{\pi^2}{6}$, the solution to the Basel problem \\
1.7320508075688772935 & Square root of three $\sqrt{3}$ \\
1.7579327566180045327 & Vijayaraghavan's infinite nested radical $\sqrt{1 + \sqrt{2 + \sqrt{3 + \sqrt{4 + \sqrt{5 + \ldots }}}}}$ \\
1.7724538509055160273 & $\Gamma(\frac{1}{2}) = \sqrt{\pi}$ \\
2.2360679774997896964 & Square root of five $\sqrt{5}$ \\
2.6651441426902251887 & $2^{\sqrt{2}}$ \\
2.6854520010653064453 & Khinchin's constant \\
2.4142135623730950488 & The silver ratio $\delta_{S}$ \\
2.5849817595792532170 & Sierpi\'nski's constant \\
2.7182818284590452354 & The natural log base $e$ \\
3.1415926535897932385 & The ratio of a circle's radius to its circumference $\pi$ \\
3.6256099082219083119 & $\Gamma(\frac{1}{4})$ \\
4.1327313541224929385 & $\sqrt{2 e \pi}$ \\
4.6692116609102990671 & Feigenbaum's constant $\delta$ \\
7.3890560989306502272 & $e^2$ \\
14.1347251417346937904 & The imaginary part of the first nontrivial zero of the Riemann zeta function (the real part is $\frac{1}{2}$) \\
15.1542622414792641898 & $e^e$ \\
36.4621596072079117710 & $\pi^\pi$ \\
\end{tabular}

In looking these up in the OEIS, you can simply type them with a decimal point and no commas between the digits. If you get no results, try chopping off a couple of the least significant digits.

\begin{thebibliography}{1}
\bibitem{aj} Alan Jeffrey, {\it Handbook of Mathematical Formulas and Integrals}, 3rd Edition. New York: Elsevier Academic Press (2004): 223, Section 11.1.4 Special values of $\Gamma(x)$
\end{thebibliography}

%%%%%
%%%%%
\end{document}
