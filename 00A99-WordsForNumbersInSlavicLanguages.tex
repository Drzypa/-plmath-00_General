\documentclass[12pt]{article}
\usepackage{pmmeta}
\pmcanonicalname{WordsForNumbersInSlavicLanguages}
\pmcreated{2013-03-22 17:33:04}
\pmmodified{2013-03-22 17:33:04}
\pmowner{PrimeFan}{13766}
\pmmodifier{PrimeFan}{13766}
\pmtitle{words for numbers in Slavic languages}
\pmrecord{8}{39956}
\pmprivacy{1}
\pmauthor{PrimeFan}{13766}
\pmtype{Topic}
\pmcomment{trigger rebuild}
\pmclassification{msc}{00A99}

% this is the default PlanetMath preamble.  as your knowledge
% of TeX increases, you will probably want to edit this, but
% it should be fine as is for beginners.

% almost certainly you want these
\usepackage{amssymb}
\usepackage{amsmath}
\usepackage{amsfonts}

% used for TeXing text within eps files
%\usepackage{psfrag}
% need this for including graphics (\includegraphics)
%\usepackage{graphicx}
% for neatly defining theorems and propositions
%\usepackage{amsthm}
% making logically defined graphics
%%%\usepackage{xypic}

% there are many more packages, add them here as you need them

% define commands here

\begin{document}
Like English, the Slavic languages derive most of their words for integers from a few words for the numbers 1 to 9 and selected powers of 10.

Also like English, the teens $10 < n < 20$ get their own irregular words which don't follow the pattern established for $20 < n < 100$. However, these words are irregular in different ways among the different languages. Note also the special Russian word for 40, whereas most of the other Slavic languages use a word which is obviously ``four tens.''

The Slavic languages developed from Old Russian and Church Slavonic, adapting the Greek alphabet for their own purposes, coming up with the Cyrillic alphabet.

Modern Russian is of course written in the Cyrillic alphabet, which is still used for some of the other slavic languages. PlanetMath has facilities for including Russian text in entries, however, for the sake of easier comparison, I've decided to transliterate the Russian words to the Roman alphabet. Note however that I've tried to do so as a speaker of some Slavic language would do, and not as an English-speaker.

If it's not too much of a digression, I'd like to mention that the Slavic languages using the Roman alphabet are fairly consistent about using ``c'' only for a ``ts'' sound (any other sound being indicated by the addition of a diacritical mark). The cited books have much more detailed information on pronunciation than can be given here.

In the following table, the first word given is the cardinal (e.g., ``twelve''), and if a second word is given, it's the ordinal (e.g., ``twelfth'').

\begin{tabular}{|r|l|l|l|l|}
$n$ & Russian (translit.) & Polish & Serbo-Croat & Slovene \\
0 & nul' & zero & nula & ni\v{c} \\
1 & od\'in, perviy & jeden, pierwszy & jedan, prvi & ena, prvi \\
2 & dva, vtor\'oy & dwa, drugi & dva, drugi & dva, drugi \\
3 & tri, tr\'etiy & trzy, trzecy & tri, tre\'ci & tri, tretje \\
4 & \v{c}etire, \v{c}etvyortiy & cztery, czwarty & \v{c}etiri, \v{c}etvrti & \v{s}tiri, \v{c}etrti \\
5 & pyat', pyatiy & pi\c{e}\'c, pi\c{a}ty & pet, peti & pet, peti \\
6 & \v{s}est', \v{s}est\'oy & sze\'s\'c, sz\'osty & \v{s}est, \v{s}esti & \v{s}est, \v{s}esti \\
7 & sem', sedim\'oy & siedem, si\'odmy & sedam, sedmi & sedem, sedmi \\
8 & vos\'em', vosem\'oy & osiem, \'osmy & osam, osmi & osem, osmi \\
9 & devyat', devyatiy & dziewi\c{e}\'c, dziewi\c{a}ty & devet, deveti & devet, deveti \\
10 & desyat', desyatiy & dziesi\c{e}\'c, dziesi\c{a}ty & deset, deseti & deset, deseti \\
11 & odinnadcat', odinnadcatiy & jedena\'scie, jedenasty & jedanaest, jedanaesti & enajst, enajsti \\
12 & dven\'adcat', dven\'adcatiy & dwana\'scie, dwunasty & dvanaest, dvanaesti & dvanajst, dvanajsti \\
13 & trin\'adcat', trin\'adcatiy & trzyna\'scie, trzynasty & trinaest, trinaesti & trinajst, trinajsti \\
14 & \v{c}etirn\'adcat', \v{c}etirn\'adcatiy & czterna\'scie, czternasty & \v{c}etrnaest, \v{c}etrnaesti & \v{s}tirinajst, \v{s}tirinajsti \\
15 & pyatn\'adcat', pyatn\'adcatiy & pi\c{e}tna\'scie, pi\c{e}tna\'sty & petnaest, petnaesti & petnajst, petnajsti \\
16 & \v{s}estn\'adcat', \v{s}estn\'adcatiy & szesna\'scie, szesnasty & \v{s}estnaest, \v{s}estnaesti & \v{s}estnajst, \v{s}estnajsti \\
17 & semn\'adcat', semn\'adcatiy & siedemna\'scie, siedemnasty & sedamnaest, sedamnaesti & sedemnajst, sedemnajsti \\
18 & vosemn\'adcat', vosemn\'adcatiy & osiemna\'scie, osiemnasty & osemnaest, osemnaesti & osemnajst, osemnajsti \\
19 & devyatn\'adcat', devyatn\'adcatiy & dziewi\c{e}tna\'scie, dziewi\c{e}tna\'sty & devetnaest, devetnaesti & devetnajst, devetnajsti \\
20 & dvadcat', dvadc\'atiy & dwadzie\'scia, dwudziesty & dvadeset, dvadeseti & dvajdeset, dvajdeseti \\
21 & dvadcat' od\'in & dwaddzie\'scia jeden, , dwudziesty pierwszy & dvadeset i jedan & enaindvajdeset \\
30 & tridcat', tridcatiy & trzydzie\'sci, trzydziesty & trideset, trideseti & trideset \\
40 & sorok, sorokov\'oy & czterdzie\'sci, czterdziesty & \v{c}etrdeset, \v{c}etrdeseti & \v{s}tirideset \\
50 & pyat'desyat' & pi\c{e}\'cdziesi\c{a}t, pi\c{e}\'cdziesi\c{a}t & pedeset & petdeset \\
60 & \v{s}est'desyat & sze\'s\'cdziesi\c{a}t, sze\'s\'cdziesi\c{a}ty & \v{s}ezdeset & \v{s}estdeset \\
70 & sem'desyat & siedemdziesi\c{a}t, siedemdziesi\c{a}ty & sedamdeset & sedemdeset \\
80 & vocem'desyat & osiemdziesi\c{a}t, osiemdziesi\c{a}ty & osamdeset & osemdeset \\
90 & devyan\'osto & dziewi\c{e}\'cdziesi\c{a}t, dziewi\c{e}\'cdziesi\c{a}t & devedeset & devetdeset \\
100 & sto & sto, setny & sto, stoci & sto, stoti \\
200 & dvesti & dwie\'scie, dwusetny & dvesta & dvesto \\
300 & trista & trzy\'scie, trzysetny & trista & tristo \\
1000 & tisya\v{c}a & tysi\c{a}c, tysi\c{e}czny & hiljada, hiljaditi & tiso\v{c} \\
2000 & dve tisya\v{c}i & dwatysi\c{a}ce & dve hiljade & \\
1000000 & milli\'on & milion & milijun & milijon \\
\end{tabular}

In Serbo-Croat, they say ``comma'' where we would say ``point,'' reflecting the European preference for the decimal comma instead of our decimal point. For example, 3.5 = ``tri koma pet.'' The fractions in Serbo-Croat are as irregular as in most other languages. A few examples:

\begin{tabular}{|l|l|}
$\frac{1}{2}$ & polovina \\
$\frac{1}{3}$ & tre\'cine \\
$\frac{1}{4}$ & \v{c}etrvina \\
$\frac{1}{100}$ & stotinka \\
\end{tabular}

\begin{thebibliography}{2}
\bibitem{aa} Anonymous, {\it Serbo-Croat Phrasebook} Bristol: Hadder \& Stoughton Ltd. (1982): 77 - 79
\bibitem{hf} Hania Forss, {\it Polish Phrase Book} Lincolnwood: NTC Publishing Group (1996): 114 - 117
\bibitem{au} Della Thompson, ed., {\it Oxford Russian Starter Dictionary}. Oxford: Oxford University Press
\end{thebibliography}
%%%%%
%%%%%
\end{document}
