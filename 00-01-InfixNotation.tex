\documentclass[12pt]{article}
\usepackage{pmmeta}
\pmcanonicalname{InfixNotation}
\pmcreated{2013-03-22 16:21:20}
\pmmodified{2013-03-22 16:21:20}
\pmowner{akrowne}{2}
\pmmodifier{akrowne}{2}
\pmtitle{infix notation}
\pmrecord{5}{38490}
\pmprivacy{1}
\pmauthor{akrowne}{2}
\pmtype{Definition}
\pmcomment{trigger rebuild}
\pmclassification{msc}{00-01}
\pmsynonym{infix}{InfixNotation}
\pmrelated{GeneralAssociativity}
\pmdefines{infix arithmetic}

\usepackage{amssymb}
\usepackage{amsmath}
\usepackage{amsfonts}

%\usepackage{psfrag}
%\usepackage{graphicx}
%%%\usepackage{xypic} 

\begin{document}
Infix notation is how we usually read and write arithmetic expressions.  In this notation, the operator goes between the operands in the expression:

$$ \text{(operand1)} \text{(operator)} \text{(operand2)} $$

E.g., $ 3 + 2 $, or $ 196 \times 11 $, etc. 

Infix notation suffers from some ambiguity; e.g.

$$ 3 + 9 \times 2 $$ 

could mean $ ( 3 + 9) \times 2 $ or $ 3 + (9 \times 2) $.  Parentheses are needed to specify the order of operations unambiguously.

Postfix notation (or reverse-Polish notation) does not suffer this ambiguity; but it is considered harder for humans to read (hence its primary use in computer applications).

The ``usual'' fix for the ambiguity problem described above is to provide a convention regarding precedence of operations.  This is typically done for computer parsing of mathematical expressions rather than in math done by hand, because in the former case, the computer \emph{must} have some standard rules to proceed.  For example, it is typical to make multiplication ``higher precedence'' than addition, so in the above case, $9 \times 2$ would be performed before adding the result to 3. 

The ambiguity problem only occurs when multiple operators are present in one expression, and thus, the associative law does not hold.  E.g., there is no ambiguity in $1 + 2 + 3$, because $(1 + 2) + 3$ is the same as $1 + (2 + 3)$, by the associative property of addition.

%%%%%
%%%%%
\end{document}
