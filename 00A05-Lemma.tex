\documentclass[12pt]{article}
\usepackage{pmmeta}
\pmcanonicalname{Lemma}
\pmcreated{2013-03-22 13:46:56}
\pmmodified{2013-03-22 13:46:56}
\pmowner{mathcam}{2727}
\pmmodifier{mathcam}{2727}
\pmtitle{lemma}
\pmrecord{19}{34492}
\pmprivacy{1}
\pmauthor{mathcam}{2727}
\pmtype{Definition}
\pmcomment{trigger rebuild}
\pmclassification{msc}{00A05}
\pmdefines{proposition}
\pmdefines{theorem}
\pmdefines{corollary}

% this is the default PlanetMath preamble.  as your knowledge
% of TeX increases, you will probably want to edit this, but
% it should be fine as is for beginners.

% almost certainly you want these
\usepackage{amssymb}
\usepackage{amsmath}
\usepackage{amsfonts}
\usepackage{amsthm}

% used for TeXing text within eps files
%\usepackage{psfrag}
% need this for including graphics (\includegraphics)
%\usepackage{graphicx}
% for neatly defining theorems and propositions
%\usepackage{amsthm}
% making logically defined graphics
%%%\usepackage{xypic}

% there are many more packages, add them here as you need them

% define commands here

\newcommand{\mc}{\mathcal}
\newcommand{\mb}{\mathbb}
\newcommand{\mf}{\mathfrak}
\newcommand{\ol}{\overline}
\newcommand{\ra}{\rightarrow}
\newcommand{\la}{\leftarrow}
\newcommand{\La}{\Leftarrow}
\newcommand{\Ra}{\Rightarrow}
\newcommand{\nor}{\vartriangleleft}
\newcommand{\Gal}{\text{Gal}}
\newcommand{\GL}{\text{GL}}
\newcommand{\Z}{\mb{Z}}
\newcommand{\R}{\mb{R}}
\newcommand{\Q}{\mb{Q}}
\newcommand{\C}{\mb{C}}
\newcommand{\<}{\langle}
\renewcommand{\>}{\rangle}
\begin{document}
\PMlinkescapeword{between}
\PMlinkescapeword{even}
\PMlinkescapeword{word}

There is no technical distinction \PMlinkescapetext{between} a lemma, a proposition, and a theorem.  A \emph{lemma} is a proven statement, typically named a lemma to distinguish it as a truth used as a stepping stone to a larger result rather than an important statement in and of itself.  Of course, some of the most powerful statements in mathematics are known as lemmas, including Zorn's Lemma, Bezout's Lemma, Gauss' Lemma, Fatou's lemma, etc., so one clearly can't get too much simply by reading into a proposition's name.  

Even less well-defined is the distinction between a proposition and a theorem.  Many authors choose to name results only one or the other, or use both more or less interchangeably.  A partially standard set of nomenclature is to use the \PMlinkescapetext{term} \emph{proposition} to denote a significant result that is still shy of deserving a proper name.  In contrast, a \emph{theorem} under this format would \PMlinkescapetext{represent} a major result, and would often be named in \PMlinkescapetext{relation} to mathematicians who worked on or solved the problem in question.

The Greek word ``lemma'' itself means ``anything which is received, such as a gift, profit, or a bribe.''  According to \cite{Higham}, the plural 'Lemmas' is commonly used.  The correct Greek plural of lemma, however, is lemmata.  The Greek ``Theoria'' means ``view, or vision" and is clearly linguistically related to the word ``theatre.''  The apparent relation is that a theorem is a mathematical fact which you see to be true (and can now show others!).

A somewhat more distinct concept (though still subject to author discretion) is that of a \emph{corollary}, which is a result that can be considered an immediate consequence of a previous theorem (typically, the preceding theorem in the text).

Finally, it is worth observing that the above terms are occasionally used to refer to a statement of the prescribed form without reference to the actual truth of the result, e.g., as in the phrase ``While the theorem itself is valid, the \PMlinkescapetext{converse theorem} is actually false."  See this \PMlinkname{attached entry}{ConverseTheorem} for more on this last example.

\begin{thebibliography}{9}
\bibitem{Higham} N. Higham, Handbook of writing for the mathematical sciences, Society for Industrial and Applied Mathematics, 1998.
(pp. 16)
\end{thebibliography}
%%%%%
%%%%%
\end{document}
