\documentclass[12pt]{article}
\usepackage{pmmeta}
\pmcanonicalname{SmallIntegersThatAreOrMightBeTheLargestOfTheirKind}
\pmcreated{2013-03-22 15:53:04}
\pmmodified{2013-03-22 15:53:04}
\pmowner{Mravinci}{12996}
\pmmodifier{Mravinci}{12996}
\pmtitle{small integers that are or might be the largest of their kind}
\pmrecord{22}{37884}
\pmprivacy{1}
\pmauthor{Mravinci}{12996}
\pmtype{Feature}
\pmcomment{trigger rebuild}
\pmclassification{msc}{00A08}
\pmrelated{LargeIntegersThatAreOrMightBeTheSmallestOfTheirKind}
\pmrelated{EveryPositiveIntegerGreaterThan30HasAtLeastOneCompositeTotative}

\endmetadata

% this is the default PlanetMath preamble.  as your knowledge
% of TeX increases, you will probably want to edit this, but
% it should be fine as is for beginners.

% almost certainly you want these
\usepackage{amssymb}
\usepackage{amsmath}
\usepackage{amsfonts}

% used for TeXing text within eps files
%\usepackage{psfrag}
% need this for including graphics (\includegraphics)
%\usepackage{graphicx}
% for neatly defining theorems and propositions
%\usepackage{amsthm}
% making logically defined graphics
%%%\usepackage{xypic}

% there are many more packages, add them here as you need them

% define commands here
\begin{document}
{\bf 1} is the largest sum-product number in binary, and the largest to be a sum-product number in any standard positional base. 1 is the largest integer whose Zeckendorf representation has no significant zeroes. Also, it is the largest (and the only) integer to be labelled non-prime without ever being a probable prime with unknown factorization.

{\bf 2} is the largest (and the only) even prime number. Also, it might be the largest $n$ such that no $n \times n$ magic square consisting of consecutive primes can be constructed. The total of such magic squares is only known up to $n = 6$.

{\bf 3} is the largest and only prime perfect totient number.

{\bf 4} is the largest composite number such that the union of the set of its totatives and the set of its divisors is the complete range of integers from 1 to itself. All larger numbers satisfying this \PMlinkname{relationship between their totatives and divisors}{RelationshipBetweenTotativesAndDivisors} are prime. In another commonality with primes, 4 might be the largest composite $n$ such that $\phi(n)\sigma_0(n) + 2$ is a multiple of $n$, with $\phi(n)$ being Euler's totient function and $\sigma_0(n)$ being a count of the divisors. (For a prime $p$, $\phi(p)\sigma_0(p) + 2 = 2p$). 

{\bf 5} might be the largest untouchable number to be odd and prime. If a larger, odd though composite untouchable number were to be found, 5 would retain the distinction of being the largest such prime. Finding a larger prime untouchable number would strip 5 of both distinctions.

{\bf 6} is the largest integer to be both a factorial and a primorial, and it's the largest squarefree factorial. 6 is the largest $n$ for which the inequality $\phi(n) > \sqrt{n}$ is false. Also, it is a Harshad number regardless of the base in any standard positional base notation. 6 is 110 in binary, 20 in ternary, 12 in base 4, 11 in base 5 and 10 in its own base, and then 6 in all bases afterwards. It is not palindromic in binary, ternary or quaternary, and is thus the largest composite strictly non-palindromic number.

{\bf 8} is the largest Fibonacci number that is also a cube.

{\bf 9} is the largest composite center of a prime quadruplet that is not a multiple of 15.

{\bf 19} is the largest prime Roman numeral palindromic number (XIX). If one allows overlines, one would have to ignore the symmetry of the overlines in order to permit for a larger prime palindrome.

{\bf 23} is the largest integer to have the same representation in both factorial base and primorial base (specifically, 321).

{\bf 24} is the largest $n$ such that $m|n$ for all $0 < m < \sqrt{n}$. Also, it is the largest integer to satisfy the equality $$\sum_{i = 1}^n i^2 = m^2,$$ where $m$ is an integer. [Tattersall, 2005]

{\bf 26} is the largest (and only) integer sandwiched between a square and a cube.

{\bf 30} is the largest integer such that none of its totatives are composite, \PMlinkname{all greater integers have at least one composite totative}{EveryPositiveIntegerGreaterThan30HasAtLeastOneCompositeTotative}. The count of those totatives happens to be equal to the count of its divisors, 30 is the largest integer for which this is true.

{\bf 41} is the largest $n$ such that the polynomial $m^2 - m + n$ yields primes for any positive $m < n$.

{\bf 46} is the largest even integer for which there is no pair of abudant numbers that add up to it. (See the empirical proof that every sufficiently large even integer can be expressed as the sum of a pair of abundant numbers).

{\bf 55} is the largest Fibonacci number that is also a triangular number.

{\bf 60} is thought to be the largest integer that does not admit to a representation under Chen's theorem (as a sum of two distinct primes or a sum of a prime and a semiprime; see A100952 in Sloane's OEIS).60 is also the largest $n$ such that $\pi(n) < \phi(n)$ is false (with $\pi(x)$ being the prime counting function).

{\bf 61} might be the largest prime number $p_x$ (where $x$, the index of $p$ in an ordered list of the primes in ascending order, $x = \pi(x)$) such that $p_x|p_{x + 1}p_{x + 2} + 1$.

{\bf 71} is the largest supersingular prime.

{\bf 90} is the largest $n$ such that $\phi(n) = \pi(n)$, where $\phi(x)$ is Euler's totient function and $\pi(x)$ is the prime-counting function.

{\bf 127} might be the largest prime $p$ satisfying the three conditions of the new Mersenne conjecture [Ribenboim, 2004]. Also, it might be the largest prime $p$ such that $2^p - 1$ is a Chen prime.

{\bf 144} is the largest Fibonacci number that is also a square. In base 10 it is the largest sum-product number, a fact that is amazing when you consider that in order to prove it so David Wilson had to test sum-product number {\em candidates} as large as $10^{84}$.

{\bf 163} is the largest Heegner discriminant.

{\bf 216} might be the largest integer which is not the sum of a prime number and a triangular number. (Sun, 2008)

{\bf 454} is the largest integer such that its shortest partition into cubes requires eight of them.

{\bf 563} might be the largest Wilson prime.

{\bf 786} might be the largest integer for which ${}_{2n}\!C_n$ is not divisible by the square of an odd prime.

{\bf 1493} might be the largest Stern prime, since there is no way to put it in the form $p + 2b^2$, where $p$ is a different prime and $b > 0$. The first hundred thousand primes have been checked and all greater than 1493 can be put into the given form.

{\bf 1806} is the largest $n$ such that $m^{n + 1} = m \mod n$ for any $m$.

For the purpose of this feature, the arbitrary cutoff is $10^4$.

\begin{thebibliography}{3}
\bibitem{rg} R. K. Guy, {\it Unsolved Problems in Number Theory}, B37. New York: Springer-Verlag (2004)
\bibitem{pr} P. Ribenboim, {\it The Little Book of Bigger Primes}, p. 83. New York: Springer-Verlag (2004)
\bibitem{jt} J. J. Tattersall, {\it Elementary number theory in nine chapters}, p. 58. Cambridge: Cambridge University Press (2005)
\bibitem{zs} Zhi-Wei Sun, ``On Sums of Primes and Triangular Numbers'' ArXiv preprint (2008): 2
\end{thebibliography}
%%%%%
%%%%%
\end{document}
