\documentclass[12pt]{article}
\usepackage{pmmeta}
\pmcanonicalname{KochelNumber}
\pmcreated{2013-03-22 16:06:58}
\pmmodified{2013-03-22 16:06:58}
\pmowner{CompositeFan}{12809}
\pmmodifier{CompositeFan}{12809}
\pmtitle{K\"ochel number}
\pmrecord{5}{38182}
\pmprivacy{1}
\pmauthor{CompositeFan}{12809}
\pmtype{Definition}
\pmcomment{trigger rebuild}
\pmclassification{msc}{00A99}
\pmsynonym{Kochel number}{KochelNumber}
\pmrelated{OpusNumber}

% this is the default PlanetMath preamble.  as your knowledge
% of TeX increases, you will probably want to edit this, but
% it should be fine as is for beginners.

% almost certainly you want these
\usepackage{amssymb}
\usepackage{amsmath}
\usepackage{amsfonts}

% used for TeXing text within eps files
%\usepackage{psfrag}
% need this for including graphics (\includegraphics)
%\usepackage{graphicx}
% for neatly defining theorems and propositions
%\usepackage{amsthm}
% making logically defined graphics
%%%\usepackage{xypic}

% there are many more packages, add them here as you need them

% define commands here

\begin{document}
The {\em K\"ochel number} $K_n$ of a \PMlinkescapetext{composition} by Wolfgang Amadeus Mozart is its index in the ordered set compiled by Ludwig von K\"ochel in the conjectured chronological \PMlinkescapetext{order} of \PMlinkescapetext{composition}. ${{K_n} \over {25}} + 10$ approximates Mozart's age at time of writing reasonably well for $n > 100$ (thus ${{K_n} \over {25}} + 1766$ approximates the year).

\begin{thebibliography}{5}
\bibitem{pr} L. von K\"ochel, {\it Chronologisch-thematisches Verzeichnis s\"amtlicher Tonwerke Wolfgang Amade Mozarts; nebst Angabe der verlorengegangenen, angefangenen, \"ubertragenen, zweifelhaften und unterschobenen Kompositionen}, (1947), Ann Arbor: J. W. Edwards.
\end{thebibliography}
%%%%%
%%%%%
\end{document}
