\documentclass[12pt]{article}
\usepackage{pmmeta}
\pmcanonicalname{SimpleInterest}
\pmcreated{2013-03-22 16:40:06}
\pmmodified{2013-03-22 16:40:06}
\pmowner{CWoo}{3771}
\pmmodifier{CWoo}{3771}
\pmtitle{simple interest}
\pmrecord{7}{38875}
\pmprivacy{1}
\pmauthor{CWoo}{3771}
\pmtype{Example}
\pmcomment{trigger rebuild}
\pmclassification{msc}{00A06}
\pmclassification{msc}{00A69}
\pmclassification{msc}{91B28}
\pmrelated{CompoundInterest}
\pmrelated{InterestRate}

\usepackage{amssymb,amscd}
\usepackage{amsmath}
\usepackage{amsfonts}
\usepackage{tabls}
% used for TeXing text within eps files
%\usepackage{psfrag}
% need this for including graphics (\includegraphics)
%\usepackage{graphicx}
% for neatly defining theorems and propositions
%\usepackage{amsthm}
% making logically defined graphics
%%\usepackage{xypic}
\usepackage{pst-plot}
\usepackage{psfrag}

% define commands here

\begin{document}
Suppose a bank account is opened at time $0$ and $M_0$ is deposited into the account.  A \emph{simple interest} is interest with the following characteristics:
\begin{enumerate}
\item it is earned at subsequent time periods $t,2t,\ldots$, where $t$ is the length of the initial time interval (1 for 1 month, 12 for 1 year, etc...)
\item the interest earned at the end of each time period is the same regardless of the time period
\end{enumerate}
The following table illustrates the structure of the simple interest.

\begin{center}
\begin{tabular}{|c||c|c|c|}
\hline  time period at & principal & interest & interest accrued \\
\hline\hline $0$ & $M_0$ & $0$ & $0$ \\
\hline $t$ & $M_0$ & $i$ & $i$ \\
\hline $2t$ & $M_0$ & $i$ & $2i$ \\
\hline $3t$ & $M_0$ & $i$ & $3i$ \\
\hline $\vdots$ & $\vdots$ & $\vdots$ & $\vdots$ \\
\hline $nt$ & $M_0$ & $i$ & $ni$ \\
\hline
\end{tabular}
\end{center}

The ``total'' interest $i(nt)$ earned (accrued) at the end of time $nt$ is $ni$.  If the account is closed and the money withdrawn at the end of $nt$, and the total amount of money received is $$M(nt)=M_0+ni.$$

The interest rate associated with the simple interest as presented above between two time periods, say $at$ and $bt$, is given by
$$r(at,bt)=\frac{1}{M_0}\frac{i(bt)-i(at)}{bt-at}=\frac{i}{M_0t},$$
which does not depend on the choice of $a$ and $b$.  In other words, the original principal $M_0$, the amount of interest $i$, and the length of the initial time interval $t$ are enough to determine the interest rate.

\textbf{Remark}.  
\begin{itemize}
\item
The expression for the effective interest rate for simple interest is a bit more complicated:
$$\operatorname{eff.}r(at,bt)=\frac{1}{M(at)}\frac{i(bt)-i(at)}{bt-at}=
\frac{1}{M_0+ai}\frac{i}{t},$$
which decreases with increasing $a$.  Imagine as $a$ becomes very large, the increase in interest has practically no impact on the ``accumulated'' principal $M(at)$.
\item
More generally, we say that an interest is \emph{simple} if its interest rate $r$ is constant with respect to time $t$.  Solving
$$r=\frac{1}{M_0}\frac{i(t)-i(0)}{t-0}$$
for $i(t)$, we get $i(t)=M_0rt$, or that the accrued interest is a linear function of $t$.  It grows directly proportionally with respect to time.
\end{itemize}
%%%%%
%%%%%
\end{document}
