\documentclass[12pt]{article}
\usepackage{pmmeta}
\pmcanonicalname{SaddlePointApproximation}
\pmcreated{2013-03-22 13:38:07}
\pmmodified{2013-03-22 13:38:07}
\pmowner{msihl}{2134}
\pmmodifier{msihl}{2134}
\pmtitle{saddle point approximation}
\pmrecord{5}{34284}
\pmprivacy{1}
\pmauthor{msihl}{2134}
\pmtype{Topic}
\pmcomment{trigger rebuild}
\pmclassification{msc}{00A05}
\pmclassification{msc}{00A79}
\pmsynonym{stationary phase method}{SaddlePointApproximation}

% this is the default PlanetMath preamble.  as your knowledge
% of TeX increases, you will probably want to edit this, but
% it should be fine as is for beginners.

% almost certainly you want these
\usepackage{amssymb}
\usepackage{amsmath}
\usepackage{amsfonts}

% used for TeXing text within eps files
%\usepackage{psfrag}
% need this for including graphics (\includegraphics)
%\usepackage{graphicx}
% for neatly defining theorems and propositions
%\usepackage{amsthm}
% making logically defined graphics
%%%\usepackage{xypic}

% there are many more packages, add them here as you need them

% define commands here
\begin{document}
The saddle point approximation (SPA), a.k.a. stationary phase approximation, is a widely used method in quantum field theory (QFT) and related fields. 
Suppose we want to evaluate the following integral in the limit $\zeta \rightarrow \infty$:
\begin{equation}\label{eq:integral}
{\mathcal I} = \lim_{\zeta \rightarrow \infty} \int_{-\infty}^{\infty} {\mbox d}x
\; {\mbox e}^{- \zeta f(x)}.
\end{equation}
The saddle point approximation can be applied if the function $f(x)$ satisfies certain conditions. Assume that $f(x)$ has a global minimum $f(x_0)=y_{min}$ at $x=x_0$, which is sufficiently separated from other local minima and whose value is sufficiently smaller than the value of those.
Consider the Taylor expansion of $f(x)$ about the point $x_0$:
\begin{equation}\label{eq:taylor}
f(x) = f(x_0) + \partial_x f(x){\Big |}_{x=x_0}(x-x_0) + \frac{1}{2} {\partial_x}^2 f(x){\Big |}_{x=x_0}(x-x_0)^2 + O(x^3).
\end{equation}
Since $f(x_0)$ is a (global) minimum, it is clear that $f'(x_0)=0$. Therefore $f(x)$ may be approximated to quadratic order as 
\begin{equation}
f(x) \approx f(x_0) + \frac{1}{2} f''(x_0) (x-x_0)^2.
\end{equation}
The above assumptions on the minima of $f(x)$ ensure that the dominant contribution to (\ref{eq:integral}) in the limit $\zeta \rightarrow \infty$ will 
come from the region of integration around $x_0$:
\begin{align}\label{eq:second}
{\mathcal I} &\approx \lim_{\zeta \rightarrow \infty} {\mbox e}^{-\zeta f(x_0)} \int_{-\infty}^{\infty} {\mbox d}x \; {\mbox e}^{-\frac{\zeta}{2} f''(x_0)(x-x_0)^2} \\ \nonumber
&\approx \lim_{\zeta \rightarrow \infty} {\mbox e}^{-\zeta f(x_0)} \left( \frac{2 \pi}{\zeta f''(x_0)}\right)^{1/2}.
\end{align}
In the last step we have performed the Gau{\ss}ian integral. 
The next nonvanishing higher order correction to (\ref{eq:second}) stems from the quartic term of the expansion (\ref{eq:taylor}). This correction may be incorporated into (\ref{eq:second}) to yield (after expanding part of the exponential):
\begin{equation}
{\mathcal I} \approx \lim_{\zeta \rightarrow \infty} {\mbox e}^{-\zeta f(x_0)} \int_{-\infty}^{\infty} {\mbox d}x \; {\mbox e}^{-\frac{\zeta}{2} f''(x_0)(x-x_0)^2} \left( 1 - \frac{\zeta}{4!} (\partial_x^4 f(x))|_{x=x_0}(x-x_0)^4 \right).
\end{equation}
...to be continued with applications to physics...
%%%%%
%%%%%
\end{document}
