\documentclass[12pt]{article}
\usepackage{pmmeta}
\pmcanonicalname{Calculator}
\pmcreated{2013-03-22 16:39:16}
\pmmodified{2013-03-22 16:39:16}
\pmowner{PrimeFan}{13766}
\pmmodifier{PrimeFan}{13766}
\pmtitle{calculator}
\pmrecord{4}{38859}
\pmprivacy{1}
\pmauthor{PrimeFan}{13766}
\pmtype{Definition}
\pmcomment{trigger rebuild}
\pmclassification{msc}{00A05}
\pmclassification{msc}{01A07}
\pmrelated{CalculatorAndCASSupportForVariousPositionalBases}

\endmetadata

% this is the default PlanetMath preamble.  as your knowledge
% of TeX increases, you will probably want to edit this, but
% it should be fine as is for beginners.

% almost certainly you want these
\usepackage{amssymb}
\usepackage{amsmath}
\usepackage{amsfonts}

% used for TeXing text within eps files
%\usepackage{psfrag}
% need this for including graphics (\includegraphics)
%\usepackage{graphicx}
% for neatly defining theorems and propositions
%\usepackage{amsthm}
% making logically defined graphics
%%%\usepackage{xypic}

% there are many more packages, add them here as you need them

% define commands here

\begin{document}
A {\em calculator} is an electronic, electrical or mechanical device (hardware) or a computer program (software) designed to perform a predefined set of arithmetic computations on numbers entered by the user and display the results.

The abacus is sometimes given as an example of an early calculator; however, the completion of a calculation requires the active participation of the user and is easily subject to user error even when the device passes a diagnostic with flying colors, whereas a modern calculator in working order calculates and displays the correct result regardless of whether or not the user sticks around to see it (this is subject to certain caveats regarding precision, however). Of course it has always been the case in the history of calculators that users may enter incorrect inputs or misunderstand the operation of the device and thus get irrelevant results.

One of the first mechanical calculators was designed by Blaise Pascal in the 17th Century. It used a gear for each digit and the gears were connected, it took up a little space on a desk. It was limited to addition and subtraction. By the 1990s, electronic calculators were ubiquitous and small enough to put in a pocket.

A basic calculator has about twenty keys: the digits 0 to 9, a decimal point, sign change, addition, subtraction, multiplication, division, equal or Enter key and a C key to clear error exceptions. Occasionally such calculators also have keys for percentage and square root. Another option sometimes found on basic calculators is a memory register and associated keys M+, MR and MC (for adding to memory register, recalling memory and clearing memory, respectively).

More advanced calculators include scientific calculators, graphing calculators, programmable calculators. Most calculators use standard infix notation, but there are reverse Polish notation calculators also available on the market.
%%%%%
%%%%%
\end{document}
