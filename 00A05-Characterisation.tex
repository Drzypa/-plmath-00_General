\documentclass[12pt]{article}
\usepackage{pmmeta}
\pmcanonicalname{Characterisation}
\pmcreated{2013-03-22 14:22:28}
\pmmodified{2013-03-22 14:22:28}
\pmowner{pahio}{2872}
\pmmodifier{pahio}{2872}
\pmtitle{characterisation}
\pmrecord{18}{35865}
\pmprivacy{1}
\pmauthor{pahio}{2872}
\pmtype{Definition}
\pmcomment{trigger rebuild}
\pmclassification{msc}{00A05}
\pmsynonym{characterization}{Characterisation}
\pmsynonym{defining property}{Characterisation}
\pmrelated{AlternativeDefinitionOfGroup}
\pmrelated{EquivalentFormulationsForContinuity}
\pmrelated{MultiplicationRuleGivesInverseIdeal}

% this is the default PlanetMath preamble.  as your knowledge
% of TeX increases, you will probably want to edit this, but
% it should be fine as is for beginners.

% almost certainly you want these
\usepackage{amssymb}
\usepackage{amsmath}
\usepackage{amsfonts}

% used for TeXing text within eps files
%\usepackage{psfrag}
% need this for including graphics (\includegraphics)
%\usepackage{graphicx}
% for neatly defining theorems and propositions
%\usepackage{amsthm}
% making logically defined graphics
%%%\usepackage{xypic}

% there are many more packages, add them here as you need them

% define commands here
\begin{document}
In mathematics, {\em characterisation} usually means a property or a condition to define a certain notion. \,A notion may, under some presumptions, have different \PMlinkescapetext{equivalent} ways to define it.

For example, let $R$ be a commutative ring with non-zero unity (the presumption). \,Then the following are equivalent:

(1) All finitely generated regular ideals of $R$ are invertible.

(2) The \PMlinkescapetext{formula} \,$(a,\,b)(c,\,d) = (ac,\,bd,\,(a+b)(c+d))$\, for multiplying ideals of $R$ is valid always when at least one of the elements $a$, $b$, $c$, $d$ of $R$ is not zero-divisor.

(3) Every overring of $R$ is integrally closed.

Each of these conditions is sufficient (and necessary) for characterising and defining the Pr\"ufer ring.
%%%%%
%%%%%
\end{document}
