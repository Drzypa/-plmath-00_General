\documentclass[12pt]{article}
\usepackage{pmmeta}
\pmcanonicalname{PreservationAndReflection}
\pmcreated{2013-03-22 17:12:18}
\pmmodified{2013-03-22 17:12:18}
\pmowner{CWoo}{3771}
\pmmodifier{CWoo}{3771}
\pmtitle{preservation and reflection}
\pmrecord{5}{39525}
\pmprivacy{1}
\pmauthor{CWoo}{3771}
\pmtype{Definition}
\pmcomment{trigger rebuild}
\pmclassification{msc}{00A35}
\pmdefines{preserve}
\pmdefines{reflect}

\endmetadata

\usepackage{amssymb,amscd}
\usepackage{amsmath}
\usepackage{amsfonts}
\usepackage{mathrsfs}

% used for TeXing text within eps files
%\usepackage{psfrag}
% need this for including graphics (\includegraphics)
%\usepackage{graphicx}
% for neatly defining theorems and propositions
\usepackage{amsthm}
% making logically defined graphics
%%\usepackage{xypic}
\usepackage{pst-plot}
\usepackage{psfrag}

% define commands here
\newtheorem{prop}{Proposition}
\newtheorem{thm}{Theorem}
\newtheorem{ex}{Example}
\newcommand{\real}{\mathbb{R}}
\newcommand{\pdiff}[2]{\frac{\partial #1}{\partial #2}}
\newcommand{\mpdiff}[3]{\frac{\partial^#1 #2}{\partial #3^#1}}
\begin{document}
In mathematics, the word ``preserve'' usually means the ``preservation of properties''.  Loosely speaking, whenever a mathematical construct $A$ has some property $P$, after $A$ is somehow ``transformed'' into $A'$, the transformed object $A'$ also has property $P$.  The constructs usually refer to sets and the transformations typically are functions or something similar.

Here is a simple example, let $f$ be a function from a set $A$ to $B$.  Let $A$ be a finite set.  Let $P$ be the property of a set being finite.  Then $f$ preserves $P$, since $f(A)$ is finite.  Note that we are not saying that $B$ is finite.  We are merely saying that the portion of $B$ that is the \emph{image} of $A$ (the transformed portion) is finite.

Here is another example.  The property of being connected in a topological space is preserved under a continuous function.  Here, the constructs are topological spaces, and the transformation is a continuous function.  In other words, if $f:X\to Y$ is a continuous function from $X$ to $Y$.  If $X$ is connected, so is $f(X)\subseteq Y$.

Many more examples can be found in abstract algebra.  Group homomorphisms, for example, preserve commutativity, as well as the property of being finitely generated.

The word ``reflect'' is the dual notion of ``preserve''.  It means that if the transformed object has property $P$, then the original object also has property $P$.  This usage is rarely found outside of category theory, and is almost exclusively reserved for functors.  For example, a faithful functor reflects isomorphism: if $F$ is a faithful functor from $\mathcal{C}$ to $\mathcal{D}$, and the object $F(A)$ is isomorphic to the object $F(B)$ in $\mathcal{D}$, then $A$ is isomorphic to $B$ in $\mathcal{C}$.
%%%%%
%%%%%
\end{document}
