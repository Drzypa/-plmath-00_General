\documentclass[12pt]{article}
\usepackage{pmmeta}
\pmcanonicalname{Proof}
\pmcreated{2013-03-22 18:10:22}
\pmmodified{2013-03-22 18:10:22}
\pmowner{PrimeFan}{13766}
\pmmodifier{PrimeFan}{13766}
\pmtitle{proof}
\pmrecord{4}{40736}
\pmprivacy{1}
\pmauthor{PrimeFan}{13766}
\pmtype{Definition}
\pmcomment{trigger rebuild}
\pmclassification{msc}{00A05}

% this is the default PlanetMath preamble.  as your knowledge
% of TeX increases, you will probably want to edit this, but
% it should be fine as is for beginners.

% almost certainly you want these
\usepackage{amssymb}
\usepackage{amsmath}
\usepackage{amsfonts}

% used for TeXing text within eps files
%\usepackage{psfrag}
% need this for including graphics (\includegraphics)
%\usepackage{graphicx}
% for neatly defining theorems and propositions
%\usepackage{amsthm}
% making logically defined graphics
%%%\usepackage{xypic}

% there are many more packages, add them here as you need them

% define commands here

\begin{document}
A {\em proof} is an argument designed to show that a statement (usually a theorem) is true.

The mathematical concept of proof differs from the scientific concept in that in mathematics, a proof is logically deduced from axioms or from other theorems which have also been logically deduced, whereas for a scientific proof a preponderance of evidence is sufficient. Thus, a valid mathematical proof assures there are no counterexamples to the proven statement.

There are several kinds of proofs, one commonly used one being proof by contradiction. A proof by contradiction starts by assuming that the opposite of the theorem is true, and then proceeds to work out the consequences of that assumption until encountering a contradiction, thus proving the theorem.

According to Paul Nahin, the most famous proof by contradiction is Euclid's proof of the infinitude of primes, which starts by assuming that there is in fact a largest prime number (and thus the primes are finite). Proofs that a given number is irrational (such as $\pi$ or $\sqrt{5}$) also tend to prove the irrationality of the number by at first assuming that the number is in fact rational and that there are two integers which form a ratio for the given number.

Another kind of proof is the proof by induction, which starts by showing the statement is true for a small case (such as $n = 1$ when dealing with integers) and that the statement is true for a larger case when it is true for the immediately smaller case (e.g., that if it's true for $n$ it is also true for $n + 1$). Thus, showing that it is true for the small case proves that it is also true for the next larger case, and the next larger case after that, and therefore all the larger cases.

A proof by construction shows that a specified object actually exists by showing how to construct that object. For example, to prove that it is possible to draw by compass and straightedge an isosceles triangle with an angle that is half of any of the two other angles, a constructive proof would give the instructions on how to draw such a triangle.

\begin{thebibliography}{3}
\bibitem{pn} Paul J. Nahin, {\it Dr. Euler's Fabulous Formula: Cures Many Mathematical Ills}. Princeton: Princeton University Press (2006): 8
\bibitem{tw} Thomas A. Whitelaw, {\it Introduction to Abstract Algebra}. New York: CRC Press (1995): 11
\end{thebibliography}
%%%%%
%%%%%
\end{document}
