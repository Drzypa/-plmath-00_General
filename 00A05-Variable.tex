\documentclass[12pt]{article}
\usepackage{pmmeta}
\pmcanonicalname{Variable}
\pmcreated{2013-03-22 15:31:39}
\pmmodified{2013-03-22 15:31:39}
\pmowner{stevecheng}{10074}
\pmmodifier{stevecheng}{10074}
\pmtitle{variable}
\pmrecord{10}{37407}
\pmprivacy{1}
\pmauthor{stevecheng}{10074}
\pmtype{Definition}
\pmcomment{trigger rebuild}
\pmclassification{msc}{00A05}
\pmrelated{Parameter}

\endmetadata

\usepackage{amssymb}
\usepackage{amsmath}
\usepackage{amsfonts}
%\usepackage{amsthm}
\usepackage{enumerate}

% used for TeXing text within eps files
%\usepackage{psfrag}
% need this for including graphics (\includegraphics)
%\usepackage{graphicx}
% making logically defined graphics
%%%\usepackage{xypic}

% define commands here
\newcommand{\complex}{\mathbb{C}}
\newcommand{\real}{\mathbb{R}}
\newcommand{\rat}{\mathbb{Q}}
\newcommand{\nat}{\mathbb{N}}

\providecommand{\abs}[1]{\lvert#1\rvert}
\providecommand{\absW}[1]{\left\lvert#1\right\rvert}
\providecommand{\absB}[1]{\Bigl\lvert#1\Bigr\rvert}
\providecommand{\norm}[1]{\lVert#1\rVert}
\providecommand{\normW}[1]{\left\lVert#1\right\rVert}
\providecommand{\normB}[1]{\Bigl\lVert#1\Bigr\rVert}
\providecommand{\defnterm}[1]{\emph{#1}}

\DeclareMathOperator{\D}{D}
\DeclareMathOperator{\linspan}{span}
\begin{document}
The word \emph{variable} as used in mathematics (and in other scientific fields that use mathematics) is somewhat vague and may have different meanings depending on the context.  Variables 
are usually denoted by a single Roman or Greek letter, e.g. $x$,
although sometimes a whole word or phrase can be used also.

Here is a list of some of the meanings of \emph{variable}:

\begin{description}
\item[(i) As ``mathematical'' variables.]
These stand for a concrete object, for example, an element of the real numbers
That is, when we write the symbol $x$, it is a stand-in for various numbers:
e.g. $2, 3, \pi, e, 578.24$.
But we do not name these numbers specifically, because we may want to talk
about all these numbers at once, in a general statement, theorem, or proof
about numbers.

Sense (i) is probably the most common usage in mainstream mathematics.

\item[(ii) As placeholders in functional notation.]
For example, we may be defining a function using the phrase 
``define the function $f(z) = z^2 + 4$ for complex numbers $z$.
This usage of a variable is slightly different from sense (i), 
because our objective is to talk about the \emph{function} $f$, 
\emph{and not its value} at a number $z$ which is $f(z)$.  The notation ``$f(z) = z^2 + 4$'' 
is merely a much more convenient way of saying: ``define the function $f$ which takes a complex number, multiplies it by itself, and then adds four to it''.
It could also be rephrased this way:
``define a function $f$ such that the statement $f(z) = z^2 + 4$ 
is true for all complex numbers $z$ (in sense (i))''.

On the other hand, the symbol $f$, if we were to contemplate it as a ``variable'',
arguably belongs to the sense (i); in this case we are talking about
\emph{some specific function}, not all functions.

\item[(iii) As ``formal'' variables.] 
For instance, we may talk about a formal polynomial
$p(x) = 1 + x + x^2$.
This is similar to sense (ii), but is not exactly the same.
The variable $x$ here is not necessarily a complex number, or in any fixed
domain at all.  It is a formal symbol, which we later replace by
actual elements of the real numbers, or matrices, etc. at our whim.
And $p$ here is \emph{not a function}; it is a polynomial.

The variables used in formal logic can also be considered to fall in sense (iii).
For example, we may have a set of variables $ \{ x, y, z \}$
and a formula from the first-order language using such variables: 
$( \exists x (( x \leq 0) \land \mathrm{R}(z) )$.


\item[(iv) As pieces of (experimental) data.]
Used in the sciences.  One may say ``at $t=4 \: \mathrm{s}$, $x=23.1 \: \mathrm{m}$''
which may really mean: ``at 4 seconds from the start of the experiment,
the object is 23.1 metres to the right of its initial position''.

So the symbols $t$ and $x$ are being used in the meaning
of ``time'' and ``position'' in general.  
There may or may not be a functional relation between
the ``variables'' $t$ and $x$.  If there is, we might say 
``$x$ is a function of $t$'', and we can talk about
quantities such as $dx/dt$.

If we want to talk about a specific (but unnamed) time,
we can use a notation such as ``when $t = t_0, \dotsc$''
for some variable $t_0$ in sense (i).

The field of probability and statistics follows a similar
practice for what are termed ``random variables'',
which are really functions defined on a measure space $\Omega$.
But in practice they are usually denoted with variable notation:
e.g. ``the random variable $X$'', and a specific value of this
random variable $X$, at some unspecified $\omega \in \Omega$, 
is denoted by $x$.

\item[(v) As state variables in computer algorithms.]
In this case, a variable $x$ stands for a computer memory location.
Or in more abstract language, $x$ is a name for a container
which may hold some object.
The contents of this container may change as time passes
or when it is modified by a program that the computer is executing.

In formal language, putting a value in the container
is often denoted by notation like ``$x \leftarrow 2$''.

\end{description}

Note that the above distinctions are not always clear-cut.
and the same symbol $x$ may be 
used for different purposes at once,
which of course, may lead to confusion.

%%%%%
%%%%%
\end{document}
