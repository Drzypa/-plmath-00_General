\documentclass[12pt]{article}
\usepackage{pmmeta}
\pmcanonicalname{MatheRealism}
\pmcreated{2014-10-29 17:12:48}
\pmmodified{2014-10-29 17:12:48}
\pmowner{WM}{16977}
\pmmodifier{WM}{16977}
\pmtitle{MatheRealism}
\pmrecord{16}{39326}
\pmprivacy{1}
\pmauthor{WM}{16977}
\pmtype{Topic}
\pmcomment{trigger rebuild}
\pmclassification{msc}{00A30}
\pmrelated{DecimalExpansion}

% this is the default PlanetMath preamble.  as your knowledge
% of TeX increases, you will probably want to edit this, but
% it should be fine as is for beginners.

% almost certainly you want these
\usepackage{amssymb}
\usepackage{amsmath}
\usepackage{amsfonts}

% used for TeXing text within eps files
%\usepackage{psfrag}
% need this for including graphics (\includegraphics)
%\usepackage{graphicx}
% for neatly defining theorems and propositions
%\usepackage{amsthm}
% making logically defined graphics
%%%\usepackage{xypic}

% there are many more packages, add them here as you need them

% define commands here

\begin{document}
\emph{MatheRealism} is the position that the amount of information in
the universe places a limit on the possible contents of mathematics.
Its supporters claim it as a philosophical foundation of mathematics.

The argument proceeds as follows.  The number of atoms is about
10^80 and will remain so forever, limited by the horizon of
observation and notwithstanding the expansion of the universe; the
remaining part of the universe is causally disconnected from us. The
number of elementary particles is less than 10^100. Although every
atom has infinitely many eigenstates, only a finite number of them can
be distinguished (due to the uncertainty relation of quantum
physics). Further the eigenstates, with exception of the ground state,
have a limited life time. All this leads to the result that only a finite number of bits
can be stored in the universe. A conservative estimate is 10^100
bits, but in any case there is an upper limit of information \emph{L}.

According to MatheRealism only such numbers exist which are computable or can be identified uniquely and addressed individually by any other means.  In particular, the supporters of
this view claim that any irrational number which cannot be
represented by less information than is contained by its infinite
string of bits, cannot get addressed at all and that even most natural
numbers cannot get addressed because their most economical
representation requires more information than \emph{L}.  According to
MatheRealism, a number which cannot be represented, addressed, or used
otherwise does not exist.  This implies that infinite sets do not exist.

Note: Although all available numbers have a finite contents of information, there is not a greatest number, because, by useful abbreviations, numbers as large as desired can be represented by means of little information.

The expression MatheRealism is touching on materialism. It may not be mixed up with the notion "realism" of current philosophy of mathematics which in fact is an idealism.


\textbf{Literature} W. M\"uckenheim: Die Mathematik des Unendlichen, Shaker, Aachen 2006.

\PMlinkescapeword{argument}
\PMlinkescapeword{atom}
\PMlinkescapeword{atoms}
\PMlinkescapeword{current}
\PMlinkescapeword{disconnected}
\PMlinkescapeword{even}
\PMlinkescapeword{expansion}
\PMlinkescapeword{foundation}
\PMlinkescapeword{information}
\PMlinkescapeword{limit}
\PMlinkescapeword{observation}
\PMlinkescapeword{opposite}
\PMlinkescapeword{places}
\PMlinkescapeword{representation}
\PMlinkescapeword{universe}
%%%%%
%%%%%
\end{document}
