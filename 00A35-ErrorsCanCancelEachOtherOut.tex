\documentclass[12pt]{article}
\usepackage{pmmeta}
\pmcanonicalname{ErrorsCanCancelEachOtherOut}
\pmcreated{2013-03-22 18:59:39}
\pmmodified{2013-03-22 18:59:39}
\pmowner{pahio}{2872}
\pmmodifier{pahio}{2872}
\pmtitle{errors can cancel each other out}
\pmrecord{13}{41861}
\pmprivacy{1}
\pmauthor{pahio}{2872}
\pmtype{Example}
\pmcomment{trigger rebuild}
\pmclassification{msc}{00A35}
\pmclassification{msc}{26A06}
\pmclassification{msc}{97D70}
%\pmkeywords{miscalculation}
\pmrelated{UniversalTrigonometricSubstitution}
\pmrelated{SubstitutionNotation}
\pmrelated{IntegrationOfRationalFunctionOfSineAndCosine}

\endmetadata

% this is the default PlanetMath preamble.  as your knowledge
% of TeX increases, you will probably want to edit this, but
% it should be fine as is for beginners.

% almost certainly you want these
\usepackage{amssymb}
\usepackage{amsmath}
\usepackage{amsfonts}

% used for TeXing text within eps files
%\usepackage{psfrag}
% need this for including graphics (\includegraphics)
%\usepackage{graphicx}
% for neatly defining theorems and propositions
%\usepackage{amsthm}
% making logically defined graphics
%%%\usepackage{xypic}

% there are many more packages, add them here as you need them

% define commands here
\newcommand{\sijoitus}[2]%
{\operatornamewithlimits{\Big/}_{\!\!\!#1}^{\,#2}}
\begin{document}
If one uses the \PMlinkname{change of variable}{ChangeOfVariableInDefiniteIntegral}
\begin{align}
\tan{x} \;:=\; t, \quad dx \;=\; \frac{dt}{1\!+\!t^2},\quad \cos^2x \;=\; \frac{1}{1\!+\!t^2}
\end{align}
for finding the value of the definite integral
$$I \;:=\; \int_{\frac{\pi}{4}}^{\frac{3\pi}{4}}\frac{dx}{2\cos^2x+1},$$
the following calculation looks appropriate and faultless:
\begin{align}
I \;=\; \int_1^{-1}\!\!\frac{dt}{3\!+\!t^2} 
    \;=\; \frac{1}{\sqrt{3}}\!\!\sijoitus{1}{\;\quad -1}\!\arctan\frac{t}{\sqrt{3}}
    \;=\; \frac{1}{\sqrt{3}}\!\left(\frac{5\pi}{6}-\frac{\pi}{6}\right) \;=\; \frac{2\pi}{3\sqrt{3}}
\end{align}

The result is quite \PMlinkescapetext{right}.\, Unfortunately, the calculation \PMlinkescapetext{contains} two errors, the effects of which cancel each other out.\\

The crucial error in (2) is using the substitution (1) when $\tan{x}$ is discontinuous in the point 
\,$x = \frac{\pi}{2}$\, on the interval \,$[\frac{\pi}{4},\,\frac{3\pi}{4}]$\, of integration.\, The error is however canceled out by the second error using the value $\frac{5\pi}{6}$ for $\arctan\frac{-1}{\sqrt{3}}$, when the right value were $-\frac{\pi}{6}$ (the values of arctan lie only between $-\frac{\pi}{2}$ and $\frac{\pi}{2}$; see cyclometric functions).\, The value $\frac{5\pi}{6}$ belongs to a different branch of the inverse tangent function than $\frac{\pi}{6}$; parts of two distinct branches cannot together form the antiderivative which must be continuous.\\

What were a right way to calculate $I$?\, The universal trigonometric substitution produces an awkward integrand
$$\frac{2\!+\!2t^2}{3\!-\!2t^2\!+\!3t^4}$$
and \PMlinkescapetext{limits} $\sqrt{2}-1$ and $1-\sqrt{2}$,\, therefore it is unusable.\, It is now better to change the interval of integration, using the properties of trigonometric functions.

Since the (graph of) cosine squared is symmetric about the line \,$x = \frac{\pi}{2}$,\, we could integrate only over\, 
$[\frac{\pi}{4},\,\frac{\pi}{2}]$ and multiply the integral by 2 (cf. integral of even and odd functions):
$$I \;=\; 2\int_{\frac{\pi}{4}}^{\frac{\pi}{2}}\frac{dx}{2\cos^2x+1}.$$
We can also get rid of the inconvenient upper limit $\frac{\pi}{2}$ by changing over to the sine in virtue of the complement formula 
$$\cos(\frac{\pi}{2}\!-\!x) \;=\; \sin{x},$$
getting
$$I \;=\; 2\int_0^{\frac{\pi}{4}}\frac{dx}{2\sin^2x+1}.$$
Then (1) is usable, and because\, $\sin^2x = \frac{t^2}{1\!+\!t^2}$,\, we obtain
$$I \;=\; 2\int_0^1\frac{dt}{\left(\frac{2t^2}{1\!+\!t^2}+1\right)(1\!+\!t^2)}
    \;=\; 2\int_0^1\frac{dt}{3t^2\!+\!1} \;=\; \frac{2}{\sqrt{3}}\!\!\sijoitus{0}{\;\quad 1}\!\arctan{t\sqrt{3}}
    \;=\; \frac{2\pi}{3\sqrt{3}}.$$





%%%%%
%%%%%
\end{document}
