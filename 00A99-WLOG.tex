\documentclass[12pt]{article}
\usepackage{pmmeta}
\pmcanonicalname{WLOG}
\pmcreated{2013-03-22 12:40:17}
\pmmodified{2013-03-22 12:40:17}
\pmowner{akrowne}{2}
\pmmodifier{akrowne}{2}
\pmtitle{WLOG}
\pmrecord{6}{32946}
\pmprivacy{1}
\pmauthor{akrowne}{2}
\pmtype{Definition}
\pmcomment{trigger rebuild}
\pmclassification{msc}{00A99}
\pmsynonym{WOLOG}{WLOG}
\pmsynonym{without loss of generality}{WLOG}

\endmetadata

\usepackage{amssymb}
\usepackage{amsmath}
\usepackage{amsfonts}

%\usepackage{psfrag}
%\usepackage{graphicx}
%%%\usepackage{xypic}
\begin{document}
``WLOG'' (or ``WOLOG'') is an acronym which stands for ``without loss of generality.''

WLOG is invoked in situations where some property of a model or system is invariant under the particular choice of instance attributes, but for the sake of demonstration, these attributes must be fixed.

For example, we might want to prove something about open intervals $(a, b)$ of the real number line.  But the proof might become too tedious if $a$ and $b$ were arbitrary real numbers, so in the proof we simply assume that $a = 0$ and $b = 1$, and \emph{without loss of generality}, the same arguments apply to general intervals $(a, b)$.  Depending on the proof, the loss of generality might be accomplished by translating and scaling the interval to $(0,1)$ \emph{before} carrying out the argument, and then translating and rescaling back to $(a, b)$
afterwards.

WLOG can also be invoked to shorten proofs where there are a number of choices of configuration, but the proof is ``the same'' for each of them.  We need only walk through the proof for one of these configurations, and ``WLOG'' serves as a note that we haven't weakened the argument.  For example, the proof of the fundamental theorem of arithmetic uses this notion, in essence settling on a ``canonical form'' for prime factorizations to simplify the argument.

For more examples, see \PMlinkexternal{approximate index of PM entries invoking WLOG}{http://planetmath.org/?op=search\&term=WLOG+without+loss+generality}.
%%%%%
%%%%%
\end{document}
