\documentclass[12pt]{article}
\usepackage{pmmeta}
\pmcanonicalname{FamousOpenQuestionsInMathematics}
\pmcreated{2013-03-22 14:42:52}
\pmmodified{2013-03-22 14:42:52}
\pmowner{rspuzio}{6075}
\pmmodifier{rspuzio}{6075}
\pmtitle{famous  open questions in mathematics}
\pmrecord{15}{36336}
\pmprivacy{1}
\pmauthor{rspuzio}{6075}
\pmtype{Example}
\pmcomment{trigger rebuild}
\pmclassification{msc}{00A20}
\pmrelated{TwoGeneratorProperty}
\pmrelated{MillenniumProblems}

\endmetadata

% this is the default PlanetMath preamble.  as your knowledge
% of TeX increases, you will probably want to edit this, but
% it should be fine as is for beginners.

% almost certainly you want these
\usepackage{amssymb}
\usepackage{amsmath}
\usepackage{amsfonts}

% used for TeXing text within eps files
%\usepackage{psfrag}
% need this for including graphics (\includegraphics)
%\usepackage{graphicx}
% for neatly defining theorems and propositions
%\usepackage{amsthm}
% making logically defined graphics
%%%\usepackage{xypic}

% there are many more packages, add them here as you need them

% define commands here
\begin{document}
Despite the fact that at least a generation of mathematicians has tried to solve these problems, they still remain open:

\begin{enumerate}
\item Goldbach conjecture
\item Twin prime conjecture
\item Poincar\'e conjecture
\item Riemann Hypothesis
\item Birch and Swinnerton-Dyer conjecture
\item P vs NP problem
\item Hodge conjecture
\item Solution to the Navier-Stokes equations
\item Collatz problem
\item Schanuel's conjecture
\item Beal conjecture
\item Irrationality of Euler's constant $\gamma$
\item Existence of an odd perfect number
\item Koethe conjecture
\item Invariant subspace problem
\item Lehmer's conjecture
\end{enumerate}

Open problems 3 to 8, together with the quest for a mathematical foundation explaining the mass gap property in Yang-Mills theory, constitute a collection of problems known as the Millennium Problems.  Please see \PMlinkexternal{http://www.claymath.org/millennium/}{http://www.claymath.org/millennium/} for more detail.

\textbf{Note:}
A mathematical foundation for the partial solution of the the mass gap property in Yang-Mills theory has been published by G. Cleaver and K. Tanaka (2000): ``Ratio of Quark Masses in Duality Theories.'',
\PMlinkexternal{on line}{http://arxiv.org/abs/hep-th/0002089v1}, and can be accessed as a 
\PMlinkexternal{PDF file}{http://arxiv.org/PS_cache/hep-th/pdf/0002/0002089v1.pdf} .

%%%%%
%%%%%
\end{document}
