\documentclass[12pt]{article}
\usepackage{pmmeta}
\pmcanonicalname{IntegralSign}
\pmcreated{2013-03-22 18:04:00}
\pmmodified{2013-03-22 18:04:00}
\pmowner{pahio}{2872}
\pmmodifier{pahio}{2872}
\pmtitle{integral sign}
\pmrecord{7}{40599}
\pmprivacy{1}
\pmauthor{pahio}{2872}
\pmtype{Definition}
\pmcomment{trigger rebuild}
\pmclassification{msc}{00A05}
\pmclassification{msc}{00A06}
\pmrelated{RiemannIntegral}
\pmrelated{RiemannStieltjesIntegral}
\pmrelated{Integral2}
\pmdefines{integrand}
\pmdefines{integrate}

% this is the default PlanetMath preamble.  as your knowledge
% of TeX increases, you will probably want to edit this, but
% it should be fine as is for beginners.

% almost certainly you want these
\usepackage{amssymb}
\usepackage{amsmath}
\usepackage{amsfonts}

% used for TeXing text within eps files
%\usepackage{psfrag}
% need this for including graphics (\includegraphics)
%\usepackage{graphicx}
% for neatly defining theorems and propositions
 \usepackage{amsthm}
% making logically defined graphics
%%%\usepackage{xypic}

% there are many more packages, add them here as you need them

% define commands here

\theoremstyle{definition}
\newtheorem*{thmplain}{Theorem}

\begin{document}
The {\em integral sign}
                               $$\int$$
is a stylised version of the {\em long s} letter.

The long s is a typographic variant of lowercase s, being the only lowercase s in the Carolingian minuscule script.\, The modern short (round) s appeared later to the ends of words, and has now replaced completely the long s in the antiqua script.

Gottfried Wilhelm Leibniz introduced the integral sign as the first letter s of the Latin word {\em summa} (`sum').\, The long shape of $\displaystyle\int$ may be thought to symbolically depict the fact that \PMlinkname{integral}{DefiniteIntegral} is a limiting case of sum.\\

A variant 
$$\oint$$
of the integral sign is used in integrals taken along a closed curve in $\mathbb{R}^2$ or a about a closed surface in $\mathbb{R}^3$; see e.g. Cauchy integral theorem, derivation of heat equation.\\

The function given after the integral sign, i.e. the function to be {\em integrated}, is the {\em integrand}.
%%%%%
%%%%%
\end{document}
