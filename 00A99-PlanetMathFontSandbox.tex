\documentclass[12pt]{article}
\usepackage{pmmeta}
\pmcanonicalname{PlanetMathFontSandbox}
\pmcreated{2013-03-22 16:45:32}
\pmmodified{2013-03-22 16:45:32}
\pmowner{PrimeFan}{13766}
\pmmodifier{PrimeFan}{13766}
\pmtitle{PlanetMath font sandbox}
\pmrecord{28}{38987}
\pmprivacy{1}
\pmauthor{PrimeFan}{13766}
\pmtype{Data Structure}
\pmcomment{trigger rebuild}
\pmclassification{msc}{00A99}

\endmetadata

% this is the default PlanetMath preamble.  as your knowledge
% of TeX increases, you will probably want to edit this, but
% it should be fine as is for beginners.

% almost certainly you want these
\usepackage{amssymb}
\usepackage{amsmath}
\usepackage{amsfonts}

% used for TeXing text within eps files
%\usepackage{psfrag}
% need this for including graphics (\includegraphics)
%\usepackage{graphicx}
% for neatly defining theorems and propositions
%\usepackage{amsthm}
% making logically defined graphics
%%%\usepackage{xypic}

% there are many more packages, add them here as you need them

% define commands here
\hsize 6in
\begin{document}
Let the definition of a failure, in the context of Riemann Hypothesis, be a non-root. Let $s_0$ be a non-root. Then $s = \psi (s_0) = s_0 + k\times\\psi(s_0))$ is a failure function since $\zeta(\Psi(s_0)$ generates infinitely many failures. Here k belongs to $\mathbb{N}.$

Proof: There is no loss of generality in takining $k = 1$. By Taylor's theorem $\zeta(s_0 + \zeta(s_0)) = e^{\zeta(s_0)} - 1$ since, by asumption, $\zeta(s_0)$ is not equal to $0$.


Poli\~nac's formula is $$\prod_{i = 1}^{\pi(n)} {p_i}^{\displaystyle \sum_{j = 1}^{\log_{p_i} n} \lfloor \frac{n}{{p_i}^j} \rfloor},$$

Ich ziemlich mu\ss{} hab eine Wiener Str\"udel!

``God made the integers, and all the rest is the work of man.''\\
\rightline{ --- Leopold Kronecker}

``A mathematician is a device for turning coffee into theorems.''\\
\rightline{ --- Pal Erd\H{o}s}

``Mathematics possesses not only truth, but supreme beauty --- a beauty cold and austere, like that of sculpture.''\\
\rightline{ --- Bertrand Russell}

``Mathematics may be defined as the subject in which we never know what we are talking about, nor whether what we are saying is true.''\\
\rightline{ --- Bertrand Russell}

``As far as the laws of mathematics refer to reality, they are not certain, and as far as they are certain, they do not refer to reality.''\\
\rightline{ --- Albert Einstein}

``I had a feeling once about Mathematics, that I saw it all --- Depth beyond depth was revealed to me --- the Byss and Abyss. I saw, as one might see the transit of Venus or even the Lord Mayor's Show, a quantity passing through infinity and changing its sign from plus to minus. I saw exactly why it happened and why the tergiversation was inevitable: and how the one step involved all the others. It was like politics. But it was after dinner and I let it go!''\\
\rightline{ --- Winston Churchill}

``Math, my dear boy, is nothing more than the lesbian sister of biology.''\\
\rightline{ --- Peter Griffin, {\it Family Guy}, ``When You Wish Upon A Weinstein''}

``How about we fire up the old Segway and find a nice quiet field to do long division in? I mean, a nice quiet field {\em in which to} do long division. Sorry, sorry, everybody.''\\
\rightline{ --- Neil Goldman, {\it Family Guy}, ``8 Simple Rules for Buying My Teenage Daughter''}

Consider $\sqrt{1+\sqrt{1+\sqrt{1+\sqrt{1+\sqrt{1+\ldots}}}}}$, etc., in \TeX{} as \verb'$\sqrt{1+\sqrt{1+\sqrt{1+\sqrt{1+\sqrt{1+\ldots}}}}}$'.

In \TeX{} and \LaTeX{} we may write $3^{4/7}$ or $3^{4 \div 7}$ or preferably, $3^{\frac{4}{7}}$.

Stanley Skewes in 1933 gave the lower bound $e^{e^{e^{79}}}$, approximately $10^{{10}^{{10}^{34}}}$.

Wolfgang Berg moved to the States in 1934, and Wac\l{}aw Sierpi\'nski followed in 1938.

$\mathbb{A} \mathbb{B} \mathbb{C} \mathbb{D} \mathbb{E} \mathbb{F} \mathbb{G} \mathbb{H} \mathbb{I} \mathbb{J} \mathbb{K} \mathbb{L} \mathbb{M} \mathbb{N} \mathbb{O} \mathbb{P} \mathbb{Q} \mathbb{R} \mathbb{S} \mathbb{T} \mathbb{U} \mathbb{V} \mathbb{W} \mathbb{X} \mathbb{Y} \mathbb{Z}$

$(1 + i)(1 - i)$ or $(1 + \mathbb{i})(1 - \mathbb{i})$

Stanis\l{}aw Ha\v{c}ek on the properties of $\hat{x}\bar{y}$

Stanis{\l}aw Ha\v{c}ek on the properties of $\hat{x}\bar{y}$

{\O}ystein Ore or \O{}ystein Ore

$3 * 4$, $f * g$, $f \ast g$

$\sqrt[3]{27} = 3$


\begin{flushleft}
\texttt{
(defun factorial) (n) \\
\ \ \ (cond ((= n 0) 1) \\
\ \ \ \ \ \ \ \ \ (t (* n (factorial (- n 1))))))}
\end{flushleft}

% from a post by Algeboy
$a \not\vert b$ or $a\nmid b$

brocard's CONJECTURE and subAnalytic set

\begin{flushleft}
\texttt{
\#include <planetMath.h> \\
while flag == True \{ \\
\quad value = oper1 \% oper2; \\
\quad counter++; \\
\}}
\end{flushleft}

\begin{verbatim}
while flag == True {
  value = oper1 % oper2;
  counter++;
}
\end{verbatim}

$\gcd(25,50)$

 or 

% $\GCD(25,50)$ uppercase is wrong

Also, $$\sum_{i = 0}^4 {8 \choose i} = 163.$$
%%%%%
%%%%%
\end{document}
