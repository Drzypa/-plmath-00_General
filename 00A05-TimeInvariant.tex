\documentclass[12pt]{article}
\usepackage{pmmeta}
\pmcanonicalname{TimeInvariant}
\pmcreated{2013-03-22 15:02:14}
\pmmodified{2013-03-22 15:02:14}
\pmowner{Mathprof}{13753}
\pmmodifier{Mathprof}{13753}
\pmtitle{time invariant}
\pmrecord{5}{36748}
\pmprivacy{1}
\pmauthor{Mathprof}{13753}
\pmtype{Definition}
\pmcomment{trigger rebuild}
\pmclassification{msc}{00A05}
\pmrelated{AutonomousSystem}
\pmdefines{time-invariant}
\pmdefines{shift-invariant}

\endmetadata

% this is the default PlanetMath preamble.  as your knowledge
% of TeX increases, you will probably want to edit this, but
% it should be fine as is for beginners.

% almost certainly you want these
\usepackage{amssymb}
\usepackage{amsmath}
\usepackage{amsfonts}

% used for TeXing text within eps files
%\usepackage{psfrag}
% need this for including graphics (\includegraphics)
%\usepackage{graphicx}
% for neatly defining theorems and propositions
%\usepackage{amsthm}
% making logically defined graphics
%%%\usepackage{xypic}

% there are many more packages, add them here as you need them

% define commands here
\begin{document}
A dynamical system is {\bf time-invariant} if its generating formula is dependent on state only, and independent of time.  A synonym for time-invariant is autonomous.  The complement of time-invariant is time-varying (or nonautonomous).

For example, the continuous-time system $\dot{x}=f(x,t)$ is time-invariant if and only if $f(x,t_1)\equiv f(x,t_2)$ for all valid states $x$ and times $t_1$ and $t_2$.  Thus $\dot{x}=\sin x$ is time-invariant, while $\dot{x}=\frac{\sin x}{1+t}$ is time-varying.

Likewise, the discrete-time system $x[n]=f[x,n]$ is time-invariant (also called shift-invariant) if and only if $f[x,n_1]\equiv f[x,n_2]$ for all valid states $x$ and time indices $n_1$ and $n_2$.  Thus $x[n]=2 x[n-1]$ is time-invariant, while $x[n]=2 n x[n-1]$ is time-varying.
%%%%%
%%%%%
\end{document}
