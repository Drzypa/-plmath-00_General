\documentclass[12pt]{article}
\usepackage{pmmeta}
\pmcanonicalname{MetricSystem}
\pmcreated{2013-03-22 18:29:37}
\pmmodified{2013-03-22 18:29:37}
\pmowner{PrimeFan}{13766}
\pmmodifier{PrimeFan}{13766}
\pmtitle{metric system}
\pmrecord{7}{41175}
\pmprivacy{1}
\pmauthor{PrimeFan}{13766}
\pmtype{Definition}
\pmcomment{trigger rebuild}
\pmclassification{msc}{00A05}
\pmrelated{DecimalPlace}

% this is the default PlanetMath preamble.  as your knowledge
% of TeX increases, you will probably want to edit this, but
% it should be fine as is for beginners.

% almost certainly you want these
\usepackage{amssymb}
\usepackage{amsmath}
\usepackage{amsfonts}

% used for TeXing text within eps files
%\usepackage{psfrag}
% need this for including graphics (\includegraphics)
%\usepackage{graphicx}
% for neatly defining theorems and propositions
%\usepackage{amsthm}
% making logically defined graphics
%%%\usepackage{xypic}

% there are many more packages, add them here as you need them

% define commands here

\begin{document}
\PMlinkescapeword{weight}
\PMlinkescapeword{measure}
\PMlinkescapeword{dimension}
\PMlinkescapeword{unit}
\PMlinkescapeword{scaling}
\PMlinkescapeword{simple}
\PMlinkescapeword{right}
\PMlinkescapeword{size}
\PMlinkescapeword{current}
\PMlinkescapeword{consistent}
\PMlinkescapeword{prefix}
\PMlinkescapeword{mass}
\PMlinkescapeword{group}

The {\em metric system} is a system of weights and measures first proposed in France and gradually coming closer to worldwide acceptance in which for each given dimension, each larger unit is ten times the smaller unit (or viceversa, each smaller unit is a tenth of the larger unit). This makes it easier and more convenient to convert larger units to smaller units and viceversa. In ancient systems prior to the metric system, units for a given dimension often related to one another by different scaling factors. For example, in Biblical times, a talent was 60 mina, a mina was 60 shekels and a shekel was 24 giru. To convert 17 talents to shekels was thus different from converting shekels to giru or talents to giru, etc. By contrast, to convert 17.29 kilograms to centigrams is a simple matter of moving the decimal point to the right (and adding significant zeroes as necessary), and the difference in converting kilograms to milligrams or hectograms entails only changes in where to move the decimal point and how far. Furthermore, the basic units are determined by measurements which are known to remain constant and double-checkable throughout the planet, and not ephemeral and hard-to-verify measurements (such as the shoe size of the current king). The system is further standardized by the use of consistent prefixes to add to the basic units to make them larger or smaller.

The basic units are:

\begin{tabular}{|l|l|}
\PMlinkname{Length}{BasicLength} & Meter \\
Mass & Gram \\
Time & Second \\
Temperature & Kelvin \\
% There are a few more units than these, I need to add them
\end{tabular}

The standard prefixes are:

\begin{tabular}{|l|r|}
% There are bigger
Kilo & 1000.0000 \\
Hecto & 100.0000 \\
Deca & 10.0000 \\
(none) & 1.0000 \\
Deci &  0.1000 \\
Centi & 0.0100 \\
Milli & 0.0010 \\
Micro & 0.0001 \\
% There are smaller. Please add more insignificant zeroes to the others if you add smaller prefixes
\end{tabular}

Measurement systems consistently based on 10 (rather than arbitrarily varying practical numbers like 12, 24 and 60) had been proposed since at least the 15th Century. It wasn't until after the French Revolution that the proposals were taken seriously. Nowadays, the metric system has been adopted by scientists all over the world, but the general population of the United States remains an important group of hold-outs.
%%%%%
%%%%%
\end{document}
