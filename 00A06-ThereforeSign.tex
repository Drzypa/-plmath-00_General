\documentclass[12pt]{article}
\usepackage{pmmeta}
\pmcanonicalname{ThereforeSign}
\pmcreated{2013-03-22 17:55:51}
\pmmodified{2013-03-22 17:55:51}
\pmowner{pahio}{2872}
\pmmodifier{pahio}{2872}
\pmtitle{therefore sign}
\pmrecord{8}{40425}
\pmprivacy{1}
\pmauthor{pahio}{2872}
\pmtype{Definition}
\pmcomment{trigger rebuild}
\pmclassification{msc}{00A06}
\pmclassification{msc}{00A05}
\pmrelated{RingsOfRationalNumbers}
\pmrelated{LogarithmicScale}

\endmetadata

% this is the default PlanetMath preamble.  as your knowledge
% of TeX increases, you will probably want to edit this, but
% it should be fine as is for beginners.

% almost certainly you want these
\usepackage{amssymb}
\usepackage{amsmath}
\usepackage{amsfonts}

% used for TeXing text within eps files
%\usepackage{psfrag}
% need this for including graphics (\includegraphics)
%\usepackage{graphicx}
% for neatly defining theorems and propositions
 \usepackage{amsthm}
% making logically defined graphics
%%%\usepackage{xypic}

% there are many more packages, add them here as you need them

% define commands here

\theoremstyle{definition}
\newtheorem*{thmplain}{Theorem}

\begin{document}
The {\em therefore sign} ``$\therefore$'' is used especially in handwritten mathematical text as a shorthand of the \PMlinkescapetext{word `therefore' or `thus', between some sentences} or relations:
$$S_1 \quad \therefore \; S_2$$
It expresses that $S_2$ has been inferred from $S_1$ or from $S_1$ and some preceding facts.  The sign is rather a punctuation mark than a symbol of logical implication.  Grammatically, it could be characterised a conclusive coordinating \PMlinkescapetext{conjunction}.  The usage of the symbol is not mathematically well-defined, and it often means `we can conclude in context' or `we can conclude from statements already shown or assumed to be true'.\\

For example, in determining an angle of a right triangle, one may write
$$\sin\alpha = \frac{1}{2} \quad \therefore \; \alpha = 30^\circ$$
Here, ``$\therefore$'' does not \PMlinkescapetext{represent} a proper implication ``$\Rightarrow$'', since the exact implication here would be
$$\sin\alpha = \frac{1}{2} \;\, \Leftarrow \;\, \alpha = 30^\circ.$$
To obtain a strict implication, we would need to introduce some of the context.
For instance, we know that, since $\alpha$ is an angle of a right triangle,
$0^\circ \le \alpha \le 90^\circ$,
so what we wrote could be interpreted as the implication
$$\sin\alpha = \frac{1}{2} \; \land \;
  0^\circ \le \alpha \le 90^\circ
  \;\, \Rightarrow \;\, \alpha = 30^\circ.$$
%%%%%
%%%%%
\end{document}
