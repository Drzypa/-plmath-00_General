\documentclass[12pt]{article}
\usepackage{pmmeta}
\pmcanonicalname{Superscript}
\pmcreated{2013-03-22 17:06:45}
\pmmodified{2013-03-22 17:06:45}
\pmowner{PrimeFan}{13766}
\pmmodifier{PrimeFan}{13766}
\pmtitle{superscript}
\pmrecord{7}{39412}
\pmprivacy{1}
\pmauthor{PrimeFan}{13766}
\pmtype{Definition}
\pmcomment{trigger rebuild}
\pmclassification{msc}{00A99}

\endmetadata

% this is the default PlanetMath preamble.  as your knowledge
% of TeX increases, you will probably want to edit this, but
% it should be fine as is for beginners.

% almost certainly you want these
\usepackage{amssymb}
\usepackage{amsmath}
\usepackage{amsfonts}

% used for TeXing text within eps files
%\usepackage{psfrag}
% need this for including graphics (\includegraphics)
%\usepackage{graphicx}
% for neatly defining theorems and propositions
%\usepackage{amsthm}
% making logically defined graphics
%%%\usepackage{xypic}

% there are many more packages, add them here as you need them

% define commands here

\begin{document}
A {\em superscript} is a symbol or group of symbols written above, and usually a bit to the right, of another symbol or group of symbols in order to show a relation among the two. The most common use of superscripts in mathematics is for notating exponentiation. For example, in the expression $4^7$, the 7 is a superscript to the 4. Iterated sum and product notation also uses superscripts most of the time, to show the iterator end values, and the same goes for integral notation. Notation of iterated functions also benefit from using superscripts to cut down on nested parentheses, e.g., $f^3(x) = f(f(f(x)))$; examples of iterated functions include the iterated totient function $\phi^i(n)$ and iterated sum of divisors function $\sigma^i(n)$. The contravariant indices of tensors are also denoted by superscripts.

Nonfiction literature in general uses number superscripts to indicate footnotes with the corresponding numbers. This usage is often avoided in mathematical books and papers to prevent possible confusion with exponentiation.
%%%%%
%%%%%
\end{document}
