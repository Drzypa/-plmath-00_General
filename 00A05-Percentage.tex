\documentclass[12pt]{article}
\usepackage{pmmeta}
\pmcanonicalname{Percentage}
\pmcreated{2013-03-22 16:37:56}
\pmmodified{2013-03-22 16:37:56}
\pmowner{CompositeFan}{12809}
\pmmodifier{CompositeFan}{12809}
\pmtitle{percentage}
\pmrecord{5}{38833}
\pmprivacy{1}
\pmauthor{CompositeFan}{12809}
\pmtype{Definition}
\pmcomment{trigger rebuild}
\pmclassification{msc}{00A05}

% this is the default PlanetMath preamble.  as your knowledge
% of TeX increases, you will probably want to edit this, but
% it should be fine as is for beginners.

% almost certainly you want these
\usepackage{amssymb}
\usepackage{amsmath}
\usepackage{amsfonts}

% used for TeXing text within eps files
%\usepackage{psfrag}
% need this for including graphics (\includegraphics)
%\usepackage{graphicx}
% for neatly defining theorems and propositions
%\usepackage{amsthm}
% making logically defined graphics
%%%\usepackage{xypic}

% there are many more packages, add them here as you need them

% define commands here

\begin{document}
A {\em percentage} is a ratio expressed in terms of a unit being 100. A percentage is usually denoted by the symbol ``\%.'' For example, $20\%$ of $\$700.47$ is $\$175.12$ (using fixed point arithmetic to two decimal places for display).

On most calculators, one sure way to calculate a percentage is by entering a decimal point before the desired percentage and multiplying that by the amount one wishes to calculate the percentage of. Some calculators have a percentage key.

When tipping at most restaurants in the United States, it is customary to tip 15\% of the check to the waiter for parties of as much as four people. One common shortcut is to divide by 10 (by moving the decimal point to the left) and then add half of that amount.

Note that the percentage symbol \% (Shift-5 in most American keyboard layouts) is overloaded in \TeX{} as a comment start indicator and in Mathematica as a shortcut for referring to the previous output.
%%%%%
%%%%%
\end{document}
