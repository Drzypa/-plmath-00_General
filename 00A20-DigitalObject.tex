\documentclass[12pt]{article}
\usepackage{pmmeta}
\pmcanonicalname{DigitalObject}
\pmcreated{2013-03-22 15:30:38}
\pmmodified{2013-03-22 15:30:38}
\pmowner{CWoo}{3771}
\pmmodifier{CWoo}{3771}
\pmtitle{digital object}
\pmrecord{11}{37374}
\pmprivacy{1}
\pmauthor{CWoo}{3771}
\pmtype{Definition}
\pmcomment{trigger rebuild}
\pmclassification{msc}{00A20}

\endmetadata

% this is the default PlanetMath preamble.  as your knowledge
% of TeX increases, you will probably want to edit this, but
% it should be fine as is for beginners.

% almost certainly you want these
\usepackage{amssymb}
\usepackage{amsmath}
\usepackage{amsfonts}

% used for TeXing text within eps files
%\usepackage{psfrag}
% need this for including graphics (\includegraphics)
%\usepackage{graphicx}
% for neatly defining theorems and propositions
%\usepackage{amsthm}
% making logically defined graphics
%%%\usepackage{xypic}

% there are many more packages, add them here as you need them

% define commands here
\begin{document}
A \emph{digital object} in a digital library is the textual or multimedia data and the metadata.

Formally, a digital object $DO$ is a quadruple $(h, SM, ST, SS)$ where
\begin{enumerate}
\item $h\in H$, where $H$ is a set of universally unique handles (labels);
\item $SM = \lbrace sm_1, sm_2, . . . , sm_n\rbrace$ is a set of streams;
\item $ST = \lbrace st_1, st_2, . . . , st_m\rbrace$ is a set of structural metadata specifications; and
\item $SS = \lbrace stsm_1, stsm_2, . . . , stsm_p\rbrace$ is a set of Structured Streams functions defined from the streams in $SM$ and from the structures in $ST$.
\end{enumerate}
%%%%%
%%%%%
\end{document}
