\documentclass[12pt]{article}
\usepackage{pmmeta}
\pmcanonicalname{Multiplication}
\pmcreated{2013-03-22 16:35:37}
\pmmodified{2013-03-22 16:35:37}
\pmowner{PrimeFan}{13766}
\pmmodifier{PrimeFan}{13766}
\pmtitle{multiplication}
\pmrecord{10}{38789}
\pmprivacy{1}
\pmauthor{PrimeFan}{13766}
\pmtype{Definition}
\pmcomment{trigger rebuild}
\pmclassification{msc}{00A06}
\pmclassification{msc}{11B25}
\pmclassification{msc}{00A05}
\pmrelated{Product}
\pmrelated{ProductOfNegativeNumbers}
\pmrelated{FactorsWithMinusSign}

% this is the default PlanetMath preamble.  as your knowledge
% of TeX increases, you will probably want to edit this, but
% it should be fine as is for beginners.

% almost certainly you want these
\usepackage{amssymb}
\usepackage{amsmath}
\usepackage{amsfonts}

% used for TeXing text within eps files
%\usepackage{psfrag}
% need this for including graphics (\includegraphics)
%\usepackage{graphicx}
% for neatly defining theorems and propositions
%\usepackage{amsthm}
% making logically defined graphics
%%%\usepackage{xypic}

% there are many more packages, add them here as you need them

% define commands here

\begin{document}
{\em Multiplication} is a mathematical operation in which two or more numbers are added up to themselves by a factor of other numbers. For example, $2 \times 3 = 2 + 2 + 2 = 3 + 3 = 6$. The numbers may be real, imaginary or complex, they may be integers or fractions. Among real numbers, if an odd number of multiplicands are negative, the overall result is negative; if an even number of multiplicands are negative, the overall result is positive. Two examples: $(-3) \times (-5) = 15$; $(-2) \times (-3) \times (-5) = (-30)$.

The usual operator is the cross with its four arms of equal length pointing northeast, northwest, southeast and southwest: $\times$. Other options are the central dot $\cdot$ and the tacit multiplication operator. In many computer programming languages the asterisk is often used as it is almost always available on the keyboard (Shift-8 in most American layouts, as well as dedicated key if the keyboard has a numeric keypad), and this is the operator likely to be used in a computer implementation of a reverse Polish notation calculator. In Mathematica, the space can sometimes function as a multiplication operator, but  more experienced users warn novices not to rely on this feature.

Just as with addition, multiplication is commutative: $xyz = xzy = yxz$, etc.

The iterative operator is the Greek capital letter pi: $$\prod_{i = 1}^n a_i,$$ which is a compact way of writing $a_1 \times a_2 \times \ldots \times a_n$.

Multiplication of complex numbers is helped by the following identity: $(a + bi) \times (x + yi)=(ax - by) + (ay + bx)i$. To give three examples: $(17 + 29i)(11 + 38i) = -915 + 965i$ (the result has both real and imaginary parts), $(1 + 2i)(1 - 2i) = 5$ (the result is a real prime) and $(4 + 7i)(7 + 4i) = 65i$ (the result has only an imaginary part).
%%%%%
%%%%%
\end{document}
