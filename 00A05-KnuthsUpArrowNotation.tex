\documentclass[12pt]{article}
\usepackage{pmmeta}
\pmcanonicalname{KnuthsUpArrowNotation}
\pmcreated{2013-03-22 12:58:43}
\pmmodified{2013-03-22 12:58:43}
\pmowner{Henry}{455}
\pmmodifier{Henry}{455}
\pmtitle{Knuth's up arrow notation}
\pmrecord{7}{33350}
\pmprivacy{1}
\pmauthor{Henry}{455}
\pmtype{Definition}
\pmcomment{trigger rebuild}
\pmclassification{msc}{00A05}
\pmsynonym{up-arrow}{KnuthsUpArrowNotation}
\pmsynonym{up arrow}{KnuthsUpArrowNotation}
\pmsynonym{up-arrow notation}{KnuthsUpArrowNotation}
\pmsynonym{up arrow notation}{KnuthsUpArrowNotation}
\pmsynonym{Knuth notation}{KnuthsUpArrowNotation}
\pmrelated{ConwaysChainedArrowNotation}

% this is the default PlanetMath preamble.  as your knowledge
% of TeX increases, you will probably want to edit this, but
% it should be fine as is for beginners.

% almost certainly you want these
\usepackage{amssymb}
\usepackage{amsmath}
\usepackage{amsfonts}

% used for TeXing text within eps files
%\usepackage{psfrag}
% need this for including graphics (\includegraphics)
%\usepackage{graphicx}
% for neatly defining theorems and propositions
%\usepackage{amsthm}
% making logically defined graphics
%%%\usepackage{xypic}

% there are many more packages, add them here as you need them

% define commands here
%\PMlinkescapeword{theory}
\begin{document}
\emph{Knuth's up arrow noation} is a way of writing numbers which would be unwieldy in standard decimal notation.  It expands on the exponential notation $m\uparrow n=m^n$.  Define $m\uparrow\uparrow 0=1$ and $m \uparrow\uparrow n=m\uparrow(m\uparrow\uparrow [n-1])$.

Obviously $m\uparrow\uparrow 1=m^1=m$, so $3\uparrow\uparrow 2=3^{3\uparrow\uparrow 1}=3^3=27$, but $2\uparrow\uparrow 3=2^{2 \uparrow\uparrow 2}=2^{2^{2\uparrow\uparrow 1}}=2^{(2^2)}=16$.

In general, $m\uparrow\uparrow n=m^{m^{\cdots^m}}$, a tower of height $n$.

Clearly, this process can be extended: $m\uparrow\uparrow\uparrow 0=1$ and $m\uparrow\uparrow\uparrow n=m\uparrow\uparrow(m\uparrow\uparrow\uparrow [n-1])$.

An alternate notation is to write $m^{(i)}n$ for $m\underbrace{\uparrow\cdots\uparrow}_{i-2 \text{~times}}n$.  ($i-2$ times because then $m^{(2)}n=m\cdot n$ and $m^{(1)}n=m+n$.)  Then in general we can define $m^{(i)}n=m^{(i-1)}(m^{(i)}(n-1))$.

To get a sense of how quickly these numbers grow, $3\uparrow\uparrow\uparrow 2=3\uparrow\uparrow 3$ is more than seven and a half trillion, and the numbers continue to grow much more than exponentially.
%%%%%
%%%%%
\end{document}
