\documentclass[12pt]{article}
\usepackage{pmmeta}
\pmcanonicalname{Pathological}
\pmcreated{2013-03-22 14:41:56}
\pmmodified{2013-03-22 14:41:56}
\pmowner{CWoo}{3771}
\pmmodifier{CWoo}{3771}
\pmtitle{pathological}
\pmrecord{10}{36310}
\pmprivacy{1}
\pmauthor{CWoo}{3771}
\pmtype{Definition}
\pmcomment{trigger rebuild}
\pmclassification{msc}{00A20}

% this is the default PlanetMath preamble.  as your knowledge
% of TeX increases, you will probably want to edit this, but
% it should be fine as is for beginners.

% almost certainly you want these
\usepackage{amssymb}
\usepackage{amsmath}
\usepackage{amsfonts}
\usepackage{amsthm}

\usepackage{mathrsfs}

% used for TeXing text within eps files
%\usepackage{psfrag}
% need this for including graphics (\includegraphics)
%\usepackage{graphicx}
% for neatly defining theorems and propositions
%
% making logically defined graphics
%%%\usepackage{xypic}

% there are many more packages, add them here as you need them

% define commands here

\newcommand{\sR}[0]{\mathbb{R}}
\newcommand{\sC}[0]{\mathbb{C}}
\newcommand{\sN}[0]{\mathbb{N}}
\newcommand{\sZ}[0]{\mathbb{Z}}

 \usepackage{bbm}
 \newcommand{\Z}{\mathbbmss{Z}}
 \newcommand{\C}{\mathbbmss{C}}
 \newcommand{\R}{\mathbbmss{R}}
 \newcommand{\Q}{\mathbbmss{Q}}



\newcommand*{\norm}[1]{\lVert #1 \rVert}
\newcommand*{\abs}[1]{| #1 |}



\newtheorem{thm}{Theorem}
\newtheorem{defn}{Definition}
\newtheorem{prop}{Proposition}
\newtheorem{lemma}{Lemma}
\newtheorem{cor}{Corollary}
\begin{document}
In mathematics, a \emph{pathological object} is mathematical
object that has a highly unexpected \PMlinkescapetext{property}. 

Pathological objects are typically percieved to, in some sense, be 
badly behaving. On the other hand, they are perfectly properly
defined mathematical objects. Therefore this ``bad behaviour'' can
simply be seen as a contradiction with our intuitive 
picture of how a certain object should behave. 

\subsubsection*{Examples}
\begin{itemize}
\item A very famous pathological function is the 
Weierstrass function, which is a continuous function 
that is nowhere differentiable. 
\item The Peano space filling curve. This pathological curve
maps the unit interval $[0,1]$ continuously onto $[0,1]\times [0,1]$. 
\item The Cantor set. This is subset of the interval $[0,1]$
has the pathological property that it is uncountable 
yet its measure is zero. 
\item The Dirichlet's function from $\R$ to $\R$ is continuous at every
irrational point and discontinuous at every rational point. 
\item Ackermann Function.
\end{itemize}

See also \cite{wiki}. 

\begin{thebibliography}{9}
\bibitem{wiki}Wikipedia \PMlinkexternal{entry on pathological, mathematics}{http://en.wikipedia.org/wiki/Pathological (mathematics)}. 
\end{thebibliography}
%%%%%
%%%%%
\end{document}
