\documentclass[12pt]{article}
\usepackage{pmmeta}
\pmcanonicalname{Subtraction}
\pmcreated{2013-03-22 16:35:31}
\pmmodified{2013-03-22 16:35:31}
\pmowner{PrimeFan}{13766}
\pmmodifier{PrimeFan}{13766}
\pmtitle{subtraction}
\pmrecord{6}{38787}
\pmprivacy{1}
\pmauthor{PrimeFan}{13766}
\pmtype{Definition}
\pmcomment{trigger rebuild}
\pmclassification{msc}{00A06}
\pmclassification{msc}{00A05}
\pmclassification{msc}{11B25}

\endmetadata

% this is the default PlanetMath preamble.  as your knowledge
% of TeX increases, you will probably want to edit this, but
% it should be fine as is for beginners.

% almost certainly you want these
\usepackage{amssymb}
\usepackage{amsmath}
\usepackage{amsfonts}

% used for TeXing text within eps files
%\usepackage{psfrag}
% need this for including graphics (\includegraphics)
%\usepackage{graphicx}
% for neatly defining theorems and propositions
%\usepackage{amsthm}
% making logically defined graphics
%%%\usepackage{xypic}

% there are many more packages, add them here as you need them

% define commands here

\begin{document}
{\em Subtraction} is a mathematical operation in which the value of a number is decreased by the values of one or more other numbers. Subtraction can be seen as a kind of addition with negative numbers. For example, $7 - 4 = 7 + (-4) = 3$.

The usual operator looks like a dash: $-$. This operator is used in standard infix notation as well as in Polish notation and reverse Polish notation.

Besides the possibility of overflow or underflow, subtraction presents no problems for fixed point arithmetic provided the operands are representable in fixed point to begin with.
%%%%%
%%%%%
\end{document}
