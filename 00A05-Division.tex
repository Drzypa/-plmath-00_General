\documentclass[12pt]{article}
\usepackage{pmmeta}
\pmcanonicalname{Division}
\pmcreated{2014-08-08 17:51:29}
\pmmodified{2014-08-08 17:51:29}
\pmowner{pahio}{2872}
\pmmodifier{pahio}{2872}
\pmtitle{division}
\pmrecord{29}{36148}
\pmprivacy{1}
\pmauthor{pahio}{2872}
\pmtype{Definition}
\pmclassification{msc}{00A05}
\pmclassification{msc}{12E99}
\pmrelated{InverseFormingInProportionToGroupOperation}
\pmrelated{DivisionInGroup}
\pmrelated{ConjugationMnemonic}
\pmrelated{Difference2}
\pmrelated{UniquenessOfDivisionAlgorithmInEuclideanDomain}
\pmdefines{quotient}
\pmdefines{ratio}
\pmdefines{fundamental property of quotient}
\pmdefines{reduction}

% this is the default PlanetMath preamble.  as your knowledge
% of TeX increases, you will probably want to edit this, but
% it should be fine as is for beginners.

% almost certainly you want these
\usepackage{amssymb}
\usepackage{amsmath}
\usepackage{amsfonts}

% used for TeXing text within eps files
%\usepackage{psfrag}
% need this for including graphics (\includegraphics)
%\usepackage{graphicx}
% for neatly defining theorems and propositions
%\usepackage{amsthm}
% making logically defined graphics
%%%\usepackage{xypic}

% there are many more packages, add them here as you need them

% define commands here
\begin{document}
{\it Division} is the operation which assigns to every two numbers (or more generally, elements of a field) $a$ and $b$ their quotient or ratio, provided that the latter, $b$, is distinct from zero.

The {\it quotient} (or {\it ratio})\, $\frac{a}{b}$\, of $a$ and $b$ may be defined as such a number (or element of the field) $x$ that\, $b \cdot x = a$.\, Thus,
                     $$b\cdot\frac{a}{b} = a,$$
which is the ``fundamental property of quotient''. 

The quotient of the numbers $a$ and $b$ ($\neq 0$) is a 
uniquely determined number, since if one had
$$\frac{a}{b} = x \neq y = \frac{a}{b},$$
then we could write 
$$b(x-y) = bx-by = a-a = 0$$
from which the supposition $b \neq 0$ would imply $x-y = 0$, i.e. 
$x = y$.

The explicit general expression for $\frac{a}{b}$ is
                  $$\frac{a}{b} = b^{-1}\cdot a$$
where $b^{-1}$ is the inverse number (the multiplicative inverse) of $a$, because
                $$b(b^{-1}a) = (bb^{-1})a = 1a = a.$$

\begin{itemize}
 \item For positive numbers the quotient may be obtained by performing the division algorithm with $a$ and $b$.\, If\, $a > b > 0$,\, then $\frac{a}{b}$ indicates how many times $b$ fits in $a$.
  \item The quotient of $a$ and $b$ does not change if both numbers (elements) are multiplied (or divided, which \PMlinkescapetext{action} is called {\em reduction}) by any \,$k \neq 0$:
$$\frac{ka}{kb} = (kb)^{-1}(ka) = b^{-1}k^{-1}ka = b^{-1}a = \frac{a}{b}$$
So we have the method for getting the quotient of complex numbers,
       $$\frac{a}{b} = \frac{\bar{b}a}{\bar{b}b},$$
where $\bar{b}$ is the complex conjugate of $b$, and the quotient of \PMlinkname{square root polynomials}{SquareRootOfSquareRootBinomial}, e.g.
   $$\frac{1}{5+2\sqrt{2}} = \frac{5-2\sqrt{2}}{(5-2\sqrt{2})(5+2\sqrt{2})} = 
\frac{5-2\sqrt{2}}{25-8} = \frac{5-2\sqrt{2}}{17};$$
in the first case one aspires after a real and in the second case after a rational denominator.
 \item The division is neither associative nor commutative, but it is right distributive over addition:
         $$\frac{a+b}{c} = \frac{a}{c}+\frac{b}{c}$$
\end{itemize}


\end{document}
