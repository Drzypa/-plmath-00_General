\documentclass[12pt]{article}
\usepackage{pmmeta}
\pmcanonicalname{EmptyProduct}
\pmcreated{2013-03-22 14:48:13}
\pmmodified{2013-03-22 14:48:13}
\pmowner{pahio}{2872}
\pmmodifier{pahio}{2872}
\pmtitle{empty product}
\pmrecord{15}{36458}
\pmprivacy{1}
\pmauthor{pahio}{2872}
\pmtype{Definition}
\pmcomment{trigger rebuild}
\pmclassification{msc}{00A99}
\pmrelated{EmptySet}
\pmrelated{IndeterminateForm}
\pmrelated{LocallyEuclidean}
\pmrelated{AnalyticContinuationOfGammaFunction}
\pmrelated{Introducing0thPower}
\pmrelated{EmptySum}

\endmetadata

% this is the default PlanetMath preamble.  as your knowledge
% of TeX increases, you will probably want to edit this, but
% it should be fine as is for beginners.

% almost certainly you want these
\usepackage{amssymb}
\usepackage{amsmath}
\usepackage{amsfonts}

% used for TeXing text within eps files
%\usepackage{psfrag}
% need this for including graphics (\includegraphics)
%\usepackage{graphicx}
% for neatly defining theorems and propositions
%\usepackage{amsthm}
% making logically defined graphics
%%%\usepackage{xypic}

% there are many more packages, add them here as you need them

% define commands here
\begin{document}
The {\em empty product} of numbers is the borderline case of product, where the number of \PMlinkescapetext{factors is zero, i.e. the set of the factors} is empty. \,The most usual examples are the following.
\begin{itemize}
\item The \PMlinkname{zeroth power}{Introducing0thPower} of a non-zero number:\, $a^0$
\item The factorial of 0:\, 0!
\item The \PMlinkname{prime factor presentation}{FundamentalTheoremOfArithmetics} of unity, which has no prime factors
\end{itemize}
The value of the empty sum of numbers is equal to the additive identity number, 0.\, Similarly, the empty product of numbers is equal to the 
\PMlinkname{multiplicative identity}{Unity} number, 1.

\textbf{Note.}\, When considering the complex numbers as pairs of real numbers one often identifies the pairs $(x,\,0)$ and the reals $x$.\, In this sense one can think that the Cartesian product $\mathbb{R}\times\{0\}$ is equal to $\mathbb{R}$.\, This seems to \PMlinkescapetext{mean} the equation
$$\mathbb{R}\times\mathbb{R}^0 = \mathbb{R}^{1+0} = \mathbb{R}^1 = \mathbb{R},$$
although the \PMlinkname{associativity}{GeneralAssociativity} of Cartesian product is nowhere stated.\, Nevertheless, it is sometimes natural to define that the Cartesian product of an empty collection of sets equals to a set with one element; so it may \PMlinkescapetext{mean} that e.g.\, $\mathbb{R}^0 = \{0\}.$

One can also consider empty products in categories.
It follows directly from the definition that an object in a category
is a \PMlinkname{product}{CategoricalDirectProduct}
of an empty family of objects in the category
if and only if it is a terminal object of the category.
Sets are a special case of this:
in the category of sets the singletons are the terminal objects,
so the empty product exists and is a singleton.
%%%%%
%%%%%
\end{document}
