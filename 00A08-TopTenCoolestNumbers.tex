\documentclass[12pt]{article}
\usepackage{pmmeta}
\pmcanonicalname{TopTenCoolestNumbers}
\pmcreated{2013-03-22 15:38:03}
\pmmodified{2013-03-22 15:38:03}
\pmowner{rspuzio}{6075}
\pmmodifier{rspuzio}{6075}
\pmtitle{top ten coolest numbers}
\pmrecord{17}{37563}
\pmprivacy{1}
\pmauthor{rspuzio}{6075}
\pmtype{Feature}
\pmcomment{trigger rebuild}
\pmclassification{msc}{00A08}

\endmetadata

% this is the default PlanetMath preamble.  as your knowledge
% of TeX increases, you will probably want to edit this, but
% it should be fine as is for beginners.

% almost certainly you want these
\usepackage{amssymb}
\usepackage{amsmath}
\usepackage{amsfonts}
\usepackage{amsthm}
\usepackage{url}

% used for TeXing text within eps files
%\usepackage{psfrag}
% need this for including graphics (\includegraphics)
%\usepackage{graphicx}
% for neatly defining theorems and propositions
%\usepackage{amsthm}
% making logically defined graphics
%%%\usepackage{xypic}

% there are many more packages, add them here as you need them

% define commands here

\newcommand{\mc}{\mathcal}
\newcommand{\mb}{\mathbb}
\newcommand{\mf}{\mathfrak}
\newcommand{\ol}{\overline}
\newcommand{\ra}{\rightarrow}
\newcommand{\la}{\leftarrow}
\newcommand{\La}{\Leftarrow}
\newcommand{\Ra}{\Rightarrow}
\newcommand{\nor}{\vartriangleleft}
\newcommand{\Gal}{\text{Gal}}
\newcommand{\GL}{\text{GL}}
\newcommand{\Z}{\mb{Z}}
\newcommand{\R}{\mb{R}}
\newcommand{\Q}{\mb{Q}}
\newcommand{\C}{\mb{C}}
\newcommand{\<}{\langle}
\renewcommand{\>}{\rangle}
\begin{document}
This is an attempt to give a count-down of the top ten coolest
numbers.  Let's first admit that this is a highly subjective
ordering--one person's 14.38 is another's $\frac{\pi^2}{6}$.  The
astute (or probably simply ``awake'') reader will notice, for example,
a definite bias toward numbers interesting to a number theorist in the
below list.  (On the other hand, who better to gauge the coolness of
numbers than a number-theorist...)  But who knows?  Maybe I can be
convinced that I've left something out, or that my ordering should be
switched in some cases.  But let's first set down some ground rules.

\textbf{What's in the list?}  What makes a number cool?  I think a
word that sums up the key characteristic of cool numbers is
``canonicity''.  Numbers that appear in this list should be somehow
fundamental to the nature of mathematics.  They could represent a
fundamental fact or theorem of mathematics, be the first instance of
an amazing class of numbers, be omnipresent in modern mathematics, or
simply have an eerily long list of interesting properties.  Perhaps a
more appropriate question to ask is the following:

\textbf{What's \emph{not} in the list?}  There are some really awesome
numbers that I didn't include in the list.  I'll go through several
examples to get a feel for what sorts of numbers don't fit the
characteristics mentioned above.

Shocking as it may seem, I first disqualify the constants appearing in
Euler's formula $e^{i\pi}+1=0$.  This was a tough decision.  Perhaps
these five ($e$, $i$, $\pi$, 1, and 0) belong at the top of the list, or
perhaps they're just too fundamentally important to be considered
exceptionally \emph{cool}.  Or maybe they're just so clich\'e'd that
we'll get a significantly more interesting list by excluding them.

Also disqualified are numbers whose primary significance is cultural,
rather than mathematical: despite being the answer to life, the
universe, and everything, 42 is a comparatively uninteresting number.  Similarly not included in the list were
876-5309, 666, and the first illegal prime number.  Similarly disqualified were constants of nature like Newton's $g$ and $G$, the fine structure constant, Avogadro's number, etc.

Finally, I disqualified number that were highly non-canonical in
construction.  For example, the prime constant and Champoleon's
constant are both mathematically interesting, but only because they
were, at least in an admittedly vague sense, constructed to be as
such.  Also along these lines are numbers like G63 and Skewe's
constant, which while mathematically interesting because of roles
they've played in proofs, are not inherently interesting in and of
themselves.

That said, I felt free to ignore any of these disqualifications when I
felt like it.  I hope you enjoy the following list, and I welcome
feedback.\\

\textbf{{\Large Honorable Mentions}}\\
\begin{itemize}
\item 65,537 - This number is arguably the number with the most
potential.  It's currently the largest Fermat prime known.  If it
turns out to be \emph{the} largest Fermat prime, it might earn itself
a place on the list, by virtue of thus also being the largest odd
value of $n$ for which an $n$-gon is constructible using only a rule
and compass.
\item Conway's constant - The construction of the number can be found
here \url{http://mathworld.wolfram.com/ConwaysConstant.html}.  Though
this number has some remarkable properties (not the least of which is
being unexpectedly algebraic), it's completely non-canonical
construction kept it from overtaking any of our list's current
members.
\item 1728 and 1729 - This pair just didn't have quite enough going
  for them to make it.  1728 is an important $j$-invariant of elliptic
  curves and modular forms, and is a perfect cube.  1729 happens to be
  the third Carmichael number, but the primary motivation for
  including 1729 is because of the mathematical folklore associated it
  to being the first \emph{taxicab number}, making it more interesting
  (math-)historically than mathematically.
\item 28 - Aside from being a perfect number, a fairly interesting
  fact in and of itself, the number 28 has some extra interesting
  ``aliquot'' properties that propels it beyond other perfect numbers.
  Specifically, the largest known collection of sociable numbers has
  cardinality 28, and though this might seem a silly feat in and of
  itself, the fact that sociable numbers and perfect numbers are
  so closely related may reveal something slightly more profound about
  28 than it just being perfect.
\item 26 - being the only number between a square and a cube is pretty cool; as well as that to Actuaries, this number has relavance to life expectancy - in than it is a turning point. (this will change over time as is just a tenuous arguement to support giving 26 a mention!).
\end{itemize}
And now, on to the top 10:

\textbf{{\Large \#10) The Golden Ratio, $\phi$}}\\ This was a tough
one.  Yes, it's cool that it satisfies the property that its
reciprocal is one less than it, but this merely reflects that it's a
root of the wholly generic polynomial $x^2-x-1=0$.  Yes, it's cool
that it may have an aesthetic quality revered by the Greeks, but this
is void from consideration for being non-mathematical.  Only slightly
less canonical is that it gives the limiting ratio of subsequent
Fibonacci numbers.  Redeeming it, however, is that this generalizes to
\emph{all} ``Fibonacci-like'' sequences, and is the solution to two
sort of canonical operations: 
\begin{align*}
\frac{1}{1+\frac{1}{1+\frac{1}{1+\frac{1}{\ddots}}}}   
\end{align*}
and
\begin{align*}
\sqrt{1+\sqrt{1+\sqrt{1+\cdots}}}
\end{align*}
Also, this number plays an important role in the hstory of algebraic
number theory.  The field it generates is the first known example of a
field in which unique factorization fails.  Trying to come to grips with 
this fact led to the invention of ideal theory, class nubers, etc.

\textbf{{\Large \#9) 691}}\\ The prime number 691 made it on here for
a couple of reasons: First, it's prime, but more importantly, it's the
first example of an \emph{irregular} prime, a class of primes of
immense importance in algebraic number theory.  (A word of caution:
it's not the \emph{smallest} irregular prime, but it's the one that
corresponds to the earliest Bernoulli number, $B_{12}$, so 691 is only
``first'' in that sense).  It also shows up as a coefficient of every
non-constant term in the $q$-expansion of the modular form
$E_{12}(z)$. Further testimony to the arithmetic significance is its
seemingly magical appearance in the algebraic $K$-theory: It's known
that $K_{22}(\Z)$ surjects onto 691.

\textbf{{\Large \#8) 78,557}}\\ The number 78,557 is here to represent
an amazing class of numbers called \emph{Sierpinski} numbers, defined
to be numbers $k$ such that $k2^n+1$ is composite for \emph{every}
$n\geq 1$.  That such numbers exist is flabbergasting...we know from
Dirichlet's theorem that primes occur infinitely often in non-trivial
arithmetic sequences.  Though the sequence formed by $78557\cdot
2^n+1$ isn't arithmetic, it certainly doesn't behave multiplicatively
either, and there's no apparent reason why there shouldn't be a large
(or infinite) number of primes in \emph{every} such sequence.  This
notwithstanding, Sierpinski's composite number theorem proves there
are in fact \emph{infinitely} many odd such numbers $k$.  As a small
disclaimer, though it's proven that 78,557 is indeed a Sierpinski
number, it is not quite yet known that it is the smallest.  There are
17 positive integers smaller than 78,557 not yet known to be
non-Sierpinski.

\textbf{{\Large \#7) $\frac{\pi^2}{6}$}}\\ Perhaps the first striking
this about this number is that it is the sum of the reciprocals of the
positive integer squares:
\begin{align*}
1+\frac{1}{4}+\frac{1}{9}+\cdots+\frac{1}{n^2}+\cdots=\frac{\pi^2}{6}.
\end{align*}
Though the choice of $2$ here for the exponent is somewhat
non-canonical (i.e. we've just noted that $\zeta(2)=\frac{\pi^2}{6}$,
where $\zeta$ stands for the Riemann zeta function), and that this is
largely interesting for math-historical reasons (it was the first sum
of this type that Euler computed), we can at least include it here to
represent the amazing array of numbers of the form $\zeta(n)$ for $n$
a positive integer.  This class of numbers incorporates two amazing
and seemingly disparate collections, depending on whether $n$ is even
(in which case $\zeta(n)$ is known to be a rational multiple of
$\pi^n$) or odd (in which case extremely little is known, even for
$\zeta(3)$.

Further, there's something slightly more canonical about the fact that
its reciprocal, $\frac{6}{\pi^2}$, gives the ``probability'' (in a
suitably-defined sense) that two randomly chosen positive integers are
relatively prime.

 \textbf{{\Large \#6) Feigenbaum's constant}}\\ - This is the entry on
 this the list with which I have the least familiarity.  The one thing
 going for it is that it seems to be highly canonical, representing
 the limiting ratio of distance between bifurcation intervals for a
 fairly large class of one-dimensional maps.  In other words, all maps
 that fall in to this category will bifurcate at the same rate, giving
 us a glimpse of order in the realm of chaotic systems.

\textbf{{\Large \#5) 2}}\\ This number caused quite a bit of
controversy in discussions leading up to the construction of this list.
The question here is canonicality.  The first argument of ``It's the
only even prime'' is merely a re-wording of ``It's the only prime
divisible by 2,'' which could uniquely characterizes \emph{any} prime
(e.g. 5 is the only prime divisible by 5, etc.).  Of debatable
canonicality is the immensely prevalent notion of ``working in
binary.''  To a computer scientist, this may seem extremely canonical,
but to a mathematician, it may simply be an (not quite) arbitrary
choice of a finite field over which to work.

Yet 2 has some remarkable features even ignoring aspects relating to
its primality.  For instance, the somewhat canonical
field of real numbers $\mathbb{R}$ has index 2 in its algebraic
closure $\mathbb{C}$.  The factor $2\pi i$ is prevalent enough in
complex and Fourier analysis that I've heard people lament that $\pi$
should have been defined to be twice its current value.  It's also the \emph{only} prime number $p$ such that $x^p+y^p=z^p$ has any rational solutions.

Finally, if nothing else, it is certainly the first prime, and could
at least be included for being the first representative of such an
amazing class of numbers.

\textbf{{\Large \#4)  808017424794512875886459904961710757005754368$\times 10^9$}}\\

The above integer is the size of the monster group, the largest
of the sporadic groups.  This gives it a relatively high degree of
canonicality.  It's unclear (at least to me) why there should be
\emph{any} sporadic groups, or why, given that they exist, there
should only be finitely many.  Since there \emph{is}, however, there
must be something fairly special about the largest possible one.

Also contributing to this number's rank on this list is the remarkable
properties of the monster group itself, which has been realized
(actually, was constructed as) a group of rotations in
196,883-dimensional space, representing in some sense a \emph{limit}
to the amount of symmetry such a space can possess.

\textbf{{\Large \#3) Euler-Mascheroni Constant, $\gamma$}}\\ One of
the most amazing facts from elementary calculus is that the harmonic
series diverges, but that if you put an exponent on the denominators
even just a \emph{hair} above 1, the result is a convergent sequence.
A refined statement says that the partial sums of the harmonic series
grow like $\ln(n)$, and a further refinement says that the error of
this approximation approaches our constant:
\begin{align*}
\lim\limits_{n\ra\infty}1+\frac{1}{2}+\frac{1}{3}+\cdots+\frac{1}{n}-\ln(n)=\gamma.
\end{align*}
This seems to represent something fundamental about the harmonic series, and
thus of the integers themselves.  

Finally, perhaps due to importance inherited from the crucially
important harmonic series, the Euler-Mascheroni constant appears
magically all over mathematics.  
% For some idea of $\gamma$'s ability
% to pop up in unforeseen places, see the MathWorld entry on the
% Euler-Mascheroni constant.

\textbf{{\Large \#2) Khinchin's constant, $K\approx 2.685452...$}}\\ 
For a real number $x$, we define a geometric mean function 
\begin{align*}
f(x)=\lim\limits_{n\ra\infty}(a_1\cdots a_n)^{1/n},
\end{align*}
where the $a_i$ are the terms of the simple continued fraction
expansion of $x$. By nothing short of a miracle of mathematics, this
function of $x$ is almost everywhere (i.e. everywhere except for a set
of measure 0) \emph{independent of }$x$!!!  In other words, except for
a ``small'' number of exceptions, this function $f(x)$ always outputs
the same value, which is called Khinchin's constant and is denoted by
$K$.  It's hard to impress upon a casual reader just how astounding
this is, but consider the following: \emph{Any} infinite collection of
non-negative integers $a_0, a_1, \ldots$ forms a continued fraction,
and indeed each continued fraction gives an infinite collection of
that form.  That the partial geometric means of these sequences is
\emph{almost everywhere constant} tells us a great deal about the
distribution of sequences showing up as continued fraction sequences,
in turn revealing something very fundamental about the structure of
real numbers.

\textbf{{\Large \#1) 163}}\\ Well, we've come down to it, this
author's humble opinion of the coolest number in existence.  Though an
unlikely candidate, I hope to show you that 163 satisfies so many
eerily related properties as to earn this title.

I'll begin with something that most number theorists already know
about this number -- it is the largest value of $d$ such that the
number field $\Q(\sqrt{-d})$ has class number 1, meaning that its ring
of integers is a unique factorization domain.  The issue of
factorization in quadratic fields, and of number fields in general, is
one of the principal driving forces of algebraic number theory, and to
be able to pinpoint the end of perfect factorization in the quadratic
case like this seems at least arguably fundamental.

But even if you don't care about factorization in number fields, the
above fact has some amazing repercussions to more basic number theory.
The two following facts in particular jump out:
\begin{itemize}
\item $e^{\pi\sqrt{163}}$ is within $10^{-12}$ of an integer.
\item The polynomial $f(x)=x^2+x+41$ has the property that for integers
$1\leq x\leq 41$, $f(x)$ is prime.
\end{itemize}
Both of these are tied intimately (the former using deep properties of
the $j$-function, the latter using relatively simple arguments
concerning the splitting of primes in number fields) to the above
quadratic imaginary number field having class number 1.  Further,
since $\Q(\sqrt{-d})$ is the \emph{last} such field, the two listed
properties are in some sense the best possible.

Most striking to me, however, is the amazing frequency with which 163
shows up in a wide variety of class number problems.  In addition to
being the last value of $d$ such that $\Q(\sqrt{-d})$ has class number
1, it is the \emph{first} value of $p$ such that
$\Q(\zeta_p+\zeta_p^{-1})$ (the maximal real subfield of the $p$-th
cyclotomic field) has class number \emph{greater} than 1.  That 163
appears as the last instance of a quadratic field having unique
factorization, and the first instance of a real cyclotomic field
\emph{not} having unique factorization, seems too remarkable to be
coincidental.  This is (maybe) further substantiated by a couple of
other factoids
\begin{itemize}
\item Hasse asked for an example of a prime and an extension such that
the prime splits completely into divisors which \emph{do not} lie in a
cyclic subgroup of the class group. The first such example is any prime
less than 163 which splits completely in the cubic field generated by
the polynomial $x^3=11x^2+14x+1$.  This field has discriminant
$163^2$.  (See Shanks' \emph{The Simplest Cubic Fields}).
\item The maximal conductor if an imaginary abelian number field of
class number 1 corresponds to the field $\Q(\sqrt{-67},\sqrt{-163})$,
which has conductor $10921=67*163$.
\end{itemize}
It is unclear whether or not these additional arithmetical properties
reflect deeper properties of the $j$-function or other modular forms,
and remains a wide open field of study.

Originally posted on \PMlinkexternal{Cam's homepage}{http://math.arizona.edu/~mcleman}
%%%%%
%%%%%
\end{document}
