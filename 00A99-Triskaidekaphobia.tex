\documentclass[12pt]{article}
\usepackage{pmmeta}
\pmcanonicalname{Triskaidekaphobia}
\pmcreated{2013-03-22 17:23:48}
\pmmodified{2013-03-22 17:23:48}
\pmowner{PrimeFan}{13766}
\pmmodifier{PrimeFan}{13766}
\pmtitle{triskaidekaphobia}
\pmrecord{7}{39765}
\pmprivacy{1}
\pmauthor{PrimeFan}{13766}
\pmtype{Definition}
\pmcomment{trigger rebuild}
\pmclassification{msc}{00A99}

\endmetadata

% this is the default PlanetMath preamble.  as your knowledge
% of TeX increases, you will probably want to edit this, but
% it should be fine as is for beginners.

% almost certainly you want these
\usepackage{amssymb}
\usepackage{amsmath}
\usepackage{amsfonts}

% used for TeXing text within eps files
%\usepackage{psfrag}
% need this for including graphics (\includegraphics)
%\usepackage{graphicx}
% for neatly defining theorems and propositions
%\usepackage{amsthm}
% making logically defined graphics
%%%\usepackage{xypic}

% there are many more packages, add them here as you need them

% define commands here

\begin{document}
\PMlinkescapeword{irrational}
\PMlinkescapeword{join}
\PMlinkescapeword{flag}
\PMlinkescapeword{stars}
\PMlinkescapeword{floor}

\emph{Triskaidekaphobia} is the irrational fear of the number 13. This fear is recognized as a mental disorder by the Diagnostic and Statistical Manual of Mental Disorders IV (DSM-IV). Few mathematicians suffer from triskaidekaphobia; generally it's artists who are afflicted. For example, the composer Arnold Schoenberg and the pianist Glenn Gould. However, Benjamin Franklin was mildly afflicted: if there were 13 people at his dinner table, he would call for his secretary to join the party. He was apparently not bothered by the fact that the United States at the time consisted of 13 states and the flag had 13 stars and 13 stripes. The rationale for Franklin's affliction seems to have been that there were 13 people at the Last Supper of Jesus Christ. Architects for the most part don't have to worry about triskaidekaphobia when building new  buildings, but when working on existing buildings they have to remember to subtract one if the building's 13th floor was renumbered 14.

\begin{thebibliography}{1}
\bibitem{wc} W. Clarke, \PMlinkexternal{{\it Some Mathematical History: The History of Numbers}}{http://faculty.lasierra.edu/~wclarke/hist.pdf} (2003): 7
\end{thebibliography}
%%%%%
%%%%%
\end{document}
