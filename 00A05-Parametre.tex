\documentclass[12pt]{article}
\usepackage{pmmeta}
\pmcanonicalname{Parametre}
\pmcreated{2013-03-22 17:06:59}
\pmmodified{2013-03-22 17:06:59}
\pmowner{pahio}{2872}
\pmmodifier{pahio}{2872}
\pmtitle{parametre}
\pmrecord{17}{39417}
\pmprivacy{1}
\pmauthor{pahio}{2872}
\pmtype{Definition}
\pmcomment{trigger rebuild}
\pmclassification{msc}{00A05}
\pmsynonym{parameter}{Parametre}
\pmrelated{Indeterminate}
\pmrelated{DerivativeForParametricForm}
\pmrelated{Curve}
\pmrelated{PerimeterOfAstroid}
\pmrelated{CissoidOfDiocles}
\pmrelated{Variable}
\pmrelated{SurfaceNormal}
\pmdefines{auxiliary variable}
\pmdefines{parametric form}
\pmdefines{parametric presentation}
\pmdefines{parameter of parabola}

% this is the default PlanetMath preamble.  as your knowledge
% of TeX increases, you will probably want to edit this, but
% it should be fine as is for beginners.

% almost certainly you want these
\usepackage{amssymb}
\usepackage{amsmath}
\usepackage{amsfonts}

% used for TeXing text within eps files
%\usepackage{psfrag}
% need this for including graphics (\includegraphics)
%\usepackage{graphicx}
% for neatly defining theorems and propositions
 \usepackage{amsthm}
% making logically defined graphics
%%%\usepackage{xypic}

% there are many more packages, add them here as you need them

% define commands here

\theoremstyle{definition}
\newtheorem*{thmplain}{Theorem}

\begin{document}
\PMlinkescapeword{constant}
\emph{Parametre} means often a quantity which is considered as constant in a certain situation but which may take different values in other situations; so the parametre is a ``variable constant''.\; But in giving a curve or a surface in {\em parametric form}, the parametres work as proper variables which determine the values of the coordinates of the points; then we can describe the parametres as ``auxiliary variables''.

The parametric \PMlinkescapetext{presentation}
\begin{align*}
\begin{cases}
  x = a\cos{t}\\
  y = a\sin{t}
\end{cases}
\end{align*}
of the origin-centered circle \PMlinkescapetext{contains both above-mentioned sorts} of parametres:\; $a$ (the radius) is a variable constant which is held constant all the time when one considers one circle;\, $t$ is an auxiliary variable which has to get all real values (e.g. from the interval \, $[0,\,2\pi]$)\, for obtaining all points of the perimetre.

In the analytic geometry, one speaks of the \emph{parametre of parabola} (a.k.a. \emph{latus rectum}):\, it means the chord of the parabola which is perpendicular to the axis and goes through the focus; it is the quantity $2p$ in the standard equation \, $x^2 = 2py$\; of the parabola ($p$ is the distance of the focus and the directrix).
%%%%%
%%%%%
\end{document}
