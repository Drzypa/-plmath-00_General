\documentclass[12pt]{article}
\usepackage{pmmeta}
\pmcanonicalname{TableOfDivisionUpTo12}
\pmcreated{2013-03-22 16:36:16}
\pmmodified{2013-03-22 16:36:16}
\pmowner{PrimeFan}{13766}
\pmmodifier{PrimeFan}{13766}
\pmtitle{table of division up to 12}
\pmrecord{7}{38800}
\pmprivacy{1}
\pmauthor{PrimeFan}{13766}
\pmtype{Data Structure}
\pmcomment{trigger rebuild}
\pmclassification{msc}{00A06}
\pmclassification{msc}{12E99}
\pmclassification{msc}{00A05}

\endmetadata

% this is the default PlanetMath preamble.  as your knowledge
% of TeX increases, you will probably want to edit this, but
% it should be fine as is for beginners.

% almost certainly you want these
\usepackage{amssymb}
\usepackage{amsmath}
\usepackage{amsfonts}

% used for TeXing text within eps files
%\usepackage{psfrag}
% need this for including graphics (\includegraphics)
%\usepackage{graphicx}
% for neatly defining theorems and propositions
%\usepackage{amsthm}
% making logically defined graphics
%%%\usepackage{xypic}

% there are many more packages, add them here as you need them

% define commands here

\begin{document}
In this table of division, the column operand is first and the row operand is second. For the sake of compactness, overlines have been used over repeating decimals when the operands are coprime to each other and to 10.

\begin{tabular}{|c|l|l|l|l|l|l|l|l|l|l|l|l|}
$\div$ & 1 & 2 & 3 & 4 & 5 & 6 & 7 & 8 & 9 & 10 & 11 & 12 \\
1 & $1$ & $0.5$ & $0.\overline{3}$ & $0.25$ & $0.2$ & $0.1\overline{6}$ & $0.\overline{142857}$ & $0.125$ & $0.\overline{1}$ & $0.1$ & $0.\overline{09}$ & $0.08\overline{3}$ \\
2 & $2$ & $1$ & $0.\overline{6}$ & $0.5$ & $0.4$ & $0.\overline{3}$ & $0.\overline{285714}$ & $0.25$ & $0.\overline{2}$ & $0.2$ & $0.\overline{18}$ & $0.1\overline{6}$ \\
3 & $3$ & $1.5$ & $1$ & $0.75$ & $0.6$ & $0.5$ & $0.\overline{428571}$ & $0.375$ & $0.\overline{3}$ & $0.3$ & $0.\overline{27}$ & $0.25$ \\
4 & $4$ & $2$ & $1.\overline{3}$ & $1$ & $0.8$ & $0.\overline{6}$ & $0.\overline{571428}$ & $0.5$ & $0.\overline{4}$ & $0.4$ & $0.\overline{36}$ & $0.\overline{3}$ \\
5 & $5$ & $2.5$ & $1.\overline{6}$ & $1.25$ & $1$ & $0.8\overline{3}$ & $0.714285$ & $0.625$ & $0.5$ & $0.5$ & $0.45$ & $0.41\overline{6}$ \\
6 & $6$ & $3$ & $2$ & $1.5$ & $1.2$ & $1$ & $0.\overline{857142}$ & $0.75$ & $0.\overline{6}$ & $0.6$ & $0.\overline{54}$ & $0.5$ \\
7 & $7$ & $3.5$ & $2.\overline{3}$ & $1.75$ & $1.4$ & $1.1\overline{6}$ & $1$ & $0.875$ & $0.\overline{7}$ & $0.7$ & $0.\overline{63}$ & $0.58\overline{3}$ \\
8 & $8$ & $4$ & $2.\overline{6}$ & $2$ & $1.6$ & $1.\overline{3}$ & $1.\overline{142857}$ & $1$ & $0.\overline{8}$ & $0.8$ & $0.\overline{72}$ & $0.\overline{6}$ \\
9 & $9$ & $4.5$ & $3$ & $2.25$ & $1.8$ & $1.5$ & $1.\overline{285714}$ & $1.125$ & $1$ & $0.9$ & $0.\overline{81}$ & $0.75$ \\
10 & $10$ & $5$ & $3.\overline{3}$ & $2.5$ & $2$ & $1.\overline{6}$ & $1.\overline{428571}$ & $1.25$ & $1.\overline{1}$ & $1$ & $0.\overline{90}$ & $0.8\overline{3}$ \\
11 & $11$ & $5.5$ & $3.\overline{6}$ & $2.75$ & $2.2$ & $1.8\overline{3}$ & $1.\overline{571428}$ & $1.375$ & $1.\overline{2}$ & $1.1$ & $1$ & $0.91\overline{6}$ \\
12 & $12$ & $6$ & $4$ & $3$ & $2.4$ & $2$ & $1.\overline{714285}$ & $1.5$ & $1.\overline{3}$ & $1.2$ & $1.\overline{09}$ & $1$ \\
\end{tabular}

The longest northwest to southeast diagonal obviously contains ones.
%%%%%
%%%%%
\end{document}
