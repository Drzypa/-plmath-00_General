\documentclass[12pt]{article}
\usepackage{pmmeta}
\pmcanonicalname{HighSchoolMathematics}
\pmcreated{2013-03-22 15:16:09}
\pmmodified{2013-03-22 15:16:09}
\pmowner{matte}{1858}
\pmmodifier{matte}{1858}
\pmtitle{high school mathematics}
\pmrecord{93}{37056}
\pmprivacy{1}
\pmauthor{matte}{1858}
\pmtype{Topic}
\pmcomment{trigger rebuild}
\pmclassification{msc}{00A20}
\pmrelated{IndexOfEntriesOnCompassAndStraightedgeConstructions}

% this is the default PlanetMath preamble.  as your knowledge
% of TeX increases, you will probably want to edit this, but
% it should be fine as is for beginners.

% almost certainly you want these
\usepackage{amssymb}
\usepackage{amsmath}
\usepackage{amsfonts}
\usepackage{amsthm}

\usepackage{mathrsfs}

% used for TeXing text within eps files
%\usepackage{psfrag}
% need this for including graphics (\includegraphics)
%\usepackage{graphicx}
% for neatly defining theorems and propositions
%
% making logically defined graphics
%%%\usepackage{xypic}

% there are many more packages, add them here as you need them

% define commands here

\newcommand{\sR}[0]{\mathbb{R}}
\newcommand{\sC}[0]{\mathbb{C}}
\newcommand{\sN}[0]{\mathbb{N}}
\newcommand{\sZ}[0]{\mathbb{Z}}

 \usepackage{bbm}
 \newcommand{\Z}{\mathbbmss{Z}}
 \newcommand{\C}{\mathbbmss{C}}
 \newcommand{\F}{\mathbbmss{F}}
 \newcommand{\R}{\mathbbmss{R}}
 \newcommand{\Q}{\mathbbmss{Q}}



\newcommand*{\norm}[1]{\lVert #1 \rVert}
\newcommand*{\abs}[1]{| #1 |}



\newtheorem{thm}{Theorem}
\newtheorem{defn}{Definition}
\newtheorem{prop}{Proposition}
\newtheorem{lemma}{Lemma}
\newtheorem{cor}{Corollary}
\begin{document}
\PMlinkescapeword{index}

The aim of this meta entry is to index entries suitable for
high school students. 

\subsubsection*{Basics}
\begin{enumerate}
\item set, union, intersection
\item natural numbers, rational numbers, real numbers
\item associativity of multiplication
\item product of negative numbers
\item equation, inequality
\item proportion equation, proportionality of numbers
\item per cent
\item mathematical induction
\item proof by contradiction
\item converse
\item contrapositive
\end{enumerate}

\subsubsection*{Algebra}
\begin{enumerate}
\item opposite number, difference
\item inverse number, division
\item multiple, product 
\item entries on rational numbers
\item irrational numbers
\item factorization of integers
\item linear equation
\item square of sum
\item difference of squares
\item grouping method for factoring polynomials
\item factoring a sum or difference of two cubes
\item zero rule of product
\item \PMlinkname{conjugation}{ConjugationMnemonic}
\item even-even-odd rule
\item completing the square
\item square roots of rationals
\item quadratic formula
\item quadratic inequality
\item strange root
\item inequality with absolute values
\item absolute value inequalities
\item long division of polynomials
\end{enumerate}

\subsubsection*{Geometry}
\begin{enumerate}
\item basic geometric figures:

\begin{itemize}
\item points
\item lines
\item planes
\item line segments
\item rays
\item angles
\item triangles
\item parallelograms
\item rectangles
\item trapezoids
\item polygons
\item regular polygons
\item base and height of triangle
\item circles
\item parts of a ball
\item cylinder
\item solid cone
\end{itemize}

\item basic geometric properties:

\begin{itemize}
\item intersections
\item \PMlinkname{between}{Betweenness}
\item endpoints
\item \PMlinkname{midpoints}{Midpoint}
\item parallelism
\item perpendicularity
\item \PMlinkname{congruence}{Congruence}
\item similarity
\item similar triangles
\item tangent of circle
\end{itemize}

\item acute angles, convex angles, radian
\item complementary angles, supplementary angles, explementary angles
\item angle between two lines, angle of view
\item projection of point
\item locus
\item normal line, angle bisector, angle bisector as locus, center normal as locus
\item measurements (lengths, areas, and volumes) of basic geometric figures
\item compass and straightedge constructions:

\begin{itemize}
\item \PMlinkname{midpoint}{Midpoint}
\item perpendicular bisector
\item dropping the perpendicular from a point to a line
\item erecting the perpendicular to a line at a point
\item \PMlinkname{angle bisector}{CompassAndStraightedgeConstructionOfAngleBisector}
\item \PMlinkname{regular triangle}{CompassAndStraightedgeConstructionOfRegularTriangle}
\item \PMlinkname{duplicating an angle}{CompassAndStraightedgeConstructionOfDuplicatingAnAngle}
\item \PMlinkname{center of a circle}{CompassAndStraightedgeConstructionOfCenterOfGivenCircle}
\item construction of tangent
\item \PMlinkname{circle passing through three noncollinear points}{Circumcenter}
\item \PMlinkname{parallel line}{CompassAndStraightedgeConstructionOfParallelLine}
\item \PMlinkname{square}{CompassAndStraightedgeConstructionOfSquare}
\item \PMlinkname{$n$-section of line segment}{NSectionOfLineSegmentWithCompassAndStraightedge}
\item circle with given center and given radius
\item \PMlinkname{similar triangles}{CompassAndStraightedgeConstructionOfSimilarTriangles}
\item \PMlinkname{geometric mean}{CompassAndStraightedgeConstructionOfGeometricMean}
\item \PMlinkname{central proportional}{ConstructionOfCentralProportion}
\item \PMlinkname{inverse point with respect to a circle}{CompassAndStraightedgeConstructionOfInversePoint}
\item \PMlinkname{regular pentagon}{CompassAndStraightedgeConstructionOfRegularPentagon}
\item \PMlinkname{construction of regular $2n$-gon from regular $n$-gon}{ConstructionOfRegular2nGonFromRegularNGon}
\end{itemize}

\item trisection of angle
\item axiomatic proofs in geometry:

\begin{itemize}
\item angles of an isosceles triangle
\item determining from angles that a triangle is isosceles
\item isosceles triangle theorem
\item converse of isosceles triangle theorem
\item parallelogram theorems
\item regular polygon and circles
\item Pythagorean theorem and its various proofs:

\begin{itemize}
\item \PMlinkname{using four congruent triangles and a square}{ProofOfPythagoreasTheorem}
\item \PMlinkname{dropping altitude to form three similar triangles}{ProofOfPythagoreanTheorem}
\item \PMlinkname{Garfield's proof}{GarfieldsProofOfPythagoreanTheorem}
\item \PMlinkname{two dissections of a square with side $a+b$}{ProofOfPythagoreanTheorem2}
\end{itemize}

\item construct the center of a given circle
\item Thales' theorem and its \PMlinkname{proof}{ProofOfThalesTheorem}
\item opposing angles in a cyclic quadrilateral are supplementary
\item mid-segment theorem
\end{itemize}
\end{enumerate}

\subsubsection*{Analytic geometry}
\begin{enumerate}
\item analytic geometry
\item Cartesian coordinates
\item coordinates of midpoint
\item slope
\item tangent line
\item condition of orthogonality
\item \PMlinkname{angle between two lines}{AngleBetweenTwoLines}
\item conics (ellipse, hyperbola, parabola)
\item polar coordinates
\end{enumerate}

\subsubsection*{Vectors and Matrices}
\begin{enumerate}
\item sum of vectors (i.e. parallelogram principle), difference of vectors
\item Euclidean vectors
\item mutual positions of vectors
\item scalar product
\item matrices, addition and multiplication of matrices
\end{enumerate}

\subsubsection*{Trigonometry}
\begin{enumerate}
\item right triangle
\item regular triangle
\item isosceles triangle
\item altitudes
\item bisectors
\item ASA, SSS, SAS, SSA (triangle solving)
\item exact trigonometry tables
\item sohcahtoa
\item determining signs of trigonometric functions
\item addition formulas for sine and cosine
\item addition formula for tangent
\item goniometric formulae
\item trigonometric equation
\end{enumerate}

\subsubsection*{Functions}
\begin{enumerate}
\item definitions and operations of functions
\item argument
\item polynomial functions (including linear functions)
\item rational functions
\item functions involving \PMlinkescapetext{radicals}
\item exponential functions
\item Briggsian logarithms
\item trigonometric functions
\item limit of real number sequence
\item geometric sequence
\item sequences and series

\end{enumerate}

\subsubsection*{Differential calculus}
\begin{enumerate}
\item concept of a limit
\item limit rules of functions, improper limit
\item continuous
\item \PMlinkname{limit of sine divided by angle at 0}{LimitOfDisplaystyleFracsinXxAsXApproaches0}
\item intermediate value theorem
\item derivative
\item derivatives of sine and cosine
\item derivative of inverse function
\item related rates
\item minimum and maximum of functions (extrema)
\item least and greatest value of function
\item mean value theorem
\item Rolle's theorem
\end{enumerate}

\subsubsection*{Integral calculus}
\begin{enumerate}
\item Riemann sum
\item integral
\item left hand rule
\item right hand rule
\item midpoint rule
\item \PMlinkname{trapezoid rule}{CompositeTrapezoidalRule}
\item fundamental theorem of calculus
\item integration techniques
\end{enumerate}

\subsubsection*{Complex numbers}
\begin{enumerate}
\item complex numbers
\item complex function
\end{enumerate}

\subsubsection*{Applications (word problems)}
\begin{enumerate}
\item graphing of equations and inequalities
\item counting and basic probability
\item length, area, volume
\item distance, rate, speed, velocity
\item money, \PMlinkescapetext{simple} interest, compound interest
\end{enumerate}
%%%%%
%%%%%
\end{document}
