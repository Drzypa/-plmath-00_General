\documentclass[12pt]{article}
\usepackage{pmmeta}
\pmcanonicalname{ArithmeticProgression}
\pmcreated{2013-03-22 13:39:00}
\pmmodified{2013-03-22 13:39:00}
\pmowner{bbukh}{348}
\pmmodifier{bbukh}{348}
\pmtitle{arithmetic progression}
\pmrecord{10}{34302}
\pmprivacy{1}
\pmauthor{bbukh}{348}
\pmtype{Definition}
\pmcomment{trigger rebuild}
\pmclassification{msc}{00A05}
\pmclassification{msc}{11B25}
\pmrelated{MulidimensionalArithmeticProgression}
\pmrelated{SumOfKthPowersOfTheFirstNPositiveIntegers}

\usepackage{amssymb}
\usepackage{amsmath}
\usepackage{amsfonts}
\begin{document}
\PMlinkescapeword{columns}
Arithmetic progression of length $n$, initial term $a_1$ and common difference $d$ is the sequence $a_1, a_1+d,a_1+2d,\dotsc,a_1+(n-1)d$.

%%%This code is not displayed by PlanetMath... seems to be a bug in LaTeX2HTML
%%%  Added later: parbox fixes it since it forces LaTeX2HTML to render the whole
%%%               thing as an image
The sum of terms of an arithmetic progression can be computed using Gauss's trick:

\parbox{\linewidth}{\begin{align*}
S&=\makebox[7em]{$(a_1+0)$}+\makebox[7em]{$(a_1+d)$}+\dotsb+\makebox[7em]{$(a_1+(n-2)d)$}
+\makebox[7em]{$(a_1+(n-1)d)$}\\
+\underline{S\vphantom{\makebox[7em]{$(a_1+(n-1)d)$}}}&\underline{{}=\makebox[7em]{$(a_1+(n-1)d)$}+
\makebox[7em]{$(a_1+(n-2)d)$}+\dotsb+\makebox[7em]{$(a_1+d)$}
+\makebox[7em]{$(a_1+0)$}}\\
2S&=\makebox[7em]{$(2a_1+(n-1)d)$}+\makebox[7em]{$(2a_1+(n-1)d)$}+\dotsb+\makebox[7em]{$(2a_1+(n-1)d)$}+
\makebox[7em]{$(2a_1+(n-1)d)$}.
\end{align*}}

We just add the sum with itself written backwards, and the sum of each of the columns equals to $(2a_1+(n-1)d)$. The sum is then
\begin{equation*}
S=\frac{(2a_1+(n-1)d)n}{2}.
\end{equation*}
%%%%%
%%%%%
\end{document}
