\documentclass[12pt]{article}
\usepackage{pmmeta}
\pmcanonicalname{Dimension}
\pmcreated{2013-03-22 14:02:50}
\pmmodified{2013-03-22 14:02:50}
\pmowner{bbukh}{348}
\pmmodifier{bbukh}{348}
\pmtitle{dimension}
\pmrecord{10}{35398}
\pmprivacy{1}
\pmauthor{bbukh}{348}
\pmtype{Topic}
\pmcomment{trigger rebuild}
\pmclassification{msc}{00-01}
\pmclassification{msc}{15A03}
\pmclassification{msc}{54F45}
\pmrelated{Dimension}
\pmrelated{Dimension2}
\pmrelated{DimensionKrull}
\pmrelated{HausdorffDimension}

\usepackage{amssymb}
\usepackage{amsmath}
\usepackage{amsfonts}

% used for TeXing text within eps files
%\usepackage{psfrag}
% need this for including graphics (\includegraphics)
%\usepackage{graphicx}
% for neatly defining theorems and propositions
%\usepackage{amsthm}
% making logically defined graphics
%%%\usepackage{xypic}

\makeatletter
\@ifundefined{bibname}{}{\renewcommand{\bibname}{References}}
\makeatother
\begin{document}
% This entry is supposed to be a popular explanation of
% different ways to define dimension in mathematics
% and why these definitions should make sense.
% This is the reason why this entry does not list any
% terms in ``defines'' field. Rigorous definitions should
% be supplied by other entries
%
% Suggestions for improvement are welcome.
\PMlinkescapeword{small}\PMlinkescapeword{natural}
The word \emph{dimension} in mathematics has many definitions, but
all of them are trying to quantify our intuition that, for
example, a sheet of paper has somehow one less dimension than a
stack of papers.

One common way to define dimension is through some notion of a
number of independent quantities needed to describe an element
of an object. For example, it is natural to say that the sheet of
paper is two-dimensional because one needs two real numbers to
specify a position on the sheet, whereas the stack of papers is
three-dimension because a position in a stack is specified by a sheet
and a position on the sheet. Following this notion, in linear
algebra the \PMlinkname{dimension of a vector space}{Dimension2} 
is defined as the minimal number of vectors such that every other
vector in the vector space is representable as a sum of these.
Similarly, the word \emph{rank} denotes various dimension-like
invariants that appear throughout the algebra.

However, if we try to generalize this notion to the mathematical
objects that do not possess an algebraic structure, then we run
into a difficulty. From the point of view of set theory there are
\PMlinkname{as many}{Cardinality} real numbers as pairs of real
numbers since there is a bijection from real numbers to pairs of
real numbers. To distinguish a plane from a cube one needs to
impose restrictions on the kind of mapping. Surprisingly, it turns
out that the continuity is not enough as was pointed out by Peano.
There are continuous functions that map a square onto a cube. So,
in topology one uses another intuitive notion that in a
high-dimensional space there are more directions than in a
low-dimensional. Hence, the (Lebesgue covering) dimension of a
topological space is defined as the smallest number $d$ such that
every covering of the space by open sets can be refined so that no
point is contained in more than $d+1$ sets. For example, no matter
how one covers a sheet of paper by sufficiently small other sheets
of paper such that two sheets can overlap each other, but
cannot merely touch, one will always find a point that is covered
by $2+1=3$ sheets.

Another definition of dimension rests on the idea that
higher-dimensional objects are in some sense larger than the
lower-dimensional ones. For example, to cover a cube with a side
length $2$ one needs at least $2^3=8$ cubes with a side length
$1$, but a square with a side length $2$ can be covered by only
$2^2=4$ unit squares. Let $N(\epsilon)$ be the minimal number of
open balls in any covering of a bounded set $S$ by balls of radius
$\epsilon$. The \PMlinkname{Besicovitch-Hausdorff dimension}{HausdorffDimension} of $S$ is defined
as $-\lim_{\epsilon\to 0} \log_\epsilon N(\epsilon)$. The
Besicovitch-Hausdorff dimension is not always defined, and when
defined it might be non-integral.
%%%%%
%%%%%
\end{document}
