\documentclass[12pt]{article}
\usepackage{pmmeta}
\pmcanonicalname{Arithmetic}
\pmcreated{2013-03-22 18:16:18}
\pmmodified{2013-03-22 18:16:18}
\pmowner{PrimeFan}{13766}
\pmmodifier{PrimeFan}{13766}
\pmtitle{arithmetic}
\pmrecord{29}{40875}
\pmprivacy{1}
\pmauthor{PrimeFan}{13766}
\pmtype{Definition}
\pmcomment{trigger rebuild}
\pmclassification{msc}{00A05}
\pmsynonym{arithmetics}{Arithmetic}

% this is the default PlanetMath preamble.  as your knowledge
% of TeX increases, you will probably want to edit this, but
% it should be fine as is for beginners.

% almost certainly you want these
\usepackage{amssymb}
\usepackage{amsmath}
\usepackage{amsfonts}

% used for TeXing text within eps files
%\usepackage{psfrag}
% need this for including graphics (\includegraphics)
%\usepackage{graphicx}
% for neatly defining theorems and propositions
%\usepackage{amsthm}
% making logically defined graphics
%%%\usepackage{xypic}

% there are many more packages, add them here as you need them

% define commands here

\begin{document}
{\em Arithmetic} is " the science of numbers and operations on sets of numbers. Arithmetic is understood to include problems on the origin and development of the concept of a number, methods and means of calculation, the study of operations on numbers of different kinds, as well as analysis of the axiomatic structure of number sets and the properties of numbers." The four basic operations are addition, subtraction, multiplication and division; in the absence of parentheses these are performed according to the rules of operator precedence. From multiplication follows exponentiation.

Most civilizations usually develop arithmetic before any other branches of mathematics, and in modern times children are usually taught arithmetic first (though in some curricula they might be taught set theory first).

Numeral systems of ancient civilizations often took letter symbols for numbers. Boyer called these letter symbols one-to-one ciphered numerals. The ancient Greeks, for example, used letters in Ionian and Dorian alphabets to represent numbers, as did the Romans in one alphabet. These systems were additive in one sense (that is, the value of the overall ``word'' was merely the sum of the values of the individual symbols). To convert rational numbers into a ciphered numeration system, Fibonacci implemented seven rules, one being a LCM multiple. For example, four of Fibonacci's rules were used by earlier cultures. One was an Egyptian scribe. In 1650 BCE Ahmes converted
$\frac{n}{p}$ by using large multiples. About 200 years earlier, a student converted $\frac{n}{pq}$ by using small LCM multiples.

It was not just the invention of zero as a symbol for that integers between 1 and -1 allowed Hindu-Arabic numerals to significant expand man's ability to perform arithmetic operations. Significantly larger numbers came into use after zero became a placeholder in the new additive-exponential system. An algorithm written in Hindu-Arabic numbers before 1600 AD was also added. Modern decimals apply few lessons learned during the 3,200 year life of Egyptian fractions. One exception is aliquot parts, used in the fundamental theorem of arithmetic, and higher arithmetic.

The next great expansion of arithmetic power occurred with the invention of calculators and computers in the 20th Century, with both kinds of devices performing arithmetic in a binary algorithm. The algorithm, stated in a cursive form, had been used prior to 2,000 BCE.

As number theory generalizes the principles of arithmetic to all numbers or all numbers of a given kind, it is sometimes called the ``higher arithmetic.''

\subsection{External links}
\begin{itemize}
\item \PMlinkexternal{Fibonacci}{http://liberabaci.blogspot.com}
\item \PMlinkexternal{n/p}{http://rmprectotable.blogspot.com}
\item \PMlinkexternal{n/pq}{http://emlr.blogspot.com}
\end{itemize}
%%%%%
%%%%%
\end{document}
