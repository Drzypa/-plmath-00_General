\documentclass[12pt]{article}
\usepackage{pmmeta}
\pmcanonicalname{Arrow}
\pmcreated{2013-03-22 12:55:46}
\pmmodified{2013-03-22 12:55:46}
\pmowner{PrimeFan}{13766}
\pmmodifier{PrimeFan}{13766}
\pmtitle{arrow}
\pmrecord{14}{33284}
\pmprivacy{1}
\pmauthor{PrimeFan}{13766}
\pmtype{Definition}
\pmcomment{trigger rebuild}
\pmclassification{msc}{00A99}
%\pmkeywords{arrow}
%\pmkeywords{vector}
%\pmkeywords{mapping}
%\pmkeywords{function}
%\pmkeywords{morphism}
%\pmkeywords{functor}
\pmrelated{Morphism}
\pmrelated{Functor}
\pmrelated{NaturalTransformation}
\pmrelated{Mapping}
\pmrelated{Vector}
\pmrelated{Category}

\endmetadata

% this is the default PlanetMath preamble.  as your knowledge
% of TeX increases, you will probably want to edit this, but
% it should be fine as is for beginners.

% almost certainly you want these
\usepackage{amssymb}
\usepackage{amsmath}
\usepackage{amsfonts}

% used for TeXing text within eps files
%\usepackage{psfrag}
% need this for including graphics (\includegraphics)
%\usepackage{graphicx}
% for neatly defining theorems and propositions
%\usepackage{amsthm}
% making logically defined graphics
%%%\usepackage{xypic}

% there are many more packages, add them here as you need them

% define commands here
%\PMlinkescapeword{theory}
\begin{document}
An {\em arrow} is a line, usually straight, capped at one end, or both ends, with an arrowhead which points in a specific direction. In a flowchart, arrows show the order in which instructions are to be performed (including where branching may occur) or they show the flow of cause and effect. In certain arrangements of numbers, such as the Collatz tree, arrows show which numbers are the input values, with the output value being pointed at by the arrowhead.

In writing about vectors, arrows can be written over the symbols for the origin and the endpoint, e.g., $\overrightarrow{AB}$ and $\overleftarrow{CD}$. \TeX{} provides commands for these.

An arrow, regarded as a morphism, functor or natural transformation also plays central roles
in Category Theory and Categorical Dynamics. 
%%%%%
%%%%%
\end{document}
