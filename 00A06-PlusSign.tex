\documentclass[12pt]{article}
\usepackage{pmmeta}
\pmcanonicalname{PlusSign}
\pmcreated{2013-03-22 17:35:43}
\pmmodified{2013-03-22 17:35:43}
\pmowner{pahio}{2872}
\pmmodifier{pahio}{2872}
\pmtitle{plus sign}
\pmrecord{5}{40009}
\pmprivacy{1}
\pmauthor{pahio}{2872}
\pmtype{Definition}
\pmcomment{trigger rebuild}
\pmclassification{msc}{00A06}
\pmclassification{msc}{00A05}
\pmsynonym{plus}{PlusSign}
\pmrelated{Sum}
\pmrelated{SumOfSeries}
\pmrelated{SignumFunction}
\pmrelated{OppositeNumber}
\pmrelated{ProductOfNegativeNumbers}

% this is the default PlanetMath preamble.  as your knowledge
% of TeX increases, you will probably want to edit this, but
% it should be fine as is for beginners.

% almost certainly you want these
\usepackage{amssymb}
\usepackage{amsmath}
\usepackage{amsfonts}

% used for TeXing text within eps files
%\usepackage{psfrag}
% need this for including graphics (\includegraphics)
%\usepackage{graphicx}
% for neatly defining theorems and propositions
 \usepackage{amsthm}
% making logically defined graphics
%%%\usepackage{xypic}

% there are many more packages, add them here as you need them

% define commands here

\theoremstyle{definition}
\newtheorem*{thmplain}{Theorem}

\begin{document}
There are two main uses of the {\em plus sign} ``$+$'' (which is a simplified form of ``\&'') in the mathematics and the applying sciences:
\begin{itemize}
\item The original use is as the sign for the binary operation {\em addition} of numbers and other elements of rings and algebras, vectors, etc.:
$$a\!+\!b\; := \mbox{\, the sum of\, } a \mbox{\, and\, }b$$
\item There is also a special use for the unary operation {\em identity mapping} concerning numbers and other ring elements:
$$+a\; :=\; a \mbox{\; (for all  } a)$$
\end{itemize}
%%%%%
%%%%%
\end{document}
