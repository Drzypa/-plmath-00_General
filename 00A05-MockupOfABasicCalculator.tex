\documentclass[12pt]{article}
\usepackage{pmmeta}
\pmcanonicalname{MockupOfABasicCalculator}
\pmcreated{2013-03-22 16:53:48}
\pmmodified{2013-03-22 16:53:48}
\pmowner{PrimeFan}{13766}
\pmmodifier{PrimeFan}{13766}
\pmtitle{mock-up of a basic calculator}
\pmrecord{4}{39153}
\pmprivacy{1}
\pmauthor{PrimeFan}{13766}
\pmtype{Example}
\pmcomment{trigger rebuild}
\pmclassification{msc}{00A05}
\pmclassification{msc}{01A07}

% this is the default PlanetMath preamble.  as your knowledge
% of TeX increases, you will probably want to edit this, but
% it should be fine as is for beginners.

% almost certainly you want these
\usepackage{amssymb}
\usepackage{amsmath}
\usepackage{amsfonts}

% used for TeXing text within eps files
%\usepackage{psfrag}
% need this for including graphics (\includegraphics)
%\usepackage{graphicx}
% for neatly defining theorems and propositions
%\usepackage{amsthm}
% making logically defined graphics
%%%\usepackage{xypic}

% there are many more packages, add them here as you need them

% define commands here

\begin{document}
This mock-up of a basic calculator is realistic in that it has almost every function one can expect on a typical basic calculator. The layout will of course be different on an actual calculator. A calculator shaped like, say, a credit card will have more columns of buttons than rows.

\begin{tabular}{|c|c|c|c|c|}
MC & & OFF & C & ON \\
MR &  &  &  & $\div$ \\
M+ & 7 & 8 & 9 & $\times$ \\
STO & 4 & 5 & 6 & $-$ \\
$\sqrt{x}$ & 1 & 2 & 3 & + \\
\% & 0 & . & $\pm$ & = \\
\end{tabular}

A basic calculator can be counted on to have buttons for the basic arithmetic operations (addition, subtraction, multiplication and division). A button for square root is sometimes provided, its operation usually being ``postfix''. The percentage key, when provided, does not always behave in a standard way. For example, on some calculators, a 15\% tip on a meal costing $\$42.54$ might be computed as $[4] [2] [.] [5] [4] [+] [1] [5] [\%] [=]$.
%%%%%
%%%%%
\end{document}
