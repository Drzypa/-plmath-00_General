\documentclass[12pt]{article}
\usepackage{pmmeta}
\pmcanonicalname{Interest}
\pmcreated{2013-03-22 16:39:51}
\pmmodified{2013-03-22 16:39:51}
\pmowner{CWoo}{3771}
\pmmodifier{CWoo}{3771}
\pmtitle{interest}
\pmrecord{6}{38870}
\pmprivacy{1}
\pmauthor{CWoo}{3771}
\pmtype{Definition}
\pmcomment{trigger rebuild}
\pmclassification{msc}{00A69}
\pmclassification{msc}{00A06}
\pmclassification{msc}{91B28}

\endmetadata

\usepackage{amssymb,amscd}
\usepackage{amsmath}
\usepackage{amsfonts}

% used for TeXing text within eps files
%\usepackage{psfrag}
% need this for including graphics (\includegraphics)
%\usepackage{graphicx}
% for neatly defining theorems and propositions
%\usepackage{amsthm}
% making logically defined graphics
%%\usepackage{xypic}
\usepackage{pst-plot}
\usepackage{psfrag}

% define commands here

\begin{document}
This entry mostly concerns the mathematical aspects of interest.  Also, we assume that the value of a unit amount of money does not change with time (so there is no consideration regarding inflation, etc...)

Let's first look at two common examples in real life:
\begin{enumerate}
\item An individual deposits a certain amount of money into an entity (a bank, for example) for a period of time, interest is accrued at the end of the time period payable to the individual by the entity.
\item One borrows a certain amount of money from an entity (a loan company, for example) for a period of time, interest is accrued at the end of the time period payable to the entity by the borrower.
\end{enumerate}

In both examples, two parties, or \emph{entities} are involved.  Let's call one the lender $L$ and the other the borrower $B$.  Between $L$ and $B$, there is an initial \emph{transaction} where a certain amount of money $M$, called the \emph{principal}, is transferred from one to another.  $M$ can be seen as a function of $L$ and $B$: $M=M(L,B)$ and has the property that $M(L,B)+M(B,L)=0$.  This property basically says that $B$ borrowing $M$ from $L$ is equivalent to $L$ borrowing $-M$ from $B$.  Let us agree that $M(L,B)\ge 0$.

At a later time $t$, an \emph{interest} is accrued.  Put it another way, $M$ has ``grown'' to $M(t)$.  Set the initial transaction at time $0$, then $M=M(0)$.  From this point of view, \emph{interest} $i$, is the ``additional'' amount of money earned, or accrued, between $0$ and $t$, or $i=M(t)-M(0)$.  If we want to emphasize $L$ and $B$ as the additional variables, then the definition, fully stated, becomes
$$i(L,B,t):=M(L,B,t)-M(L,B,0).$$
We can calculate interest between any two time periods.  For example, if we know $i(L,B,t)$ and $i(L,B,s)$, then the interest earned from $t$ to $s$ is simply the difference of the two: $$i(L,B,t,s):=i(L,B,s)-i(L,B,t)=M(L,B,s)-M(L,B,t).$$
Keep in mind is that in this definition, only one transaction has taken place: the initial one at $0$ (we assume that at time $t$, an interest has been merely been accrued but no actual transfer of money is taking place).  Interests resulting from additional transactions that happened between $0$ and $t$ need to be evaluated separately.

The last paragraph merely says that $i$ is additive with respect to \emph{transactions}: if $L_1$ loans $M_1$ to $B_1$ at $t_1$, and $L_2$ loans $M_2$ to $B_2$ at $t_2$, then the \emph{total} interest earned by the lenders from the borrowers at time $t$ is $$i(L,B,t)= i_1(L_1,B_1,t)+ i_2(L_2,B_2,t),$$
where $L$ is $L_1$ and $L_2$ considered as a single lending entity, and $B$ is $B_1$ and $B_2$ considered as a single borrowing entity.

\textbf{Remarks}.
\begin{itemize}
\item
Note that $t$ does not always have to be positive.  It is possible that we are interested in finding what $i$ is at time $t\le 0$.  For example, one may want to know the amount of interest lost had he saved his money in a bank and earned interest 30 years earlier, well before his current retirement age.
\item
Another point worth stressing is that the defintion only works if we stick to the same units.  If $M$ is expressed as U.S. Dollars, then that needs to be the unit of choice for money throughout.  Any other denominations or currencies need to be converted.  The same goes with time $t$.
\end{itemize}

\begin{thebibliography}{8}
\bibitem{sk} S. G. Kellison, {\em Theory of Interest}, McGraw-Hill/Irwin, 2nd Edition, (1991).
\end{thebibliography}
%%%%%
%%%%%
\end{document}
