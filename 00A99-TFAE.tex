\documentclass[12pt]{article}
\usepackage{pmmeta}
\pmcanonicalname{TFAE}
\pmcreated{2013-03-22 12:44:40}
\pmmodified{2013-03-22 12:44:40}
\pmowner{ariels}{338}
\pmmodifier{ariels}{338}
\pmtitle{TFAE}
\pmrecord{4}{33048}
\pmprivacy{1}
\pmauthor{ariels}{338}
\pmtype{Definition}
\pmcomment{trigger rebuild}
\pmclassification{msc}{00A99}
\pmsynonym{the following are equivalent}{TFAE}

% this is the default PlanetMath preamble.  as your knowledge
% of TeX increases, you will probably want to edit this, but
% it should be fine as is for beginners.

% almost certainly you want these
\usepackage{amssymb}
\usepackage{amsmath}
\usepackage{amsfonts}

% used for TeXing text within eps files
%\usepackage{psfrag}
% need this for including graphics (\includegraphics)
%\usepackage{graphicx}
% for neatly defining theorems and propositions
%\usepackage{amsthm}
% making logically defined graphics
%%%\usepackage{xypic}

% there are many more packages, add them here as you need them

% define commands here

\newcommand{\Prob}[2]{\mathbb{P}_{#1}\left\{#2\right\}}
\newcommand{\norm}[1]{\left\|#1\right\|}

% Some sets
\newcommand{\Nats}{\mathbb{N}}
\newcommand{\Ints}{\mathbb{Z}}
\newcommand{\Reals}{\mathbb{R}}
\newcommand{\Complex}{\mathbb{C}}
\begin{document}
The abbreviation ``TFAE'' is shorthand for ``\textbf{t}he \textbf{f}ollowing \textbf{a}re \textbf{e}quivalent''.  It is used before a set of equivalent conditions (each implies all the others).

In a definition, when one of the conditions is somehow ``better'' (simpler, shorter, ...), it makes sense to phrase the definition with that condition, and mention that the others are equivalent.  ``TFAE'' is typically used when none of the conditions can take priority over the others.  Actually proving the claimed equivalence must, of course, be done separately.
%%%%%
%%%%%
\end{document}
