\documentclass[12pt]{article}
\usepackage{pmmeta}
\pmcanonicalname{DecimalPlace}
\pmcreated{2013-03-22 17:27:24}
\pmmodified{2013-03-22 17:27:24}
\pmowner{CWoo}{3771}
\pmmodifier{CWoo}{3771}
\pmtitle{decimal place}
\pmrecord{12}{39839}
\pmprivacy{1}
\pmauthor{CWoo}{3771}
\pmtype{Definition}
\pmcomment{trigger rebuild}
\pmclassification{msc}{00A05}
\pmrelated{MetricSystem}
\pmdefines{decimal place value}
\pmdefines{hundreds}
\pmdefines{thousands}
\pmdefines{tenths}
\pmdefines{hundredths}
\pmdefines{thousandths}
\pmdefines{ten thousandths}

\endmetadata

\usepackage{amssymb,amscd}
\usepackage{amsmath}
\usepackage{amsfonts}
\usepackage{mathrsfs}
\usepackage{tabls}
% used for TeXing text within eps files
%\usepackage{psfrag}
% need this for including graphics (\includegraphics)
%\usepackage{graphicx}
% for neatly defining theorems and propositions
\usepackage{amsthm}
% making logically defined graphics
%%\usepackage{xypic}
\usepackage{pst-plot}
\usepackage{psfrag}

% define commands here
\newtheorem{prop}{Proposition}
\newtheorem{thm}{Theorem}
\newtheorem{ex}{Example}
\newcommand{\real}{\mathbb{R}}
\newcommand{\pdiff}[2]{\frac{\partial #1}{\partial #2}}
\newcommand{\mpdiff}[3]{\frac{\partial^#1 #2}{\partial #3^#1}}
\begin{document}
\PMlinkescapeword{mean}
\PMlinkescapeword{right}

A \emph{decimal place} of a number (a real number) $r$ is the position of a digit in its decimal expansion relative to the decimal point.  Let us write $r$ as a decimal number:
\begin{center}
$\xymatrix @C=0pt @H=10pt {A_nA_{n-1}\ldots A_1 & \bullet & D_1D_2\ldots D_m \ldots \\
& \save *\txt{the decimal point} \restore \ar[u] & }$
\end{center}
where each of the $A_i$ and $D_j$ is a digit in the decimal system (one of $0,1,2,3,4,5,6,7,8,9$).  In this decimal expansion,
\begin{itemize}
\item the $i$-th decimal place to the left of the decimal point is the position of $A_i$, and 
\item the $j$-th decimal place to the right of the decimal place is the position of $D_j$.  
\end{itemize}
The digits $A_i$ and $D_j$ are called the \emph{decimal place values}.  $A_i$ is the decimal place value corresponding to the $i$-th decimal place to the left of the decimal point, while $D_j$ is the decimal place value corresponding to the $j$-th decimal place to the right of the decimal point.

For decimal places closer to the decimal point, specific names are used.  Below are some of the most commonly used names:

\begin{center}
\begin{tabular}{|c||r|r|r|}
\hline name & position from & direction from & example \\
 & the decimal point & the decimal point & (position of 7) \\
\hline \hline ones & 1st & left & $7.2$ \\
\hline tens & 2nd & left & $71(=71.0)$ \\
\hline hundreds & 3rd & left & $735(=735.0)$ \\
\hline thousands & 4th & left & $7126(=7126.0)$ \\
\hline tenths & 1st & right & $0.7$ \\
\hline hundredths & 2nd & right & $12.47$ \\
\hline thousandths & 3rd & right & $9.837$ \\
\hline ten thousandths & 4th & right & $6.0037$ \\
\hline
\end{tabular}
\end{center}

\textbf{Remark}.  Instead of saying the $n$-th decimal place to the left or right of the decimal point, we simply say the $n$-th decimal place to mean the $n$-th decimal place to the \emph{right} of the decimal point, and say $(-n)$-th decimal place to mean the $n$-th decimal place to the \emph{left} of the decimal point.  

For example, in $1632.9758$, the digit $5$ is located on the third decimal place, while $6$ is located on the negative third decimal place.
%%%%%
%%%%%
\end{document}
