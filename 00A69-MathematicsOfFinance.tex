\documentclass[12pt]{article}
\usepackage{pmmeta}
\pmcanonicalname{MathematicsOfFinance}
\pmcreated{2013-03-22 16:40:17}
\pmmodified{2013-03-22 16:40:17}
\pmowner{rspuzio}{6075}
\pmmodifier{rspuzio}{6075}
\pmtitle{mathematics of finance}
\pmrecord{7}{38878}
\pmprivacy{1}
\pmauthor{rspuzio}{6075}
\pmtype{Topic}
\pmcomment{trigger rebuild}
\pmclassification{msc}{00A69}
\pmclassification{msc}{91B28}
\pmclassification{msc}{00A06}

\endmetadata

% this is the default PlanetMath preamble.  as your knowledge
% of TeX increases, you will probably want to edit this, but
% it should be fine as is for beginners.

% almost certainly you want these
\usepackage{amssymb}
\usepackage{amsmath}
\usepackage{amsfonts}

% used for TeXing text within eps files
%\usepackage{psfrag}
% need this for including graphics (\includegraphics)
%\usepackage{graphicx}
% for neatly defining theorems and propositions
%\usepackage{amsthm}
% making logically defined graphics
%%%\usepackage{xypic}

% there are many more packages, add them here as you need them

% define commands here

\begin{document}
From ancient times, a socially significant application of mathematics 
has been in accounting.  This entry points to places on
PlanetMath which deal with this topic.

\begin{enumerate}
\item interest
\item simple interest
\item compound interest, continuously compounded
\item interest rate
\item effective interest rate
\item instantaneous effective interest rate
\item present value, net present value
\item coupon rate
\item rate of return
\item yield rate, yield curve
\item force of interest
\item annuity
\item perpetuity
\item call option, put option, 
\item American option, European option
\item arbitrage, the arbitrage theorem
\item risk neutral
\item Black-Scholes formula
\item capital asset pricing model (CAPM)
\end{enumerate}

%%%%%
%%%%%
\end{document}
