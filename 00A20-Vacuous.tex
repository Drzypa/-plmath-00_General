\documentclass[12pt]{article}
\usepackage{pmmeta}
\pmcanonicalname{Vacuous}
\pmcreated{2013-03-22 14:42:27}
\pmmodified{2013-03-22 14:42:27}
\pmowner{matte}{1858}
\pmmodifier{matte}{1858}
\pmtitle{vacuous}
\pmrecord{9}{36326}
\pmprivacy{1}
\pmauthor{matte}{1858}
\pmtype{Definition}
\pmcomment{trigger rebuild}
\pmclassification{msc}{00A20}
\pmsynonym{vacuously}{Vacuous}
\pmsynonym{vacuously true}{Vacuous}
\pmsynonym{vacuous truth}{Vacuous}

\endmetadata

% this is the default PlanetMath preamble.  as your knowledge
% of TeX increases, you will probably want to edit this, but
% it should be fine as is for beginners.

% almost certainly you want these
\usepackage{amssymb}
\usepackage{amsmath}
\usepackage{amsfonts}
\usepackage{amsthm}

\usepackage{mathrsfs}

% used for TeXing text within eps files
%\usepackage{psfrag}
% need this for including graphics (\includegraphics)
%\usepackage{graphicx}
% for neatly defining theorems and propositions
%
% making logically defined graphics
%%%\usepackage{xypic}

% there are many more packages, add them here as you need them

% define commands here

\newcommand{\sR}[0]{\mathbb{R}}
\newcommand{\sC}[0]{\mathbb{C}}
\newcommand{\sN}[0]{\mathbb{N}}
\newcommand{\sZ}[0]{\mathbb{Z}}

 \usepackage{bbm}
 \newcommand{\Z}{\mathbbmss{Z}}
 \newcommand{\C}{\mathbbmss{C}}
 \newcommand{\R}{\mathbbmss{R}}
 \newcommand{\Q}{\mathbbmss{Q}}



\newcommand*{\norm}[1]{\lVert #1 \rVert}
\newcommand*{\abs}[1]{| #1 |}



\newtheorem{thm}{Theorem}
\newtheorem{defn}{Definition}
\newtheorem{prop}{Proposition}
\newtheorem{lemma}{Lemma}
\newtheorem{cor}{Corollary}
\begin{document}
Suppose $X$ is a set and $P$ is a property defined as follows:
\begin{eqnarray*}
& &  \mbox{$X$ has property $P$ if and only if} \\
& &  \mbox{$\forall Y[$ $Y$ satisfies condition $1] \Rightarrow$ $Y$ satisfies condition $2$ }
\end{eqnarray*} 
where condition $1$ and condition $2$ define the property. 
If condition $1$ is never satisfied then $X$ satisfies property $P$
\emph{vacuously}.

\subsubsection*{Examples}
\begin{enumerate}
\item If $X$ is the set $\{1,2,3\}$ and $P$ is the property defined as above with condition $1=$ $Y$ is a infinite subset of $X$, and condition $2=$ $Y$ contains $7$.  Then $X$ has property $P$ vacously; every infinite subset of $\{1,2,3\}$ contains the number $7$ \cite{wiki}. 
\item The empty set is a Hausdorff space (vacuously). 
\item Suppose property $P$ is defined by the statement\,: \\
\emph{The present King of France does not exist.}\\
Then either of the following propositions is satisfied vacuously. \\
\emph{The present king of France is bald.}\\
\emph{The present King of France is not bald.} 
\end{enumerate}

\begin{thebibliography}{9}
\bibitem{wiki}Wikipedia \PMlinkexternal{entry on Vacuous truth}{http://en.wikipedia.org/wiki/Vacuous_truth}.
\end{thebibliography}
%%%%%
%%%%%
\end{document}
