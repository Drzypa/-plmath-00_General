\documentclass[12pt]{article}
\usepackage{pmmeta}
\pmcanonicalname{Rigid}
\pmcreated{2013-03-22 14:38:10}
\pmmodified{2013-03-22 14:38:10}
\pmowner{matte}{1858}
\pmmodifier{matte}{1858}
\pmtitle{rigid}
\pmrecord{11}{36219}
\pmprivacy{1}
\pmauthor{matte}{1858}
\pmtype{Definition}
\pmcomment{trigger rebuild}
\pmclassification{msc}{00-01}
\pmsynonym{rigidity result}{Rigid}
\pmsynonym{rigidity theorem}{Rigid}
\pmsynonym{rigidity}{Rigid}

% this is the default PlanetMath preamble.  as your knowledge
% of TeX increases, you will probably want to edit this, but
% it should be fine as is for beginners.

% almost certainly you want these
\usepackage{amssymb}
\usepackage{amsmath}
\usepackage{amsfonts}
\usepackage{amsthm}

\usepackage{mathrsfs}

% used for TeXing text within eps files
%\usepackage{psfrag}
% need this for including graphics (\includegraphics)
%\usepackage{graphicx}
% for neatly defining theorems and propositions
%
% making logically defined graphics
%%%\usepackage{xypic}

% there are many more packages, add them here as you need them

% define commands here

\newcommand{\sR}[0]{\mathbb{R}}
\newcommand{\sC}[0]{\mathbb{C}}
\newcommand{\sN}[0]{\mathbb{N}}
\newcommand{\sZ}[0]{\mathbb{Z}}

 \usepackage{bbm}
 \newcommand{\Z}{\mathbbmss{Z}}
 \newcommand{\C}{\mathbbmss{C}}
 \newcommand{\R}{\mathbbmss{R}}
 \newcommand{\Q}{\mathbbmss{Q}}



\newcommand*{\norm}[1]{\lVert #1 \rVert}
\newcommand*{\abs}[1]{| #1 |}



\newtheorem{thm}{Theorem}
\newtheorem{defn}{Definition}
\newtheorem{prop}{Proposition}
\newtheorem{lemma}{Lemma}
\newtheorem{cor}{Corollary}
\begin{document}
Suppose $C$ is a collection of mathematical objects 
(for instance, sets, or functions).
Then we say that $C$ is \emph{rigid} if every $c\in C$ 
is uniquely determined by less information about $c$ than 
one would expect.

It should be emphasized that the above ``definition''  does not
define a \emph{mathematical object}. Instead, it describes in what sense
the adjective rigid is typically used in mathematics, 
by mathematicians. 

Let us illustrate this by some examples:

\begin{enumerate}
%\item Liouville's theorem: If 
%the difference of two entire functions is bounded, the functions are equal. 
\item Harmonic functions on the unit disk are rigid in the sense that
 they are uniquely determined by their boundary values. 
\item By the fundamental theorem of algebra, polynomials in $\sC$
are rigid in the sense that any polynomial is completely determined by
its values on any countably infinite set, say $\sN$, or the unit disk. 
\item Linear maps $\mathscr{L}(X,Y)$ between vector spaces $X,Y$
are rigid in the sense that any $L\in\mathscr{L}(X,Y)$ is completely 
determined by its values on any set of basis vectors of $X$. 
\item Mostow's rigidity theorem
\end{enumerate}
%%%%%
%%%%%
\end{document}
