\documentclass[12pt]{article}
\usepackage{pmmeta}
\pmcanonicalname{Welldefined}
\pmcreated{2013-03-22 17:31:32}
\pmmodified{2013-03-22 17:31:32}
\pmowner{pahio}{2872}
\pmmodifier{pahio}{2872}
\pmtitle{well-defined}
\pmrecord{9}{39921}
\pmprivacy{1}
\pmauthor{pahio}{2872}
\pmtype{Definition}
\pmcomment{trigger rebuild}
\pmclassification{msc}{00A05}
\pmsynonym{well defined}{Welldefined}
\pmrelated{function}
\pmrelated{WellDefinednessOfProductOfFinitelyGeneratedIdeals}

% this is the default PlanetMath preamble.  as your knowledge
% of TeX increases, you will probably want to edit this, but
% it should be fine as is for beginners.

% almost certainly you want these
\usepackage{amssymb}
\usepackage{amsmath}
\usepackage{amsfonts}

% used for TeXing text within eps files
%\usepackage{psfrag}
% need this for including graphics (\includegraphics)
%\usepackage{graphicx}
% for neatly defining theorems and propositions
 \usepackage{amsthm}
% making logically defined graphics
%%%\usepackage{xypic}

% there are many more packages, add them here as you need them

% define commands here

\theoremstyle{definition}
\newtheorem*{thmplain}{Theorem}

\begin{document}
A mathematical concept is \emph{well-defined} (German {\em wohldefiniert}, French {\em bien d\'efini}), if its contents is 
\PMlinkescapetext{independent} on the form or the alternative representative which is used for defining it.  

For example, in defining the \PMlinkname{power}{FractionPower} $x^r$ with $x$ a positive real and $r$ a rational number, 
we can freely choose the fraction form $\frac{m}{n}$ ($m\in\mathbb{Z}$,\, $n\in\mathbb{Z}_+$) of $r$ and take 
$$x^r \;:=\; \sqrt[n]{x^m}$$ 
and be sure that the value of $x^r$ does not depend on that choice (this is justified in the entry fraction power).  So, 
the $x^r$ is well-defined.

In many instances well-defined is a synonym for the formal definition of a function between sets.  For example, 
the function\, $f(x) := x^2$\, is a well-defined function from the real numbers to the real numbers because
every input, $x$, is assigned to precisely one output, $x^2$.  However,\, $f(x) := \pm\sqrt{x}$\, is not well-defined
in that one input $x$ can be assigned any one of two possible outputs, $\sqrt{x}$ or $-\sqrt{x}$.

More subtle examples include expressions such as
\begin{equation*}
   f\!\left(\frac{a}{b}\right) \;:=\; a\!+\!b,\quad \frac{a}{b}\in\mathbb{Q}.
\end{equation*}
Certainly every input has an output, for instance,\, $f(1/2) = 3$.  However, the expression is \emph{not}
well-defined since\, $1/2 = 2/4$\, yet\, $f(1/2) = 3$\, while\, $f(2/4) = 6$\, and\, $3 \neq 6$.

One must question whether a function is well-defined whenever it is defined on a domain of equivalence classes 
in such a manner that each output is determined for a representative of each equivalence class.  For example, the
function\, $f(a/b) := a\!+\!b$\, was defined using the representative $a/b$ of the equivalence class of fractions
equivalent to $a/b$.
%%%%%
%%%%%
\end{document}
