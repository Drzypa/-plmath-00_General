\documentclass[12pt]{article}
\usepackage{pmmeta}
\pmcanonicalname{Strict}
\pmcreated{2013-03-22 14:45:23}
\pmmodified{2013-03-22 14:45:23}
\pmowner{rspuzio}{6075}
\pmmodifier{rspuzio}{6075}
\pmtitle{strict}
\pmrecord{12}{36397}
\pmprivacy{1}
\pmauthor{rspuzio}{6075}
\pmtype{Definition}
\pmcomment{trigger rebuild}
\pmclassification{msc}{00-01}
\pmdefines{strict}
\pmdefines{strictly}

% this is the default PlanetMath preamble.  as your knowledge
% of TeX increases, you will probably want to edit this, but
% it should be fine as is for beginners.

% almost certainly you want these
\usepackage{amssymb}
\usepackage{amsmath}
\usepackage{amsfonts}

% used for TeXing text within eps files
%\usepackage{psfrag}
% need this for including graphics (\includegraphics)
%\usepackage{graphicx}
% for neatly defining theorems and propositions
%\usepackage{amsthm}
% making logically defined graphics
%%%\usepackage{xypic}

% there are many more packages, add them here as you need them

% define commands here
\begin{document}
In mathematical writing, the adjective \emph{strict} is used in to modify technical \PMlinkescapetext{terms} which have \PMlinkescapetext{multiple} meanings.  It indicates that the exclusive meaning of the term is to be understood.  (More formally, one could say that this is the meaning which implies the other meanings.)

This term is commonly used in the context of inequalities --- the phrases ``strictly less than'' and ``strictly greater than'' mean ``less than and not equal to'' and ``greater than and not equal to'', respectively.  A related use occurs when comparing numbers to zero --- ``strictly positive'' and ``strictly negative'' mean ``positive and not equal to zero'' and ``negative and not equal to zero'', respectively.  Also, in the context of functions, the adverb ``strictly '' is used to modify the terms ``monotonic'', ``increasing'', and ``decreasing''.

On the other hand, sometimes one wants to specify the inclusive meanings of terms.  In the context of comparisons, one can use the phrases ``non-negative'', ``non-positive'', ``non-increasing'', and ``non-decreasing'' to make it clear that the inclusive sense of the terms is intended.

Using such terminology helps avoid possible ambiguity and confusion.  For instance, upon reading the phrase ``x is negative'', it is not immediately clear whether x = 0 is possible, since some authors consider zero to be positive while others consider zero not to be positive.  Therefore, it is prudent to write either ``x is strictly negative'' or ``x is non-positive'' unless the distinction is unimportant or the context makes obvious which meaning is intended or it has been explicitly stated that the term "positive" is to be used in only one sense.  (Here, in Planet Math, we have taken the third option by adopting the convention that zero is neither positive nor negative.  Hence, the terms "strictly positive" and "positive" are synonyms here.  However, the adverb "strictly" may still be necessary in some of the other contexts described avove.)
%%%%%
%%%%%
\end{document}
