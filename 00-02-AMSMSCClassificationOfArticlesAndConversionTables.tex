\documentclass[12pt]{article}
\usepackage{pmmeta}
\pmcanonicalname{AMSMSCClassificationOfArticlesAndConversionTables}
\pmcreated{2013-03-22 19:22:28}
\pmmodified{2013-03-22 19:22:28}
\pmowner{bci1}{20947}
\pmmodifier{bci1}{20947}
\pmtitle{AMS MSC classification of articles and conversion tables}
\pmrecord{26}{42328}
\pmprivacy{1}
\pmauthor{bci1}{20947}
\pmtype{Topic}
\pmcomment{trigger rebuild}
\pmclassification{msc}{00-02}
\pmclassification{msc}{00-01}
%\pmkeywords{AMS MSC classification 2010 - pdf}
\pmdefines{MSC}
\pmdefines{philosophy of mathematics classification}
\pmdefines{of articles and conversion tablesmiscellaneous mathematics classifications}

\endmetadata

% this is the default PlanetMath preamble. as your knowledge
% of TeX increases, you will probably want to edit this, but
\usepackage{amsmath, amssymb, amsfonts, amsthm, amscd, latexsym}
%%\usepackage{xypic}
\usepackage[mathscr]{eucal}
% define commands here
\theoremstyle{plain}
\newtheorem{lemma}{Lemma}[section]
\newtheorem{proposition}{Proposition}[section]
\newtheorem{theorem}{Theorem}[section]
\newtheorem{corollary}{Corollary}[section]
\theoremstyle{definition}
\newtheorem{definition}{Definition}[section]
\newtheorem{example}{Example}[section]
%\theoremstyle{remark}
\newtheorem{remark}{Remark}[section]
\newtheorem*{notation}{Notation}
\newtheorem*{claim}{Claim}
\renewcommand{\thefootnote}{\ensuremath{\fnsymbol{footnote%%@
}}}
\numberwithin{equation}{section}
\newcommand{\Ad}{{\rm Ad}}
\newcommand{\Aut}{{\rm Aut}}
\newcommand{\Cl}{{\rm Cl}}
\newcommand{\Co}{{\rm Co}}
\newcommand{\DES}{{\rm DES}}
\newcommand{\Diff}{{\rm Diff}}
\newcommand{\Dom}{{\rm Dom}}
\newcommand{\Hol}{{\rm Hol}}
\newcommand{\Mon}{{\rm Mon}}
\newcommand{\Hom}{{\rm Hom}}
\newcommand{\Ker}{{\rm Ker}}
\newcommand{\Ind}{{\rm Ind}}
\newcommand{\IM}{{\rm Im}}
\newcommand{\Is}{{\rm Is}}
\newcommand{\ID}{{\rm id}}
\newcommand{\GL}{{\rm GL}}
\newcommand{\Iso}{{\rm Iso}}
\newcommand{\Sem}{{\rm Sem}}
\newcommand{\St}{{\rm St}}
\newcommand{\Sym}{{\rm Sym}}
\newcommand{\SU}{{\rm SU}}
\newcommand{\Tor}{{\rm Tor}}
\newcommand{\U}{{\rm U}}
\newcommand{\A}{\mathcal A}
\newcommand{\Ce}{\mathcal C}
\newcommand{\D}{\mathcal D}
\newcommand{\E}{\mathcal E}
\newcommand{\F}{\mathcal F}
\newcommand{\G}{\mathcal G}
\newcommand{\Q}{\mathcal Q}
\newcommand{\R}{\mathcal R}
\newcommand{\cS}{\mathcal S}
\newcommand{\cU}{\mathcal U}
\newcommand{\W}{\mathcal W}
\newcommand{\bA}{\mathbb{A}}
\newcommand{\bB}{\mathbb{B}}
\newcommand{\bC}{\mathbb{C}}
\newcommand{\bD}{\mathbb{D}}
\newcommand{\bE}{\mathbb{E}}
\newcommand{\bF}{\mathbb{F}}
\newcommand{\bG}{\mathbb{G}}
\newcommand{\bK}{\mathbb{K}}
\newcommand{\bM}{\mathbb{M}}
\newcommand{\bN}{\mathbb{N}}
\newcommand{\bO}{\mathbb{O}}
\newcommand{\bP}{\mathbb{P}}
\newcommand{\bR}{\mathbb{R}}
\newcommand{\bV}{\mathbb{V}}
\newcommand{\bZ}{\mathbb{Z}}
\newcommand{\bfE}{\mathbf{E}}
\newcommand{\bfX}{\mathbf{X}}
\newcommand{\bfY}{\mathbf{Y}}
\newcommand{\bfZ}{\mathbf{Z}}
\renewcommand{\O}{\Omega}
\renewcommand{\o}{\omega}
\newcommand{\vp}{\varphi}
\newcommand{\vep}{\varepsilon}
\newcommand{\diag}{{\rm diag}}
\newcommand{\grp}{{\mathbb G}}
\newcommand{\dgrp}{{\mathbb D}}
\newcommand{\desp}{{\mathbb D^{\rm{es}}}}
\newcommand{\Geod}{{\rm Geod}}
\newcommand{\geod}{{\rm geod}}
\newcommand{\hgr}{{\mathbb H}}
\newcommand{\mgr}{{\mathbb M}}
\newcommand{\ob}{{\rm Ob}}
\newcommand{\obg}{{\rm Ob(\mathbb G)}}
\newcommand{\obgp}{{\rm Ob(\mathbb G')}}
\newcommand{\obh}{{\rm Ob(\mathbb H)}}
\newcommand{\Osmooth}{{\Omega^{\infty}(X,*)}}
\newcommand{\ghomotop}{{\rho_2^{\square}}}
\newcommand{\gcalp}{{\mathbb G(\mathcal P)}}
\newcommand{\rf}{{R_{\mathcal F}}}
\newcommand{\glob}{{\rm glob}}
\newcommand{\loc}{{\rm loc}}
\newcommand{\TOP}{{\rm TOP}}
\newcommand{\wti}{\widetilde}
\newcommand{\what}{\widehat}
\renewcommand{\a}{\alpha}
\newcommand{\be}{\beta}
\newcommand{\ga}{\gamma}
\newcommand{\Ga}{\Gamma}
\newcommand{\de}{\delta}
\newcommand{\del}{\partial}
\newcommand{\ka}{\kappa}
\newcommand{\si}{\sigma}
\newcommand{\ta}{\tau}
\newcommand{\lra}{{\longrightarrow}}
\newcommand{\ra}{{\rightarrow}}
\newcommand{\rat}{{\rightarrowtail}}
\newcommand{\oset}[1]{\overset {#1}{\ra}}
\newcommand{\osetl}[1]{\overset {#1}{\lra}}
\newcommand{\hr}{{\hookrightarrow}}

\begin{document}
\section{Links to the AMS MSC 2010 Classification PDF of all MSC entries available, and the AMS MSC website}
Because the AMS MSC classification list or table does not seem to be available at present when creating a new entry two links are here provided to the AMS websites that list the complete Table of AMS MSC2010 classifications:

\PMlinkexternal{All MSC 2010 in one PDF :}{http://www.ams.org/mathscinet/msc/pdfs/classifications2010.pdf}

\PMlinkexternal{The AMS MSC website with its maths specialized Search Engine}{http://www.ams.org/mathscinet/msc/msc2010.html}


\subsection{Conversion Tables}

http://www.ams.org/mathscinet/msc/pdfs/classifications2010.pdf

CONVERSIONS: http://www.ams.org/mathscinet/msc/conv.html?from=2000 


$MSC2000~ Classification~ Codes~ \to  ~MSC2010~ Classification~ Codes ~Update.$
Date: 14 October 2009


http://www.ams.org/mathscinet/msc/conv.html?from=2010

MSC2010 Classification Codes --> MSC2000 Classification Codes


\subsection{General Classifications}
{\bf 00-01 Instructional Expositions}

00-02 Research Expositions

00A05 General mathematics

00A35 Methodology of mathematics, didactics

00A66 Mathematics and visual arts, visualization

00A79 Physics 

00A69 General applied mathematics

00A73 Dimensional analysis

00A15 Bibliographies

00A71 Theory of mathematical modeling 

00A30 Philosophy of mathematics and 03A05

00A99 Miscellaneous topics

00B99 None of the above, but in this section}.


\section{Several Examples of AMS MSC Classifications Utilized in PM articles}

msc:00-01, msc:00-02

00A15 Bibliographies


\subsection{Algebraic Logics}

03G05 Boolean algebras [See also 06Exx]

03G12 quantum logic [See also 06C15, 81P10]

03G20 $\L{}$ukasiewicz and Post algebras [See also 06D25, 06D30]

03G10 Lattices and related structures [See also 06Bxx]

03G30 Categorical logic, topoi [See also 18B25, 18C05, 18C10]

03H10 Other applications of nonstandard models (economics, physics, etc.)

03G15 Cylindric and polyadic algebras; relation algebras

03G20 Lukasiewicz and Post algebras [See also 06D25, 06D30]

03G25 Other algebras related to logic [See also 03F45, 06D20, 06E25, 06F35]

03G27 Abstract algebraic logic

03G30 Categorical logic, topoi [See also 18B25, 18C05, 18C10]

\subsection{COMBINATORICS}

05-XX COMBINATORICS 

For finite fields, see 11Txxg

05-00 General reference works (handbooks, dictionaries, bibliographies,etc.)

05-01 Instructional exposition (textbooks, tutorial papers, etc.)

05-02 Research exposition (monographs, survey articles)

05Axx Enumerative combinatorics --For enumeration in graph theory, see 05C30g , 05A05 Permutations, words, matrices

05A10 Factorials, binomial coefficients, combinatorial functions [See also 11B65, 33Cxx]

05A15 Exact enumeration problems, generating functions [See also 33Cxx,33Dxx]

05A16 Asymptotic enumeration

05A17 Partitions of integers [See also 11P81, 11P82, 11P83]

05A18 Partitions of sets

05A19 Combinatorial identities, bijective combinatorics

05A20 Combinatorial inequalities

05A30 $q$-calculus and related topics [See also 33Dxx]

05A40 Umbral calculus

05Bxx Designs and configurations--For applications of design theory,  see 94C30g

05B05 Block designs [See also 51E05, 62K10]

05B07 Triple systems

05B10 Difference sets (number-theoretic, group-theoretic, etc.)[See also 11B13]

05B15 Orthogonal arrays, Latin squares, Room squares

05B20 Matrices (incidence, Hadamard, etc.)

05B25 Finite geometries [See also 51D20, 51Exx]

05B30 Other designs, configurations [See also 51E30]

05B35 Matroids, geometric lattices [See also 52B40, 90C27]

05B40 Packing and covering [See also 11H31, 52C15, 52C17]

05B45 Tessellation and tiling problems [See also 52C20, 52C22]

05B50 Polyominoes

05B99 None of the above, but in this section

05Cxx Graph theory fFor applications of graphs, see 68R10, 81Q30, 81T15, 82B20, 82C20, 90C35, 92E10, 94C15g

05C05 Trees

05C07 Vertex degrees [See also 05E30]

05C10 Planar graphs; geometric and topological aspects of graph theory [See also 57M15, 57M25]

05C12 Distance in graphs

05C15 Coloring of graphs and hypergraphs


\subsection{ORDER, LATTICES, ORDERED ALGEBRAIC STRUCTURES}
[See also 18B35]

06-00 General reference works (handbooks, dictionaries, bibliographies, etc.)

06-01 Instructional exposition (textbooks, tutorial papers, etc.)

06-02 Research exposition (monographs, survey articles)

06-06 Proceedings, conferences, collections, etc.

06Axx Ordered sets

06A05 Total order

06A06 Partial order, general

06A07 Combinatorics of partially ordered sets

06A11 Algebraic aspects of posets

06A12 Semilattices [See also 20M10; for topological semilattices see 22A26]

06A15 Galois correspondences, closure operators

06A75 Generalizations of ordered sets

06A99 None of the above, but in this section

06Bxx Lattices [See also 03G10]

06B05 Structure theory

06B10 Ideals, congruence relations

06B15 Representation theory

06B20 Varieties of lattices

06B23 Complete lattices, completions

06B25 Free lattices, projective lattices, word problems [See also 03D40,08A50, 20F10]

06B30 Topological lattices, order topologies [See also 06F30, 22A26, 54F05, 54H12]

06B35 Continuous lattices and posets, applications [See also 06B30, 06D10,06F30, 18B35, 22A26, 68Q55]

06B75 Generalizations of lattices

06B99 None of the above, but in this section

06Cxx Modular lattices, complemented lattices

06C05 Modular lattices, Desarguesian lattices

06C10 Semimodular lattices, geometric lattices

06C15 Complemented lattices, orthocomplemented lattices and posets [See also 03G12, 81P10]

06C20 Complemented modular lattices, continuous geometries

06C99 None of the above, but in this section

06Dxx Distributive lattices

06D05 Structure and representation theory

06D10 Complete distributivity

06D15 Pseudocomplemented lattices

06D20 Heyting algebras [See also 03G25]

06D22 Frames, locales For topological questions see 54{XXg

06D25 Post algebras [See also 03G20]

06D30 De Morgan algebras, Lukasiewicz algebras [See also 03G20]

06D35 MV--algebras

06D50 Lattices and duality

06D72 Fuzzy lattices (soft algebras) and related topics

06D75 Other generalizations of distributive lattices

06D99 None of the above, but in this section

06Exx Boolean algebras (Boolean rings) [See also 03G05]

06E05 Structure theory

06E10 Chain conditions, complete algebras

06E15 Stone spaces (Boolean spaces) and related structures

06E20 Ring--theoretic properties [See also 16E50, 16G30]

\subsection{General Algebraic Systems}

08-XX GENERAL ALGEBRAIC SYSTEMS

08-00 General reference works (handbooks, dictionaries, bibliographies, etc.)

08-02 Research exposition (monographs, survey articles)

08-06 Proceedings, conferences, collections, etc.

08Axx algebraic structures [See also 03C05]

08A02 Relational systems, laws of composition 08A05 Structure theory

08A30 Subalgebras, congruence relations

08A35 Automorphisms, endomorphisms

08A70 Applications of universal algebra in computer science

08A72 Fuzzy algebraic structures

08Cxx Other classes of algebra

08C05 Categories of algebras [See also 18C05]

08C10 Axiomatic model classes [See also 03Cxx, in particular 03C60]

08C15 Quasivarieties

08C20 Natural dualities for classes of algebras [See also 06E15, 18A40, 22A30]

08C99 None of the above, but in this section

\subsection{Algebraic number theory, Galois theory, cohomology and polynomials}
11Sxx Algebraic number theory: local and $p$-adic fields

11S05 Polynomials

11S15 Ramification and extension theory

11S20 Galois theory

11S23 Integral representations

11S25 Galois cohomology [See also 12Gxx, 16H05]

11S31 Class field theory; $p$-adic formal groups [See also 14L05]

11S37 Langlands--Weil conjectures, nonabelian class field theory [See also 11Fxx, 22E50]

11S40 Zeta functions and $L$--functions [See also 11M41, 19F27]

11S45 Algebras and orders, and their zeta functions [See also 11R52, 11R54, 16Hxx, 16Kxx]

11S70 $K$--theory of local fields [See also 19Fxx]

11S80 Other analytic theory (analogues of beta and gamma functions, $p$-adic integration, etc.)

11S82 Non--Archimedean dynamical systems [See mainly 37Pxx]

11S85 Other nonanalytic theory

11S90 Prehomogeneous vector spaces

11S99 None of the above, but in this section

11Txx Finite fields and commutative rings (number--theoretic aspects)

11T06 Polynomials

11T22 Cyclotomy

11T23 Exponential sums

11T24 Other character sums and Gauss sums

11T30 Structure theory

11T55 Arithmetic theory of polynomial rings over finite fields

11T60 Finite upper half--planes

11T71 Algebraic coding theory; cryptography

11T99 None of the above, but in this section

11Uxx Connections with logic

11U05 Decidability [See also 03B25]

11U07 Ultraproducts [See also 03C20]

11U09 Model theory [See also 03Cxx]

11U10 Nonstandard arithmetic [See also 03H15]

11U99 None of the above, but in this section


\subsection{POLYNOMIALS and Field Theory}
12-XX FIELD THEORY AND POLYNOMIALS

12-00 General reference works (handbooks, dictionaries, bibliographies, etc.)

12-01 Instructional exposition (textbooks, tutorial papers, etc.)

12-02 Research exposition (monographs, survey articles)

12-06 Proceedings, conferences, collections, etc.

12Dxx Real and complex fields

12D05 Polynomials: factorization

12D10 Polynomials: location of zeros (algebraic theorems) -For the analytic
theory, see 26C10, 30C15g

12D15 Fields related with sums of squares (formally real fields, Pythagorean fields, etc.) [See also 11Exx]

12D99 None of the above, but in this section

12Exx General field theory

12E05 Polynomials (irreducibility, etc.)

12E10 Special polynomials

12E12 Equations

12E15 Skew fields, division rings [See also 11R52, 11R54, 11S45, 16Kxx]

12E20 Finite fields (field--theoretic aspects)

12E25 Hilbertian fields; Hilbert' s irreducibility theorem

12E30 Field arithmetic

12E99 None of the above, but in this section

12Fxx Field extensions

12F05 Algebraic extensions

12F10 Separable extensions, Galois theory

12F12 Inverse Galois theory

12F15 Inseparable extensions

12F20 Transcendental extensions

12F99 None of the above, but in this section

12Gxx Homological methods (field theory)

12G05 Galois cohomology [See also 14F22, 16Hxx, 16K50]

12G10 Cohomological dimension

12G99 None of the above, but in this section

12Hxx Differential and difference algebra

12H05 Differential algebra [See also 13Nxx]

12H10 Difference algebra [See also 39Axx]

12H20 Abstract differential equations [See also 34Mxx]

12H25 $p$-adic differential equations [See also 11S80, 14G20]

12H99 None of the above, but in this section

12Jxx Topological fields

12J05 Normed fields

12J10 Valued fields

12J12 Formally $p$-adic fields

12J15 Ordered fields

12J17 Topological semi fields

12J20 General valuation theory [See also 13A18]

12J25 Non-Archimedean valued fields [See also 30G06, 32P05, 46S10, 47S10]

12J27 Krasner--Tate algebras [See mainly 32P05; see also 46S10, 47S10]

12J99 None of the above, but in this section

12Kxx Generalizations of fields

12K05 Near--fields [See also 16Y30]

12K10 Semi fields [See also 16Y60]

12K99 None of the above, but in this section

12Lxx Connections with logic

12L05 Decidability [See also 03B25]

12L10 Ultraproducts [See also 03C20]

12L12 Model theory [See also 03C60]

12L15 Nonstandard arithmetic [See also 03H15]


\subsection{COMMUTATIVE ALGEBRA}

13-XX COMMUTATIVE ALGEBRA

13-00 General reference works (handbooks, dictionaries, bibliographies, etc.) 

13-01 Instructional exposition (textbooks, tutorial papers, etc.) 

13-02 Research exposition (monographs, survey articles)

13D05 Homological dimension

13D07 Homological functors on modules (Tor, Ext, etc.)

13D09 Derived categories

13Axx General commutative ring theory

13A02 Graded rings [See also 16W50]

13A05 Divisibility; factorizations [See also 13F15]

13A15 Ideals; multiplicative ideal theory

13A18 Valuations and their generalizations [See also 12J20]

13A30 Associated graded rings of ideals (Rees ring, form ring), analytic spread and related topics

13A35 Characteristic--methods (Frobenius endomorphism) and reduction to characteristic ; tight closure [See also 13B22]

13A50 Actions of groups on commutative rings; invariant theory [See also 14L24]

13A99 None of the above, but in this section

13Bxx Ring extensions and related topics

13B02 Extension theory

13B05 Galois theory

13-03 Historical (must also be assigned at least one classification number from Section 01)

13-04 Explicit machine computation and programs (not the theory of computation or programming)

13-06 Proceedings, conferences, collections, etc.

13B21 Integral dependence; going up, going down

13B22 Integral closure of rings and ideals [See also 13A35]; integrally closed rings, related rings (Japanese, etc.)

13B25 Polynomials over commutative rings [See also 11C08, 11T06, 13F20, 13M10]

13B30 Rings of fractions and localization [See also 16S85] 13B35 Completion [See also 13J10]

13B40 Etale and at extensions; Henselization; Artin approximation [See also 13J15, 14B12, 14B25]

13B99 None of the above, but in this section

13Cxx Theory of modules and ideals

13B02 Extension theory

13B05 Galois theory

13B10 Morphisms


\subsection{ALGEBRAIC GEOMETRY}

14-XX ALGEBRAIC GEOMETRY

14-00 General reference works (handbooks, dictionaries, bibliographies, etc.)

14-01 Instructional exposition (textbooks, tutorial papers, etc.)

14-02 Research exposition (monographs, survey articles)

14-06 Proceedings, conferences, collections, etc.

14Axx Foundations

14A05 Relevant commutative algebra [See also 13XX] 14A10 Varieties and morphisms

14A15 Schemes and morphisms

14A20 Generalizations (algebraic spaces, stacks)

14A22 Noncommutative algebraic geometry [See also 16S38] 14A25 Elementary questions

14A99 None of the above, but in this section

14Bxx Local theory

14B05 Singularities [See also 14E15, 14H20, 14J17, 32Sxx, 58Kxx]

14B07 deformations of singularities [See also 14D15, 32S30]

14B10 Infnitesimal methods [See also 13D10]

14B12 Local deformation theory, Artin approximation, etc. [See also 13B40, 13D10]

14B15 Local cohomology [See also 13D45, 32C36]

14B20 Formal neighborhoods

14B25 Local structure of morphisms: etale, at, etc. [See also 13B40]

14B25 Local structure of morphisms: etale, at, etc. [See also 13B40], infinitesimal methods [See also 14B10, 14B12, 14D15, 32Gxx]

13D15 Grothendieck groups, $K$--theory [See also 14C35, 18F30, 19Axx, 19D50]

13D22 Homological conjectures (intersection theorems)

13D30 Torsion theory [See also 13C12, 18E40]

13D40 Hilbert--Samuel and Hilbert--Kunz functions; Poincare series

13D45 Local cohomology [See also 14B15]

13D99 None of the above, but in this section

13Exx Chain conditions, finiteness conditions

13E05 Noetherian rings and modules

13E10 Artinian rings and modules, finite--dimensional algebras

13E15 Rings and modules of finite generation or presentation; number of generators

13E99 None of the above, but in this section

13Fxx Arithmetic rings and other special rings

13F05 Dedekind, Prufer, Krull and Mori rings and their extensions

14Hxx {\bf Curves}

14H05 Algebraic functions; function fields [See also 11R58]

14H10 Families, moduli (algebraic)

14H15 Families, moduli (analytic) [See also 30F10, 32G15]

14H20 Singularities, local rings [See also 13Hxx, 14B05]

14H25 Arithmetic ground fields [See also 11Dxx, 11G05, 14Gxx]

14H30 coverings, fundamental group [See also 14E20, 14F35]

14H37 Automorphisms

14H40 Jacobians, Prym varieties [See also 32G20]

14H42 Theta functions; Schottky problem [See also 14K25, 32G20]

14H45 Special curves and curves of low genus

14H50 Plane and space curves


\subsection{Category Theory}

18Axx general theory of categories and functors 

18A05 Definitions, generalizations 

18A10 graphs, diagram schemes, precategories [See especially 20L05]

18A15 Foundations, relations to logic and deductive systems [See also 03-XX]

18A20 epimorphisms, monomorphisms, special classes of morphisms, null Morphisms

18A22 Special properties of functors (faithful, full, etc.)

18A23 Natural morphisms, dinatural morphisms

18A25 functor categories, comma categories

18A30 limits and colimits (products, sums, directed limits, pushouts, fiber products, equalizers, kernels, ends and coends, etc.)

18A32 Factorization of morphisms, substructures, quotient structures, congruences, amalgams

18A35 Categories admitting limits (complete categories), functors preserving limits, completions

18A40 adjoint functors (universal constructions, reective subcategories, Kan extensions, etc.)

18A99 None of the above, but in this section

18Bxx Special categories

18B05 Category of sets, characterizations [See also 03XX]

18D05 (Category theory; homological algebra, Categories with structure: Double categories, $2$-categories, bicategories and generalizations)

18-00 (Category theory; homological algebra: General reference works (handbooks, dictionaries, bibliographies, etc.))

18E05 (Category theory; homological algebra: Abelian categories, Preadditive, additive categories)


\subsection{Group Theory}

20C30 Representations of finite symmetric groups

20C32 Representations of infinite symmetric groups

20F05 Generators, relations, and presentations

20F06 Cancellation theory; application of van Kampen diagrams [See also 57M05]

20F11 Groups of finite Morley rank [See also 03C45, 03C60]

20F12 Commutator calculus

20F14 Derived series, central series, and generalizations

20F16 Solvable groups, supersolvable groups [See also 20D10]


\subsection{REAL FUNCTIONS}

26-XX REAL FUNCTIONS [See also 54C30]

26-00 General reference works (handbooks, dictionaries, bibliographies,etc.)

26-01 Instructional exposition (textbooks, tutorial papers, etc.)

26-02 Research exposition (monographs, survey articles)

26Axx Functions of one variable

26A03 Foundations: limits and generalizations, elementary topology of the line

26A06 One--variable calculus

26A09 Elementary functions

26A12 Rate of growth of functions, orders of infinity, slowly varying functions [See also 26A48]

26A15 Continuity and related questions (modulus of continuity, semicontinuity, discontinuities, etc.)
 -For properties determined by Fourier coefficients, see 42A16; for those determined by approximation properties, see 41A25, 41A27g

26A16 Lipschitz (Holder) classes

26A18 Iteration [See also 37Bxx, 37Cxx, 37Exx, 39B12, 47H10, 54H25]

26A21 Classification of real functions; Baire classification of sets and functions [See also 03E15, 28A05, 54C50, 54H05]

26A24 Differentiation (functions of one variable): general theory, generalized derivatives, mean--value theorems [See also 28A15]

26A27 Non-differentiability (nondifferentiable functions, points of non-differentiability), discontinuous derivatives

26A30 Singular functions, Cantor functions, functions with other special properties

26A33 Fractional derivatives and integrals

26A36 Anti-differentiation

26A39 Denjoy and Perron integrals, other special integrals

26A42 Integrals of Riemann, Stieltjes and Lebesgue type [See also 28{XX]

26A45 Functions of bounded variation, generalizations

26A46 Absolutely continuous functions

26A48 Monotonic functions, generalizations

26A51 Convexity, generalizations

26A99 None of the above, but in this section

26Bxx Functions of several variables

26B05 Continuity and differentiation questions

26B10 Implicit function theorems, Jacobians, transformations with several variables

26B12 Calculus of vector functions

26B15 Integration: length, area, volume [See also 28A75, 51M25]

26B20 Integral formulas (Stokes, Gauss, Green, etc.)

26B25 Convexity, generalizations

26B30 Absolutely continuous functions, functions of bounded variation

26B35 Special properties of functions of several variables, Holder conditions, etc.

26B40 Representation and superposition of functions

26B99 None of the above, but in this section

26Cxx Polynomials, rational functions

26C05 Polynomials: analytic properties, etc. [See also 12Dxx, 12Exx]

26C10 Polynomials: location of zeros [See also 12D10, 30C15, 65H05]

26C15 Rational functions [See also 14Pxx]

26C99 None of the above, but in this section

26Dxx Inequalities 
-For maximal function inequalities, see 42B25; for functional inequalities, see 39B72; for probabilistic inequalities, see 60E15g

26D05 Inequalities for trigonometric functions and polynomials

26D07 Inequalities involving other types of functions

26D10 Inequalities involving derivatives and differential and integral operators

26D15 Inequalities for sums, series and integrals

26D20 Other analytical inequalities

26D99 None of the above, but in this section

26Exx Miscellaneous topics [See also 58Cxx]

26E05 Real--analytic functions [See also 32B05, 32C05]

26E10 C1--functions, quasi--analytic functions [See also 58C25]

26E15 Calculus of functions on infinite--dimensional spaces [See also 46G05, 58Cxx]

26E20 Calculus of functions taking values in infinite--dimensional spaces [See also 46E40, 46G10, 58Cxx]

26E25 Set-valued functions [See also 28B20, 49J53, 54C60] --For nonsmooth analysis, see 49J52, 58Cxx, 90Cxxg

26E30 Non--Archimedean analysis [See also 12J25]

26E35 Nonstandard analysis [See also 03H05, 28E05, 54J05]

26E40 Constructive real analysis [See also 03F60]

26E50 Fuzzy real analysis [See also 03E72, 28E10]

26E60 Means [See also 47A64]

26E70 Real analysis on time scales or measure chains --For dynamic equations on time scales or measure chains see 34N05g

26E99 None of the above, but in this section

\subsection{MEASURE AND INTEGRATION}

28-XX MEASURE AND INTEGRATION 

For analysis on manifolds, see 58-XXg

28-00 General reference works (handbooks, dictionaries, bibliographies,etc.)

28-01 Instructional exposition (textbooks, tutorial papers, etc.)

28-02 Research exposition (monographs, survey articles)

28-06 Proceedings, conferences, collections, etc.

28Axx Classical measure theory

28A05 Classes of sets (Borel fields, B--rings, etc.), measurable sets, Suslin sets, analytic sets [See also 03E15, 26A21, 54H05]

28A10 Real- or complex- valued set functions

28A12 Contents, measures, outer measures, capacities

28A15 Abstract differentiation theory, differentiation of set functions [See also 26A24]

28A20 Measurable and nonmeasurable functions, sequences of measurable functions, modes of convergence

28A25 Integration with respect to measures and other set functions

28A33 Spaces of measures, convergence of measures [See also 46E27, 60Bxx]

28A35 Measures and integrals in product spaces

28A50 Integration and disintegration of measures

28A51 Lifting theory [See also 46G15]

28A60 Measures on Boolean rings, measure algebras [See also 54H10]

28A75 Length, area, volume, other geometric measure theory [See also 26B15, 49Q15]

28A78 Hausdorff and packing measures

28A80 Fractals [See also 37Fxx]

\subsection{FUNCTIONS OF A COMPLEX VARIABLE}

30-XX FUNCTIONS OF A COMPLEX VARIABLE 

For analysis on manifolds, see 58-XXg

30-00 General reference works (handbooks, dictionaries, bibliographies,etc.)

30-01 Instructional exposition (textbooks, tutorial papers, etc.)

30-02 Research exposition (monographs, survey articles)

30-06 Proceedings, conferences, collections, etc.

30Axx General properties

30A05 Monogenic properties of complex functions (including polygenic and areolar monogenic functions)

30A10 Inequalities in the complex domain

30A99 None of the above, but in this section

30Bxx Series expansions

30B10 Power series (including lacunary series)

30B20 Random power series

30B30 Boundary behavior of power series, over-convergence

30B40 Analytic continuation

30B50 Dirichlet series and other series expansions, exponential series[See also 11M41, 42{XX]

30B60 Completeness problems, closure of a system of functions

30B70 Continued fractions [See also 11A55, 40A15]

30B99 None of the above, but in this section

30Cxx Geometric function theory

30C10 Polynomials

30C15 Zeros of polynomials, rational functions, and other analytic functions (e.g. zeros of functions with bounded Dirichlet integral) For algebraic theory, see 12D10; for real methods, see 26C10g

30C20 Conformal mappings of special domains

30C25 Covering theorems in conformal mapping theory

30C30 Numerical methods in conformal mapping theory [See also 65E05]

30C35 General theory of conformal mappings

30C40 Kernel functions and applications

30C45 Special classes of univalent and multivalent functions (starlike, convex, bounded rotation, etc.)

30C50 Coe_cient problems for univalent and multivalent functions

30C55 General theory of univalent and multivalent functions

30C62 Quasiconformal mappings in the plane

30C65 Quasiconformal mappings in $R^n$, other generalizations

30C70 Extremal problems for conformal and quasiconformal mappings, variational methods

30C75 Extremal problems for conformal and quasiconformal mappings, other methods

30C80 Maximum principle; Schwarz's lemma, Lindel method of principle, analogues and generalizations; subordination

30C85 Capacity and harmonic measure in the complex plane [See also 31A15]

30C99 None of the above, but in this section

30Dxx Entire and meromorphic functions, and related topics

30D05 Functional equations in the complex domain, iteration and composition of analytic functions [See also 34Mxx, 37Fxx, 39{XX]

30D10 Representations of entire functions by series and integrals

30D15 Special classes of entire functions and growth estimates

30D20 Entire functions, general theory

30D30 Meromorphic functions, general theory

30D35 Distribution of values, Nevanlinna theory

30D40 Cluster sets, prime ends, boundary behavior

30D45 Bloch functions, normal functions, normal families

30D60 Quasi-analytic and other classes of functions

30D99 None of the above, but in this section

30Exx Miscellaneous topics of analysis in the complex domain

30E05 Moment problems, interpolation problems

30E10 Approximation in the complex domain

30E15 Asymptotic representations in the complex domain

30E20 Integration, integrals of Cauchy type, integral representations of analytic functions [See also 45Exx]

30E25 Boundary value problems [See also 45Exx]

30E99 None of the above, but in this section

30Fxx Riemann surfaces

30F10 Compact Riemann surfaces and uniformization [See also 14H15, 32G15]

30F15 Harmonic functions on Riemann surfaces

30F20 Classification theory of Riemann surfaces

30F25 Ideal boundary theory

30F30 Differentials on Riemann surfaces

30F35 Fuchsian groups and automorphic functions [See also 11Fxx, 20H10, 22E40, 32Gxx, 32Nxx]

30F40 Kleinian groups [See also 20H10]

30F45 Conformal metrics (hyperbolic, Poincar$\'e$, distance functions)

30F50 Klein surfaces

30F60 Teichmuller theory [See also 32G15]

30F99 None of the above, but in this section

30Gxx Generalized function theory

30G06 Non-Archimedean function theory [See also 12J25]; nonstandard function theory [See also 03H05]

30G12 Finely holomorphic functions and topological function theory

30G20 Generalizations of Bers or Vekua type (pseudoanalytic, $p$--analytic, etc.)

30G25 Discrete analytic functions

30G30 Other generalizations of analytic functions (including abstract--valued functions)

30G35 Functions of hypercomplex variables and generalized variables 30G99 None of the above, but in this section

30Hxx Spaces and algebras of analytic functions

30H05 Bounded analytic functions

30H10 Hardy spaces

30H15 Nevanlinna class and Smirnov class

30H20 Bergman spaces, Fock spaces

30H25 Besov spaces and $Q_p$-spaces

30H30 Bloch spaces

30H35 BMO--spaces

30H50 Algebras of analytic functions

30H80 Corona theorems

30H99 None of the above, but in this section

30Jxx Function theory on the disc

30J05 Inner functions

30J10 Blaschke products

30J15 Singular inner functions

30J99 None of the above, but in this section

30Kxx Universal holomorphic functions

30K05 Universal Taylor series

30K10 Universal Dirichlet series

30K15 Bounded universal functions

30K20 Compositional universality

30K99 None of the above, but in this section

30Lxx Analysis on metric spaces

30L05 Geometric embeddings of metric spaces

30L10 Quasiconformal mappings in metric spaces

35Q40 Partial differential equations

81Q05  Quantum theory: General mathematical topics and methods in quantum theory



%%%%%
%%%%%
\end{document}
