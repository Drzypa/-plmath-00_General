\documentclass[12pt]{article}
\usepackage{pmmeta}
\pmcanonicalname{TopicEntryOnAlgebra}
\pmcreated{2013-03-22 18:00:02}
\pmmodified{2013-03-22 18:00:02}
\pmowner{rspuzio}{6075}
\pmmodifier{rspuzio}{6075}
\pmtitle{topic entry on algebra}
\pmrecord{11}{40514}
\pmprivacy{1}
\pmauthor{rspuzio}{6075}
\pmtype{Topic}
\pmcomment{trigger rebuild}
\pmclassification{msc}{00A20}
\pmrelated{TopicEntryOnTheAlgebraicFoundationsOfMathematics}
\pmrelated{OverviewOfTheContentOfPlanetMath}

% this is the default PlanetMath preamble.  as your knowledge
% of TeX increases, you will probably want to edit this, but
% it should be fine as is for beginners.

% almost certainly you want these
\usepackage{amssymb}
\usepackage{amsmath}
\usepackage{amsfonts}

% used for TeXing text within eps files
%\usepackage{psfrag}
% need this for including graphics (\includegraphics)
%\usepackage{graphicx}
% for neatly defining theorems and propositions
%\usepackage{amsthm}
% making logically defined graphics
%%%\usepackage{xypic}

% there are many more packages, add them here as you need them

% define commands here

\begin{document}
The subject of algebra may be defined as the study of algebraic
systems, where an algebraic system consists of a set together
with a certain number of operations, which are functions (or
partial functions) on this set.  A prototypical example of an 
algebraic system is the ring of integers, which consists of the
set of integers, $\{ \ldots, -2, -1, 0, 1, 2, \ldots \}$ together
with the operations $+$ and $\times$.  

In addition to studying individual systems, algebraists consider
classes of systems defined by common properties.  For instance,
the example cited above is an example of a ring, which is an algebraic 
system with two operations which satisfy certain axioms, such as
distributivity of one operation over the other.

The reason for considering classes of systems 
is in order to save work by stating and proving theorems at the 
appropriate level of generality.  For instance, while the statement
that every integer equals the sum of four squares is specific to the
ring of integers (there are many rings in which this is not the case)
and its proof makes use of specific facts about integers, the proof
of the fact that the product of two sums of integers equals the sum
of all products of numbers appearing in the first sum by numbers
appearing in the second sum only involves the distributive law, so
an analogous theorem will hold for any ring.  Clearly, it is 
wasteful to restate the same theorem and its proof for every ring
so we state and prove it once as a theorem about rings, then apply
it to specific instances of rings.  

\begin{enumerate}
\item \PMlinkid{Concepts in abstract algebra}{ConceptsInAbstractAlgebra}
\item topics on group theory
\item topics on ring theory
\item topics on ideal theory
\item topics on field theory
\item topics on homological algebra
\item topics on category theory
\item algebraic k-theory
\item Special notations in algebra
\item Topics on polynomials
\item Topics on field extensions and Galois theory
\item Entries on finitely generated ideals
\item \PMlinkid{Topic entry}{2530} on linear algebra
\item \PMlinkid{Concepts in linear algebra}{5663}
\item Matrices of special form
\item \PMlinkname{Matrix decompositions}{MatrixFactorization}
\item Bibliography for group theory
\item topics on universal algebra
\end{enumerate}

%%%%%
%%%%%
\end{document}
