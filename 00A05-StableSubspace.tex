\documentclass[12pt]{article}
\usepackage{pmmeta}
\pmcanonicalname{StableSubspace}
\pmcreated{2013-03-22 17:56:52}
\pmmodified{2013-03-22 17:56:52}
\pmowner{lalberti}{18937}
\pmmodifier{lalberti}{18937}
\pmtitle{stable subspace}
\pmrecord{6}{40446}
\pmprivacy{1}
\pmauthor{lalberti}{18937}
\pmtype{Definition}
\pmcomment{trigger rebuild}
\pmclassification{msc}{00A05}
\pmsynonym{invariant subspace}{StableSubspace}
\pmsynonym{stable subset}{StableSubspace}
\pmsynonym{invariant subset}{StableSubspace}

% this is the default PlanetMath preamble.  as your knowledge
% of TeX increases, you will probably want to edit this, but
% it should be fine as is for beginners.

% almost certainly you want these
\usepackage{amssymb}
\usepackage{amsmath}
\usepackage{amsfonts}

% used for TeXing text within eps files
%\usepackage{psfrag}
% need this for including graphics (\includegraphics)
%\usepackage{graphicx}
% for neatly defining theorems and propositions
%\usepackage{amsthm}
% making logically defined graphics
%%%\usepackage{xypic}

% there are many more packages, add them here as you need them

% define commands here

\begin{document}
A subset $S$ of a larger set $T$ is said to a \emph{stable subset} for a function $f: T\to T$ \underline{iff} $f(S)\subset S$.\\
Alternative phrasings with the same meaning are:
\begin{itemize}
\item $f$ is an invariant subset for $f$
\item $f$ stabilizes $S$
\item $S$ is stable under (the action of) $f$
\item $S$ is invariant under (the action of) $f$
\item $S$ is left stable by/under $f$
\item $S$ is left invariant by/under $f$
\end{itemize}
%%%%%
%%%%%
\end{document}
