\documentclass[12pt]{article}
\usepackage{pmmeta}
\pmcanonicalname{SevensegmentDisplay}
\pmcreated{2013-03-22 17:06:51}
\pmmodified{2013-03-22 17:06:51}
\pmowner{PrimeFan}{13766}
\pmmodifier{PrimeFan}{13766}
\pmtitle{seven-segment display}
\pmrecord{5}{39414}
\pmprivacy{1}
\pmauthor{PrimeFan}{13766}
\pmtype{Definition}
\pmcomment{trigger rebuild}
\pmclassification{msc}{00A99}

\endmetadata

% this is the default PlanetMath preamble.  as your knowledge
% of TeX increases, you will probably want to edit this, but
% it should be fine as is for beginners.

% almost certainly you want these
\usepackage{amssymb}
\usepackage{amsmath}
\usepackage{amsfonts}

% used for TeXing text within eps files
%\usepackage{psfrag}
% need this for including graphics (\includegraphics)
\usepackage{graphicx}
% for neatly defining theorems and propositions
%\usepackage{amsthm}
% making logically defined graphics
%%%\usepackage{xypic}

% there are many more packages, add them here as you need them

% define commands here

\begin{document}
Most calculators use a {\em seven-segment display} to show both input and output numbers. Here is Champernowne's constant $C_{10}$ to 9 decimal places in a seven-segment display:

\begin{center}
\includegraphics{SevenSegmentDisplay}
\end{center}

Technically, the space for the decimal point counts as a segment, so for each place value there are actually eight segments. Generally, calculator manufacturers prefer the base 10 pandigital number 1234567980 as a demo number for the calculator packaging.

Sometimes wear and tear can cause a segment to not light up or darken properly; this can occasionally lead to confusion between digits. The simplest way to test for this is to input $8 \times \frac{10^x - 1}{9}$, where $x$ is the the number of decimal place values available on the display (that is, a bunch of 8s). At various times in the past some manufacturers have developed variants of the glyphs for the digits to make it more obvious when a segment has burnt out, such as a 0 that looks like a lowercase O, or a 6 that looks like a lowercase B. Given that many scientific calculators offer the option of doing calculations in hexadecimal, the latter option is not acceptable for such calculators. The 7 on Sharp-brand calculators uses the upper left vertical segment.
%%%%%
%%%%%
\end{document}
