\documentclass[12pt]{article}
\usepackage{pmmeta}
\pmcanonicalname{OpusNumber}
\pmcreated{2013-03-22 16:44:58}
\pmmodified{2013-03-22 16:44:58}
\pmowner{PrimeFan}{13766}
\pmmodifier{PrimeFan}{13766}
\pmtitle{opus number}
\pmrecord{4}{38974}
\pmprivacy{1}
\pmauthor{PrimeFan}{13766}
\pmtype{Definition}
\pmcomment{trigger rebuild}
\pmclassification{msc}{00A99}
\pmrelated{KochelNumber}

% this is the default PlanetMath preamble.  as your knowledge
% of TeX increases, you will probably want to edit this, but
% it should be fine as is for beginners.

% almost certainly you want these
\usepackage{amssymb}
\usepackage{amsmath}
\usepackage{amsfonts}

% used for TeXing text within eps files
%\usepackage{psfrag}
% need this for including graphics (\includegraphics)
%\usepackage{graphicx}
% for neatly defining theorems and propositions
%\usepackage{amsthm}
% making logically defined graphics
%%%\usepackage{xypic}

% there are many more packages, add them here as you need them

% define commands here

\begin{document}
An {\em opus number} is a number attached to a particular musical composition or set of compositions by a publisher to denote the order of publication by the given composer. 
Theoretically, opus numbers would give musicologists a simple means to chart the growth of a composer chronologically, to make statements to the effect that there is a direct correlation between such and such composer's handling of a particular musical technique and the opus numbers.

In practice, however, many factors contribute to making the opus numbers for certain composers less than useful, such as their being assigned by the composer rather than the publisher. Anton\'in Dvo\v{r}\'ak, for example, tried to fool publishers in an effort to get the best deal, and thus often gave misleadingly low or high opus numbers to his compositions. The confusion with Carl Nielsen's opus numbers arose not from any wheeling and dealing, but from his not having an exclusive publisher and not keeping track of which opus numbers he hadn't used (see sequence A113529 in Sloane's OEIS for a listing of opus numbers skipped over by Carl Nielsen).

Other composers just never had any opus numbers given to their compositions by either publishers or themselves; musicologists then sometimes try to create a catalog with numbers named after themselves. In the case of Mozart, the K\"ochel numbers by Ludwig von K\"ochel are almost universally accepted, but in the case of Domenico Scarlatti, three different musicologists came up with three different catalogs, leading to many of Scarlatti's sonatas to be known by three different numbers. The lack of opus numbers is not a problem for composers whose repertory works consist of less than a dozen of large-scale works (e.g., Bruckner and Mahler).

\begin{thebibliography}{1}
\bibitem{hs} Sch\"onzeler, H. H. {\it Dvo\v{r}\'ak} London: M. Boyers (1984): 220 - 239
\end{thebibliography}
%%%%%
%%%%%
\end{document}
