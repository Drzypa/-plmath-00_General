\documentclass[12pt]{article}
\usepackage{pmmeta}
\pmcanonicalname{BlackboardBold}
\pmcreated{2013-03-22 16:45:35}
\pmmodified{2013-03-22 16:45:35}
\pmowner{PrimeFan}{13766}
\pmmodifier{PrimeFan}{13766}
\pmtitle{blackboard bold}
\pmrecord{5}{38988}
\pmprivacy{1}
\pmauthor{PrimeFan}{13766}
\pmtype{Definition}
\pmcomment{trigger rebuild}
\pmclassification{msc}{00A99}

\endmetadata

% this is the default PlanetMath preamble.  as your knowledge
% of TeX increases, you will probably want to edit this, but
% it should be fine as is for beginners.

% almost certainly you want these
\usepackage{amssymb}
\usepackage{amsmath}
\usepackage{amsfonts}

% used for TeXing text within eps files
%\usepackage{psfrag}
% need this for including graphics (\includegraphics)
%\usepackage{graphicx}
% for neatly defining theorems and propositions
%\usepackage{amsthm}
% making logically defined graphics
%%%\usepackage{xypic}

% there are many more packages, add them here as you need them

% define commands here

\begin{document}
Certain capital letters are sometimes written in {\em blackboard bold} and considered to be a distinct symbol with a distinct meaning from their regular counterparts. This is to help alleviate overloading of the letters of the Latin-1 alphabet in mathematics and physics. For example, the blackboard letter $\mathbb{C}$ signifies the complex numbers, while in the same set of formulas one could also use $C$ as some variable or constant, complex or not, confident that it would not be confused with the complex numbers in general.

Mathematicians are not unanimous in their approval of blackboard bold, however, and Donald Knuth did not include support for it in \TeX{}. The American Mathematical Society created the \TeX{} package \verb=amsfonts= which PlanetMath document preambles contain by default. The entire Latin-1 alphabet is available thus, while Unicode in Letterlike Symbols only makes available a subset of the most commonly used (such as $\mathbb{Q}$ and $\mathbb{Z}$).

In typewritten dissertations, and even some published books, simple bold of non-italics letters is used.

%%%%%
%%%%%
\end{document}
