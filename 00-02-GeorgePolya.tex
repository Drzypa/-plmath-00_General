\documentclass[12pt]{article}
\usepackage{pmmeta}
\pmcanonicalname{GeorgePolya}
\pmcreated{2013-03-22 18:26:56}
\pmmodified{2013-03-22 18:26:56}
\pmowner{bci1}{20947}
\pmmodifier{bci1}{20947}
\pmtitle{George P\'olya}
\pmrecord{12}{41110}
\pmprivacy{1}
\pmauthor{bci1}{20947}
\pmtype{Biography}
\pmcomment{trigger rebuild}
\pmclassification{msc}{00-02}
\pmclassification{msc}{01A70}

% this is the default PlanetMath preamble.  as your knowledge
% of TeX increases, you will probably want to edit this, but
% it should be fine as is for beginners.

% almost certainly you want these
\usepackage{amssymb}
\usepackage{amsmath}
\usepackage{amsfonts}

% used for TeXing text within eps files
%\usepackage{psfrag}
% need this for including graphics (\includegraphics)
%\usepackage{graphicx}
% for neatly defining theorems and propositions
%\usepackage{amsthm}
% making logically defined graphics
%%%\usepackage{xypic}

% there are many more packages, add them here as you need them

% define commands here
\usepackage{amsmath, amssymb, amsfonts, amsthm, amscd, latexsym}
%%\usepackage{xypic}
\usepackage[mathscr]{eucal}

\setlength{\textwidth}{6.5in}
%\setlength{\textwidth}{16cm}
\setlength{\textheight}{9.0in}
%\setlength{\textheight}{24cm}

\hoffset=-.75in     %%ps format
%\hoffset=-1.0in     %%hp format
\voffset=-.4in

\theoremstyle{plain}
\newtheorem{lemma}{Lemma}[section]
\newtheorem{proposition}{Proposition}[section]
\newtheorem{theorem}{Theorem}[section]
\newtheorem{corollary}{Corollary}[section]

\theoremstyle{definition}
\newtheorem{definition}{Definition}[section]
\newtheorem{example}{Example}[section]
%\theoremstyle{remark}
\newtheorem{remark}{Remark}[section]
\newtheorem*{notation}{Notation}
\newtheorem*{claim}{Claim}

\renewcommand{\thefootnote}{\ensuremath{\fnsymbol{footnote%%@
}}}
\numberwithin{equation}{section}

\newcommand{\Ad}{{\rm Ad}}
\newcommand{\Aut}{{\rm Aut}}
\newcommand{\Cl}{{\rm Cl}}
\newcommand{\Co}{{\rm Co}}
\newcommand{\DES}{{\rm DES}}
\newcommand{\Diff}{{\rm Diff}}
\newcommand{\Dom}{{\rm Dom}}
\newcommand{\Hol}{{\rm Hol}}
\newcommand{\Mon}{{\rm Mon}}
\newcommand{\Hom}{{\rm Hom}}
\newcommand{\Ker}{{\rm Ker}}
\newcommand{\Ind}{{\rm Ind}}
\newcommand{\IM}{{\rm Im}}
\newcommand{\Is}{{\rm Is}}
\newcommand{\ID}{{\rm id}}
\newcommand{\GL}{{\rm GL}}
\newcommand{\Iso}{{\rm Iso}}
\newcommand{\Sem}{{\rm Sem}}
\newcommand{\St}{{\rm St}}
\newcommand{\Sym}{{\rm Sym}}
\newcommand{\SU}{{\rm SU}}
\newcommand{\Tor}{{\rm Tor}}
\newcommand{\U}{{\rm U}}

\newcommand{\A}{\mathcal A}
\newcommand{\Ce}{\mathcal C}
\newcommand{\D}{\mathcal D}
\newcommand{\E}{\mathcal E}
\newcommand{\F}{\mathcal F}
\newcommand{\G}{\mathcal G}
\newcommand{\Q}{\mathcal Q}
\newcommand{\R}{\mathcal R}
\newcommand{\cS}{\mathcal S}
\newcommand{\cU}{\mathcal U}
\newcommand{\W}{\mathcal W}

\newcommand{\bA}{\mathbb{A}}
\newcommand{\bB}{\mathbb{B}}
\newcommand{\bC}{\mathbb{C}}
\newcommand{\bD}{\mathbb{D}}
\newcommand{\bE}{\mathbb{E}}
\newcommand{\bF}{\mathbb{F}}
\newcommand{\bG}{\mathbb{G}}
\newcommand{\bK}{\mathbb{K}}
\newcommand{\bM}{\mathbb{M}}
\newcommand{\bN}{\mathbb{N}}
\newcommand{\bO}{\mathbb{O}}
\newcommand{\bP}{\mathbb{P}}
\newcommand{\bR}{\mathbb{R}}
\newcommand{\bV}{\mathbb{V}}
\newcommand{\bZ}{\mathbb{Z}}

\newcommand{\bfE}{\mathbf{E}}
\newcommand{\bfX}{\mathbf{X}}
\newcommand{\bfY}{\mathbf{Y}}
\newcommand{\bfZ}{\mathbf{Z}}

\renewcommand{\O}{\Omega}
\renewcommand{\o}{\omega}
\newcommand{\vp}{\varphi}
\newcommand{\vep}{\varepsilon}

\newcommand{\diag}{{\rm diag}}
\newcommand{\grp}{{\mathbb G}}
\newcommand{\dgrp}{{\mathbb D}}
\newcommand{\desp}{{\mathbb D^{\rm{es}}}}
\newcommand{\Geod}{{\rm Geod}}
\newcommand{\geod}{{\rm geod}}
\newcommand{\hgr}{{\mathbb H}}
\newcommand{\mgr}{{\mathbb M}}
\newcommand{\ob}{{\rm Ob}}
\newcommand{\obg}{{\rm Ob(\mathbb G)}}
\newcommand{\obgp}{{\rm Ob(\mathbb G')}}
\newcommand{\obh}{{\rm Ob(\mathbb H)}}
\newcommand{\Osmooth}{{\Omega^{\infty}(X,*)}}
\newcommand{\ghomotop}{{\rho_2^{\square}}}
\newcommand{\gcalp}{{\mathbb G(\mathcal P)}}

\newcommand{\rf}{{R_{\mathcal F}}}
\newcommand{\glob}{{\rm glob}}
\newcommand{\loc}{{\rm loc}}
\newcommand{\TOP}{{\rm TOP}}

\newcommand{\wti}{\widetilde}
\newcommand{\what}{\widehat}

\renewcommand{\a}{\alpha}
\newcommand{\be}{\beta}
\newcommand{\ga}{\gamma}
\newcommand{\Ga}{\Gamma}
\newcommand{\de}{\delta}
\newcommand{\del}{\partial}
\newcommand{\ka}{\kappa}
\newcommand{\si}{\sigma}
\newcommand{\ta}{\tau}
\newcommand{\med}{\medbreak}
\newcommand{\medn}{\medbreak \noindent}
\newcommand{\bign}{\bigbreak \noindent}
\newcommand{\lra}{{\longrightarrow}}
\newcommand{\ra}{{\rightarrow}}
\newcommand{\rat}{{\rightarrowtail}}
\newcommand{\oset}[1]{\overset {#1}{\ra}}
\newcommand{\osetl}[1]{\overset {#1}{\lra}}
\newcommand{\hr}{{\hookrightarrow}}
\begin{document}
\section{George P\'olya} 
 American mathematician, 
 Born: Gy\"orgy P\'olya in Budapest, Hungary in 1887, ({\em d.} 1985 in Palo Alto, USA)

 An excellent problem solver. He designed a complete strategy for problem solving
that can help both the beginner and the advanced mathematician to solve both mathematical
and physical problems. 

 ``{\em His first job was to tutor the young son, Gregor, of a Hungarian baron. Gregor struggled due to his lack of problem solving skills.}'' Thus, according to Long (\cite{LTDW1996}), Polya  insisted that the skill of ``{\em solving problems was not an inborn quality but, something that could be taught}''.

 In 1940, George Polya and his wife, Stella, (the only daughter of Swiss Dr. Weber, in Zurich) moved to the United States because of their justified fear of Nazism in Germany (\cite{LTDW1996}). 

 He taught at first, at Brown University, and then he moved permanently with his wife 
to Stanford University. Became Professor Emeritus at Stanford in 1953. He also taught many classes to elementary and secondary classroom teachers, inspiring them how to motivate and teach their students how to solve problems.
His research was in several mathematical areas: functional analysis, probability, number theory, algebra, combinatorics and geometry.
Recieved The Mathematical Association of America Award ''for articles of expository excellence published in the College Mathematics Journal''.
He published in 1945 the book ``{\em How to Solve It}'' that sold in more than one million copies in 18 languages.  Although an appropriate strategy can be learned by solving many problems, it is learned much faster if several, similar examples are worked out first with a teacher on an one--on--one basis. Here are some of the highlights of his simple strategies for problem solving:

\begin{enumerate}
\item \textbf{Understand the Problem}
\item Devise a Plan on how to approach the Problem; such a plan may include one or several of the following:
\item Make a first guess to begin with, and then verify the answer
\item Solve a simpler problem
\item Consider special cases that are much easier to solve
\item Look for a pattern
\item Draw a picture
\item Use a model  
\item Use direct reasoning but double-check your results
\item Use a formula that you fully understand and have used before
\item Eliminate possibilities
\item Carry out the Plan, as modified by partial solutions
\item If plan doesn't work, make an improved plan but do not give up
\item Last-but-not-least, look back and examine critically your solution(s):
\item Does the solution make sense? Does it check out in particular cases?
\item Make sure there are no gaps and no steps missing.
\end{enumerate}

He published also a two-volume book, ``{\em Mathematics and Plausible Reasoning}'' in 1954, and 
{\em Mathematical Discovery} in 1962.


\begin{thebibliography}{9}
\bibitem{LTDW1996}
Long, C. T., \& DeTemple, D. W., {\em Mathematical reasoning for elementary teachers}. (1996). Reading MA: Addison-Wesley 

\bibitem{}
Reimer, L., \& Reimer, W. {\em Mathematicians are people too}. (Volume 2). (1995) Dale Seymour Publications 

\bibitem{PG57}
Polya, G. {\em How to solve it}. (1957) Garden City, NY: Doubleday and Co., Inc.

\bibitem{AM}
A. Motter,, \PMlinkexternal{``A Biography of George Polya''}{http://www.math.wichita.edu/history/men/polya.html}


\end{thebibliography}
%%%%%
%%%%%
\end{document}
