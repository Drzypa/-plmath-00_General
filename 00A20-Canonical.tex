\documentclass[12pt]{article}
\usepackage{pmmeta}
\pmcanonicalname{Canonical}
\pmcreated{2013-03-22 14:44:32}
\pmmodified{2013-03-22 14:44:32}
\pmowner{mathcam}{2727}
\pmmodifier{mathcam}{2727}
\pmtitle{canonical}
\pmrecord{6}{36379}
\pmprivacy{1}
\pmauthor{mathcam}{2727}
\pmtype{Definition}
\pmcomment{trigger rebuild}
\pmclassification{msc}{00A20}
\pmrelated{CanonicalFormOfElementOfNumberField}

% this is the default PlanetMath preamble.  as your knowledge
% of TeX increases, you will probably want to edit this, but
% it should be fine as is for beginners.

% almost certainly you want these
\usepackage{amssymb}
\usepackage{amsmath}
\usepackage{amsfonts}
\usepackage{amsthm}

\usepackage{mathrsfs}

% used for TeXing text within eps files
%\usepackage{psfrag}
% need this for including graphics (\includegraphics)
%\usepackage{graphicx}
% for neatly defining theorems and propositions
%
% making logically defined graphics
%%%\usepackage{xypic}

% there are many more packages, add them here as you need them

% define commands here

\newcommand{\sR}[0]{\mathbb{R}}
\newcommand{\sC}[0]{\mathbb{C}}
\newcommand{\sN}[0]{\mathbb{N}}
\newcommand{\sZ}[0]{\mathbb{Z}}

 \usepackage{bbm}
 \newcommand{\Z}{\mathbbmss{Z}}
 \newcommand{\C}{\mathbbmss{C}}
 \newcommand{\R}{\mathbbmss{R}}
 \newcommand{\Q}{\mathbbmss{Q}}



\newcommand*{\norm}[1]{\lVert #1 \rVert}
\newcommand*{\abs}[1]{| #1 |}



\newtheorem{thm}{Theorem}
\newtheorem{defn}{Definition}
\newtheorem{prop}{Proposition}
\newtheorem{lemma}{Lemma}
\newtheorem{cor}{Corollary}
\begin{document}
A mathematical object is said to be \emph{canonical}
if it arises in a natural way without introducing any additional objects. 

\subsubsection*{Examples}
\begin{enumerate}
\item Suppose $A\times B$ is the Cartesian product of sets $A,B$. 
Then $A\times B$ has two \PMlinkescapetext{canonical projections} 
$A\times B\to A$
and $A\times B\to B$ defined in a natural way. Of course, if
we assume more structure of $A,B$ there are also other projections. 
\item \PMlinkname{canonical projection}{CanonicalProjection} (in group theory)
\end{enumerate}

\subsubsection*{Notes}
For a discussion of the theological use of canonical, see \cite{wiki}. 

\begin{thebibliography}{9}
\bibitem{wiki} Wikipedia, article on \PMlinkexternal{canonical}{http://en.wikipedia.org/wiki/Canonical}.
\end{thebibliography}
%%%%%
%%%%%
\end{document}
