\documentclass[12pt]{article}
\usepackage{pmmeta}
\pmcanonicalname{SurrealNumber}
\pmcreated{2013-03-22 12:58:49}
\pmmodified{2013-03-22 12:58:49}
\pmowner{Henry}{455}
\pmmodifier{Henry}{455}
\pmtitle{surreal number}
\pmrecord{9}{33352}
\pmprivacy{1}
\pmauthor{Henry}{455}
\pmtype{Definition}
\pmcomment{trigger rebuild}
\pmclassification{msc}{00A05}
\pmdefines{omnific integers}

\endmetadata

% this is the default PlanetMath preamble.  as your knowledge
% of TeX increases, you will probably want to edit this, but
% it should be fine as is for beginners.

% almost certainly you want these
\usepackage{amssymb}
\usepackage{amsmath}
\usepackage{amsfonts}

% used for TeXing text within eps files
%\usepackage{psfrag}
% need this for including graphics (\includegraphics)
%\usepackage{graphicx}
% for neatly defining theorems and propositions
%\usepackage{amsthm}
% making logically defined graphics
%%%\usepackage{xypic}

% there are many more packages, add them here as you need them

% define commands here
%\PMlinkescapeword{theory}
\begin{document}
The surreal numbers are a generalization of the reals.  Each surreal number consists of two parts (called the left and right), each of which is a set of surreal numbers.  For any surreal number $N$, these parts can be called $N_L$ and $N_R$.  (This could be viewed as an ordered pair of sets, however the surreal numbers were intended to be a basis for mathematics, not something to be embedded in set theory.)  A surreal number is written $N=\langle N_L\mid N_R\rangle$.

Not every number of this form is a surreal number.  The surreal numbers satisfy two additional properties.  First, if $x\in N_R$ and $y\in N_L$ then $x\nleq y$.  Secondly, they must be well founded.  These properties are both satisfied by the following construction of the surreal numbers and the $\leq$ relation by mutual induction:

$\langle\mid\rangle$, which has both left and right parts empty, is $0$.

Given two (possibly empty) sets of surreal numbers $R$ and $L$ such that for any $x\in R$ and $y\in L$, $x\nleq y$, $\langle L\mid R\rangle$.

Define $N\leq M$ if there is no $x\in N_L$ such that $M\leq x$ and no $y\in M_R$ such that $y\leq N$.

This process can be continued transfinitely, to define infinite and infinitesimal numbers.  For instance if $\mathbb{Z}$ is the set of integers then $\omega=\langle \mathbb{Z}\mid \rangle$.  Note that this does not make equality the same as identity: $\langle 1\mid 1\rangle=\langle \mid\rangle$, for instance.

It can be shown that $N$ is ``sandwiched'' between the elements of $N_L$ and $N_R$: it is larger than any element of $N_L$ and smaller than any element of $N_R$.

Addition of surreal numbers is defined by 

$$N+M=\langle \{N+x\mid x\in M_L\}\cup\{M+x\mid y\in N_L\}\mid \{N+x\mid x\in M_R\}\cup\{M+x\mid y\in N_R\}\rangle$$

It follows that $-N=\langle -N_R\mid -N_L\rangle$.

The definition of multiplication can be written more easily by defining $M\cdot  N_L=\{M\cdot x\mid x\in N_L\}$ and similarly for $N_R$.

Then

\begin{align*}
N\cdot M=&\langle M\cdot N_L+N\cdot M_L-N_L\cdot M_L,M\cdot N_R+N\cdot M_R-N_R\cdot M_R\mid \\
&M\cdot N_L+N\cdot M_R-N_L\cdot M_R,M\cdot N_R+N\cdot M_L-N_R\cdot M_L\rangle
\end{align*}

The surreal numbers satisfy the axioms for a field under addition and multiplication (whether they really are a field is complicated by the fact that they are too large to be a set).

The integers of surreal mathematics are called the \emph{omnific integers}.  In general positive integers $n$ can always be written $\langle n-1\mid\rangle$ and so $-n=\langle \mid 1-n\rangle=\langle \mid (-n)+1\rangle$.  So for instance $1=\langle 0\mid\rangle$.

In general, $\langle a\mid b\rangle$ is the simplest number between $a$ and $b$.  This can be easily used to define the dyadic fractions: for any integer $a$, $a+\frac{1}{2}=\langle a\mid a+1\rangle$.  Then $\frac{1}{2}=\langle 0\mid 1\rangle$, $\frac{1}{4}=\langle 0\mid \frac{1}{2}\rangle$, and so on.  This can then be used to locate non-dyadic fractions by pinning them between a left part which gets infinitely close from below and a right part which gets infinitely close from above.

Ordinal arithmetic can be defined starting with $\omega$ as defined above and adding numbers such as $\langle \omega\mid\rangle=\omega+1$ and so on.  Similarly, a starting infinitesimal can be found as $\langle 0\mid 1,\frac{1}{2},\frac{1}{4}\ldots\rangle=\frac{1}{\omega}$, and again more can be developed from there.
%%%%%
%%%%%
\end{document}
