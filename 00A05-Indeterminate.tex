\documentclass[12pt]{article}
\usepackage{pmmeta}
\pmcanonicalname{Indeterminate}
\pmcreated{2013-03-22 14:47:33}
\pmmodified{2013-03-22 14:47:33}
\pmowner{CWoo}{3771}
\pmmodifier{CWoo}{3771}
\pmtitle{indeterminate}
\pmrecord{5}{36444}
\pmprivacy{1}
\pmauthor{CWoo}{3771}
\pmtype{Definition}
\pmcomment{trigger rebuild}
\pmclassification{msc}{00A05}
\pmrelated{Parameter}

\endmetadata

% this is the default PlanetMath preamble.  as your knowledge
% of TeX increases, you will probably want to edit this, but
% it should be fine as is for beginners.

% almost certainly you want these
\usepackage{amssymb,amscd}
\usepackage{amsmath}
\usepackage{amsfonts}

% used for TeXing text within eps files
%\usepackage{psfrag}
% need this for including graphics (\includegraphics)
%\usepackage{graphicx}
% for neatly defining theorems and propositions
%\usepackage{amsthm}
% making logically defined graphics
%%%\usepackage{xypic}

% there are many more packages, add them here as you need them

% define commands here
\begin{document}
\PMlinkescapeword{solvable}
\PMlinkescapeword{alphabet}

An {\em indeterminate}\, is simply a variable that is not known or solvable. \,It is usually denoted by a mathematical alphabet ($x$, $y$, $z$, or $\alpha$, $\beta$, etc...). \,It is important to distinguish between a variable and an indeterminate in that a variable is solvable, at least conditionally. \,To make this more precise, let's see two examples:

\begin{enumerate}
\item Let $x$ be a variable such that \,$2+3x=a+bx$, where $a,b\in\mathbb{Q}$.  Then \,$x=(a-2)/(3-b)$. \,Here $x$ is solvable conditioned on the equation given.  Any values of $a$ and $b\,(\neq 3)$ will yield a value for $x$.
\item Let $x$ be an indeterminate such that \,$2+3x=a+bx$, where \,$a,\,b\in\mathbb{Q}$. \,Since $x$ can not be solved, we have \,$2=a$\, and \,$3=b$. \,Note that if $a$ and $b$ are previously assigned to be values other than 2 and 3 respectively, then $x$ is no longer an indeterminate.
\end{enumerate}
%%%%%
%%%%%
\end{document}
