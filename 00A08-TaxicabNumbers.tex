\documentclass[12pt]{article}
\usepackage{pmmeta}
\pmcanonicalname{TaxicabNumbers}
\pmcreated{2013-03-22 15:43:00}
\pmmodified{2013-03-22 15:43:00}
\pmowner{alozano}{2414}
\pmmodifier{alozano}{2414}
\pmtitle{taxicab numbers}
\pmrecord{6}{37664}
\pmprivacy{1}
\pmauthor{alozano}{2414}
\pmtype{Feature}
\pmcomment{trigger rebuild}
\pmclassification{msc}{00A08}

% this is the default PlanetMath preamble.  as your knowledge
% of TeX increases, you will probably want to edit this, but
% it should be fine as is for beginners.

% almost certainly you want these
\usepackage{amssymb}
\usepackage{amsmath}
\usepackage{amsthm}
\usepackage{amsfonts}

% used for TeXing text within eps files
%\usepackage{psfrag}
% need this for including graphics (\includegraphics)
%\usepackage{graphicx}
% for neatly defining theorems and propositions
%\usepackage{amsthm}
% making logically defined graphics
%%%\usepackage{xypic}

% there are many more packages, add them here as you need them

% define commands here

\newtheorem{thm}{Theorem}
\newtheorem{defn}{Definition}
\newtheorem{prop}{Proposition}
\newtheorem{lemma}{Lemma}
\newtheorem{cor}{Corollary}

\theoremstyle{definition}
\newtheorem{exa}{Example}

% Some sets
\newcommand{\Nats}{\mathbb{N}}
\newcommand{\Ints}{\mathbb{Z}}
\newcommand{\Reals}{\mathbb{R}}
\newcommand{\Complex}{\mathbb{C}}
\newcommand{\Rats}{\mathbb{Q}}
\newcommand{\Gal}{\operatorname{Gal}}
\newcommand{\Cl}{\operatorname{Cl}}
\begin{document}
The number $1729$ has a reputation of its own. The reason is the famous exchange between \PMlinkexternal{G. H. Hardy}{http://www-groups.dcs.st-and.ac.uk/~history/Mathematicians/Hardy.html}, a famous British mathematician (1877-1947), and \PMlinkexternal{Srinivasa Ramanujan}{http://www-groups.dcs.st-and.ac.uk/~history/Mathematicians/Ramanujan.html} , one of India's greatest mathematical geniuses (1887-1920):

\begin{quote}
In 1917, during one visit to Ramanujan in a hospital (he was ill for much of his last three years), Hardy mentioned that the number of the taxi cab
that had brought him was $1729$, which, as numbers go, Hardy thought 
was ``rather a dull number''.  At this, Ramanujan perked up, and said
``No, it is a very interesting number; it is the smallest number
expressible as a sum of two cubes in two different ways.''
\end{quote}

Indeed:
$$1729=1+12^3=9^3+10^3.$$
Moreover, there are other reasons why $1729$ is far from dull. $1729$ is the third Carmichael number. Even more strange, beginning at the 
$1729$th decimal digit of the transcental number $e$, the next ten 
successive digits of $e$ are 0719425863. This is the first appearance
of all ten digits in a row without repititions.

More generally, the smallest natural number which can be expressed as the sum of $n$ positive cubes is called the $n$th taxicab number. The first taxicab numbers are:
$$2=1^3+1^3,\ 1729 =1^3+12^3=9^3+10^3,\ 87539319=167^3+436^3=228^3+423^3=255^3+414^3$$
followed by $6963472309248$ (found by E. Rosenstiel, J.A. Dardis, and C.R. Rosenstiel in 1991) and $48988659276962496$ (found by David Wilson on November 21st, 1997).
%%%%%
%%%%%
\end{document}
