\documentclass[12pt]{article}
\usepackage{pmmeta}
\pmcanonicalname{MSCClassificationOfObjectsArticlesSearch}
\pmcreated{2013-03-22 19:22:32}
\pmmodified{2013-03-22 19:22:32}
\pmowner{bci1}{20947}
\pmmodifier{bci1}{20947}
\pmtitle{MSC classification of objects: articles search }
\pmrecord{24}{42329}
\pmprivacy{1}
\pmauthor{bci1}{20947}
\pmtype{Bibliography}
\pmcomment{trigger rebuild}
\pmclassification{msc}{00-02}
\pmclassification{msc}{00-01}
%\pmkeywords{AMS classification}
%\pmkeywords{general applied mathematics}
%\pmkeywords{algebraic geometry}
%\pmkeywords{number theory}
%\pmkeywords{functional analysis}
%\pmkeywords{harmonic analysis}
%\pmkeywords{algebraic logics}
%\pmkeywords{physics}
%\pmkeywords{general mathematics}
%\pmkeywords{mathematical physics}
%\pmkeywords{dimensional analysis}
%\pmkeywords{geometry}
%\pmkeywords{algebra}
%\pmkeywords{topology}
%\pmkeywords{algebraic}
\pmdefines{AMS classification for: general applied mathematics}
\pmdefines{algebraic geometry}
\pmdefines{number theory}
\pmdefines{functional analysis}
\pmdefines{harmonic analysis}
\pmdefines{algebraic logics}
\pmdefines{physics}
\pmdefines{general mathematics}
\pmdefines{mathematical physics}
\pmdefines{dimensional analysis}
\pmdefines{geometry}
\pmdefines{algebra}
\pmdefines{topology}
\pmdefines{algebr}

\endmetadata

% this is the default PlanetMath preamble. as your knowledge
% of TeX increases, you will probably want to edit this, but
\usepackage{amsmath, amssymb, amsfonts, amsthm, amscd, latexsym}
%%\usepackage{xypic}
\usepackage[mathscr]{eucal}
% define commands here
\theoremstyle{plain}
\newtheorem{lemma}{Lemma}[section]
\newtheorem{proposition}{Proposition}[section]
\newtheorem{theorem}{Theorem}[section]
\newtheorem{corollary}{Corollary}[section]
\theoremstyle{definition}
\newtheorem{definition}{Definition}[section]
\newtheorem{example}{Example}[section]
%\theoremstyle{remark}
\newtheorem{remark}{Remark}[section]
\newtheorem*{notation}{Notation}
\newtheorem*{claim}{Claim}
\renewcommand{\thefootnote}{\ensuremath{\fnsymbol{footnote%%@
}}}
\numberwithin{equation}{section}
\newcommand{\Ad}{{\rm Ad}}
\newcommand{\Aut}{{\rm Aut}}
\newcommand{\Cl}{{\rm Cl}}
\newcommand{\Co}{{\rm Co}}
\newcommand{\DES}{{\rm DES}}
\newcommand{\Diff}{{\rm Diff}}
\newcommand{\Dom}{{\rm Dom}}
\newcommand{\Hol}{{\rm Hol}}
\newcommand{\Mon}{{\rm Mon}}
\newcommand{\Hom}{{\rm Hom}}
\newcommand{\Ker}{{\rm Ker}}
\newcommand{\Ind}{{\rm Ind}}
\newcommand{\IM}{{\rm Im}}
\newcommand{\Is}{{\rm Is}}
\newcommand{\ID}{{\rm id}}
\newcommand{\GL}{{\rm GL}}
\newcommand{\Iso}{{\rm Iso}}
\newcommand{\Sem}{{\rm Sem}}
\newcommand{\St}{{\rm St}}
\newcommand{\Sym}{{\rm Sym}}
\newcommand{\SU}{{\rm SU}}
\newcommand{\Tor}{{\rm Tor}}
\newcommand{\U}{{\rm U}}
\newcommand{\A}{\mathcal A}
\newcommand{\Ce}{\mathcal C}
\newcommand{\D}{\mathcal D}
\newcommand{\E}{\mathcal E}
\newcommand{\F}{\mathcal F}
\newcommand{\G}{\mathcal G}
\newcommand{\Q}{\mathcal Q}
\newcommand{\R}{\mathcal R}
\newcommand{\cS}{\mathcal S}
\newcommand{\cU}{\mathcal U}
\newcommand{\W}{\mathcal W}
\newcommand{\bA}{\mathbb{A}}
\newcommand{\bB}{\mathbb{B}}
\newcommand{\bC}{\mathbb{C}}
\newcommand{\bD}{\mathbb{D}}
\newcommand{\bE}{\mathbb{E}}
\newcommand{\bF}{\mathbb{F}}
\newcommand{\bG}{\mathbb{G}}
\newcommand{\bK}{\mathbb{K}}
\newcommand{\bM}{\mathbb{M}}
\newcommand{\bN}{\mathbb{N}}
\newcommand{\bO}{\mathbb{O}}
\newcommand{\bP}{\mathbb{P}}
\newcommand{\bR}{\mathbb{R}}
\newcommand{\bV}{\mathbb{V}}
\newcommand{\bZ}{\mathbb{Z}}
\newcommand{\bfE}{\mathbf{E}}
\newcommand{\bfX}{\mathbf{X}}
\newcommand{\bfY}{\mathbf{Y}}
\newcommand{\bfZ}{\mathbf{Z}}
\renewcommand{\O}{\Omega}
\renewcommand{\o}{\omega}
\newcommand{\vp}{\varphi}
\newcommand{\vep}{\varepsilon}
\newcommand{\diag}{{\rm diag}}
\newcommand{\grp}{{\mathbb G}}
\newcommand{\dgrp}{{\mathbb D}}
\newcommand{\desp}{{\mathbb D^{\rm{es}}}}
\newcommand{\Geod}{{\rm Geod}}
\newcommand{\geod}{{\rm geod}}
\newcommand{\hgr}{{\mathbb H}}
\newcommand{\mgr}{{\mathbb M}}
\newcommand{\ob}{{\rm Ob}}
\newcommand{\obg}{{\rm Ob(\mathbb G)}}
\newcommand{\obgp}{{\rm Ob(\mathbb G')}}
\newcommand{\obh}{{\rm Ob(\mathbb H)}}
\newcommand{\Osmooth}{{\Omega^{\infty}(X,*)}}
\newcommand{\ghomotop}{{\rho_2^{\square}}}
\newcommand{\gcalp}{{\mathbb G(\mathcal P)}}
\newcommand{\rf}{{R_{\mathcal F}}}
\newcommand{\glob}{{\rm glob}}
\newcommand{\loc}{{\rm loc}}
\newcommand{\TOP}{{\rm TOP}}
\newcommand{\wti}{\widetilde}
\newcommand{\what}{\widehat}
\renewcommand{\a}{\alpha}
\newcommand{\be}{\beta}
\newcommand{\ga}{\gamma}
\newcommand{\Ga}{\Gamma}
\newcommand{\de}{\delta}
\newcommand{\del}{\partial}
\newcommand{\ka}{\kappa}
\newcommand{\si}{\sigma}
\newcommand{\ta}{\tau}
\newcommand{\lra}{{\longrightarrow}}
\newcommand{\ra}{{\rightarrow}}
\newcommand{\rat}{{\rightarrowtail}}
\newcommand{\oset}[1]{\overset {#1}{\ra}}
\newcommand{\osetl}[1]{\overset {#1}{\lra}}
\newcommand{\hr}{{\hookrightarrow}}

\begin{document}
\section{Links to the AMS MSC 2010 Classification PDF of all MSC entries available, and the AMS MSC website}
Because the AMS MSC does not seem to be available at present as a link with NS 1.5, here are the links to 
the AMS that provide the MSC classifications:

\PMlinkexternal{All MSC 2010 in one PDF :}{http://www.ams.org/mathscinet/msc/pdfs/classifications2010.pdf}

\PMlinkexternal{The AMS MSC website with its maths specialized Search Engine}{http://www.ams.org/mathscinet/msc/msc2010.html}


\subsection{Conversion Tables}

http://www.ams.org/mathscinet/msc/pdfs/classifications2010.pdf

CONVERSIONS: http://www.ams.org/mathscinet/msc/conv.html?from=2000 


$MSC2000~ Classification~ Codes~ \to  ~MSC2010~ Classification~ Codes ~Update.$
Date: 14 October 2009


http://www.ams.org/mathscinet/msc/conv.html?from=2010

MSC2010 Classification Codes --> MSC2000 Classification Codes


\subsection{General Classifications}
{\bf 00-01 Instructional Expositions}

00-02 Research Expositions

00A05 General mathematics

00A35 Methodology of mathematics, didactics

00A66 Mathematics and visual arts, visualization

00A79 Physics 

00A69 General applied mathematics

00A73 Dimensional analysis

00A15 Bibliographies

00A71 Theory of mathematical modeling 

00A30 Philosophy of mathematics and 03A05

00A99 Miscellaneous topics

00B99 None of the above, but in this section}.
\subsection{Examples of Objects at PM with General Classifications}

{\bf 00-01 - General: Instructional exposition (textbooks, tutorial papers, etc.--here actually shown are only Encyclopedia articles):}


$(1+ \alpha/n)^n$ is monotone for large $n$ owned by Uri Weiss

{\em ad hoc}, owned by Cam McLeman

{\bf AMS MSC classification of articles and conversion tables},  owned by bci1

{\bf dimension}, owned by Boris Bukh

{\bf expression}, owned by Warren Buck

{\bf infix},  notation owned by Aaron Krowne

{\bf rigid} , owned by matte

{\bf strict},  owned by Raymond Puzio

{\bf Textbook projects on PlanetMath},  owned by John Smith

{\bf{ toy theorem}, owned by matte


{\em One can see many such examples by clicking on the AMS MSC classification specified at the bottom of any published article at PM.} 




\section{Algebraic Logic Examples of AMS MSC Classifications Utilized in PM articles}

msc:00-01, msc:00-02

00A15 Bibliographies


\subsection{Algebraic Logics}

03G05 Boolean algebras [See also 06Exx]

03G12 quantum logic [See also 06C15, 81P10]

03G20 $\L{}$ukasiewicz and Post algebras [See also 06D25, 06D30]

03G10 Lattices and related structures [See also 06Bxx]

03G30 Categorical logic, topoi [See also 18B25, 18C05, 18C10]

03H10 Other applications of nonstandard models (economics, physics, etc.)

03G15 Cylindric and polyadic algebras; relation algebras

03G20 Lukasiewicz and Post algebras [See also 06D25, 06D30]

03G25 Other algebras related to logic [See also 03F45, 06D20, 06E25, 06F35]

03G27 Abstract algebraic logic

03G30 Categorical logic, topoi [See also 18B25, 18C05, 18C10]

\subsection{Topology-General}

54-00 (General topology : General reference works (handbooks, dictionaries, bibliographies, etc.))

54D30 (General topology : Fairly general properties : Compactness)

54-01 (General topology : Instructional exposition (textbooks, tutorial papers, etc.))

54B10 (General topology : Basic constructions : Product spaces)

54D05 (General topology : Fairly general properties : Connected and locally connected spaces (general aspects))

54D45 (General topology : Fairly general properties : Local compactness, $\sigma$-compactness)

\subsection{Quantum Algebra}

\PMlinkexternal{Book: Quantum Algebra and Symmetries, Second Edition, 780 pages, 27 Mb pdf, (printed book 1,120 pages, in three volumes with an extensive Index of Contents and comprehensive References list)}{http://aux.planetmath.org/files/books/288/QuantumAlgebraSymmetry.pdf}

AMS MSC: 18-00 (Category theory; homological algebra :: General reference works (handbooks, dictionaries, bibliographies, etc.))

55R40 (Algebraic topology :: Fiber spaces and bundles :: Homology of classifying spaces, characteristic classes)

81T05 (Quantum theory :: Quantum field theory; related classical field theories :: Axiomatic quantum field theory; operator algebras)

81V05 (Quantum theory :: Applications to specific physical systems :: Strong interaction, including quantum chromodynamics)

81Q60 (Quantum theory :: General mathematical topics and methods in quantum theory :: Supersymmetric quantum mechanics)

(Cited from the Books section at PM).

%%%%%
%%%%%
\end{document}
