\documentclass[12pt]{article}
\usepackage{pmmeta}
\pmcanonicalname{Obvious}
\pmcreated{2013-03-22 14:43:42}
\pmmodified{2013-03-22 14:43:42}
\pmowner{rspuzio}{6075}
\pmmodifier{rspuzio}{6075}
\pmtitle{obvious}
\pmrecord{10}{36357}
\pmprivacy{1}
\pmauthor{rspuzio}{6075}
\pmtype{Definition}
\pmcomment{trigger rebuild}
\pmclassification{msc}{00A20}
\pmsynonym{easy to see}{Obvious}
\pmsynonym{clear}{Obvious}

\endmetadata

% this is the default PlanetMath preamble.  as your knowledge
% of TeX increases, you will probably want to edit this, but
% it should be fine as is for beginners.

% almost certainly you want these
\usepackage{amssymb}
\usepackage{amsmath}
\usepackage{amsfonts}

% used for TeXing text within eps files
%\usepackage{psfrag}
% need this for including graphics (\includegraphics)
%\usepackage{graphicx}
% for neatly defining theorems and propositions
%\usepackage{amsthm}
% making logically defined graphics
%%%\usepackage{xypic}

% there are many more packages, add them here as you need them

% define commands here
\begin{document}
\PMlinkescapeword{language}
\PMlinkescapeword{simple}
\PMlinkescapeword{cuts}
\PMlinkescapeword{length}
\PMlinkescapeword{potential}
\PMlinkescapeword{terms}
\PMlinkescapeword{average}


Mathematicians use phrases like ``it is obvious that'', ``it is easy to see that'', ``it is clear that'', and ``is trivial'' to indicate that some steps have been omitted.  The use of such language may be classified under three headings --- honest, dishonest, and pedagogical.

The honest use of these phrases occurs when only a few steps have been omitted and these steps are simple enough that the average reader can easily fill in the gaps.  Omitting such steps can be beneficial because it cuts down the length of an exposition and keeps the main ideas from getting lost amidst a morass of boring details and routine operations.  By reminding the reader of small omissions in an unobtrusive fashion, such phrases help put the reader at ease --- if they are left out, it is easy for the reader to be thrown off-course by a missing step or be left with an uneasy feeling that there might be a hole in a proof.

The dishonest use of these phrases occurs when a somewhat lengthy calculation has been left out because the author was too lazy to write it down.  By using these phrases, the author hopes to intimidate potential critics from pointing out that material is missing by insinuating that anyone who would point out that something is missing is too stupid to fill in a few obvious steps.  Frequent dishonest use of these terms may be a symptom of mathematheosis.

The pedagogical use of these phrases occurs when the author has deliberately left the filling-in of missing steps as an exercise to the reader.
%%%%%
%%%%%
\end{document}
