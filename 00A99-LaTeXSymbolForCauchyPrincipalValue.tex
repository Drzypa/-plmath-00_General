\documentclass[12pt]{article}
\usepackage{pmmeta}
\pmcanonicalname{LaTeXSymbolForCauchyPrincipalValue}
\pmcreated{2013-03-22 17:49:51}
\pmmodified{2013-03-22 17:49:51}
\pmowner{perucho}{2192}
\pmmodifier{perucho}{2192}
\pmtitle{\LaTeX symbol for Cauchy principal value}
\pmrecord{5}{40296}
\pmprivacy{1}
\pmauthor{perucho}{2192}
\pmtype{Feature}
\pmcomment{trigger rebuild}
\pmclassification{msc}{00A99}
\pmrelated{LogarithmicIntegral2}

\endmetadata

% this is the default PlanetMath preamble.  as your knowledge
% of TeX increases, you will probably want to edit this, but
% it should be fine as is for beginners.

% almost certainly you want these
\usepackage{amssymb}
\usepackage{amsmath}
\usepackage{amsfonts}
\usepackage{amsthm}

% used for TeXing text within eps files
%\usepackage{psfrag}
% need this for including graphics (\includegraphics)
%\usepackage{graphicx}
% for neatly defining theorems and propositions
%\usepackage{amsthm}
% making logically defined graphics
%%%\usepackage{xypic}

% there are many more packages, add them here as you need them

% define commands here
\newtheorem{theorem}{Theorem}
\newtheorem{defn}{Definition}
\newtheorem{prop}{Proposition}
\newtheorem{lemma}{Lemma}
\newtheorem{cor}{Corollary}
\def\Xint#1{\mathchoice
   {\XXint\displaystyle\textstyle{#1}}%
   {\XXint\textstyle\scriptstyle{#1}}%
   {\XXint\scriptstyle\scriptscriptstyle{#1}}%
   {\XXint\scriptscriptstyle\scriptscriptstyle{#1}}%
   \!\int}
\def\XXint#1#2#3{{\setbox0=\hbox{$#1{#2#3}{\int}$}
     \vcenter{\hbox{$#2#3$}}\kern-.5\wd0}}
\def\ddashint{\Xint=}
\def\dashint{\Xint-}


\begin{document}
The usual symbol (dashed integral) used to denote Cauchy's principal value of an integral can be created in 
\emph{\LaTeX{}} through macros.{\footnote{\emph{UK} List of \emph{\TeX{}} is a reference.}} These one are given by the following instructions, which must be included on the {\em preamble}.\\
\begin{verbatim}
\def\Xint#1{\mathchoice
   {\XXint\displaystyle\textstyle{#1}}%
   {\XXint\textstyle\scriptstyle{#1}}%
   {\XXint\scriptstyle\scriptscriptstyle{#1}}%
   {\XXint\scriptscriptstyle\scriptscriptstyle{#1}}%
   \!\int}
\def\XXint#1#2#3{{\setbox0=\hbox{$#1{#2#3}{\int}$}
     \vcenter{\hbox{$#2#3$}}\kern-.5\wd0}}
\def\ddashint{\Xint=}
\def\dashint{\Xint-}
\end{verbatim}
The commands to execute those macros are \, $``\backslash dashint"$ \, and\, $``\backslash ddashint"$ \, for single dash and double dash, respectively. Let us expose a few examples.
\begin{itemize}
\item\, \begin{equation*}\ddashint_\Omega F(\zeta,\eta)d\zeta d\eta\end{equation*}
\item\, \begin{equation*}\dashint_{z_0}^z f(\zeta)d\zeta\end{equation*}
\item\, \begin{equation*}Ei(z)=-\dashint_{-z}^\infty \frac{e^{-\zeta}}{\zeta}d\zeta=\dashint_{-\infty}^z\frac{e^\zeta}{\zeta}d\zeta,\quad\Re{z}>0,\quad \text{(exponential integral)}\end{equation*}
\item\, \begin{equation*}li(z)=\dashint_0^z\frac{d\zeta}{\log\zeta}\equiv Ei(\log z),\; \Re{z}>1,\quad
\text{(logarithmic integral)}\end{equation*}
\item\, \begin{equation*}\Gamma(z)\Gamma(1-z)=-z\Gamma(-z)\Gamma(z)=\dashint_0^\infty \frac{\zeta^{z-1}}{\zeta+1}d\zeta
=\pi\csc\pi z,\quad 0<\Re{z}<1,\quad\text{(Gamma function reflection's formula)}\end{equation*}
\end{itemize}

%%%%%
%%%%%
\end{document}
