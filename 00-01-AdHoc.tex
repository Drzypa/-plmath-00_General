\documentclass[12pt]{article}
\usepackage{pmmeta}
\pmcanonicalname{AdHoc}
\pmcreated{2013-03-22 14:45:20}
\pmmodified{2013-03-22 14:45:20}
\pmowner{mathcam}{2727}
\pmmodifier{mathcam}{2727}
\pmtitle{ad hoc}
\pmrecord{9}{36396}
\pmprivacy{1}
\pmauthor{mathcam}{2727}
\pmtype{Definition}
\pmcomment{trigger rebuild}
\pmclassification{msc}{00-01}

% this is the default PlanetMath preamble.  as your knowledge
% of TeX increases, you will probably want to edit this, but
% it should be fine as is for beginners.

% almost certainly you want these
\usepackage{amssymb}
\usepackage{amsmath}
\usepackage{amsfonts}
\usepackage{amsthm}

\usepackage{mathrsfs}

% used for TeXing text within eps files
%\usepackage{psfrag}
% need this for including graphics (\includegraphics)
%\usepackage{graphicx}
% for neatly defining theorems and propositions
%
% making logically defined graphics
%%%\usepackage{xypic}

% there are many more packages, add them here as you need them

% define commands here

\newcommand{\sR}[0]{\mathbb{R}}
\newcommand{\sC}[0]{\mathbb{C}}
\newcommand{\sN}[0]{\mathbb{N}}
\newcommand{\sZ}[0]{\mathbb{Z}}

 \usepackage{bbm}
 \newcommand{\Z}{\mathbbmss{Z}}
 \newcommand{\C}{\mathbbmss{C}}
 \newcommand{\R}{\mathbbmss{R}}
 \newcommand{\Q}{\mathbbmss{Q}}



\newcommand*{\norm}[1]{\lVert #1 \rVert}
\newcommand*{\abs}[1]{| #1 |}



\newtheorem{thm}{Theorem}
\newtheorem{defn}{Definition}
\newtheorem{prop}{Proposition}
\newtheorem{lemma}{Lemma}
\newtheorem{cor}{Corollary}
\begin{document}
The Latin phrase \emph{ad hoc} translates as ``toward this'', and is used in mathematics to describe anything that has been made up specifically for one particular purpose.

\subsubsection*{Examples}
\begin{enumerate}
\item an ad hoc definition is a temporary definition that might
be useful for the discussion at hand. For instance, let a \emph{nice set} 
be a closed set with smooth boundary. 
\item If a proof makes use of an ad hoc construction,  the proof
(typically) does not make use of a general theory for the problem. 
\end{enumerate}

See also
\PMlinkexternal{Article on ad hoc on Wikipedia}{http://en.wikipedia.org/wiki/Ad_hoc}
%%%%%
%%%%%
\end{document}
