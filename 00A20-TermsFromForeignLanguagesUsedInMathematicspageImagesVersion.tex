\documentclass[12pt]{article}
\usepackage{pmmeta}
\pmcanonicalname{TermsFromForeignLanguagesUsedInMathematicspageImagesVersion}
\pmcreated{2013-03-22 16:06:45}
\pmmodified{2013-03-22 16:06:45}
\pmowner{Wkbj79}{1863}
\pmmodifier{Wkbj79}{1863}
\pmtitle{terms from foreign languages used in mathematics (page images version)}
\pmrecord{37}{38178}
\pmprivacy{1}
\pmauthor{Wkbj79}{1863}
\pmtype{Topic}
\pmcomment{trigger rebuild}
\pmclassification{msc}{00A20}
\pmclassification{msc}{00A99}
\pmrelated{TermsFromForeignLanguagesUsedInMathematics}
\pmrelated{MathematicsVocabulary}

% this is the default PlanetMath preamble.  as your knowledge
% of TeX increases, you will probably want to edit this, but
% it should be fine as is for beginners.

% almost certainly you want these
\usepackage{amssymb}
\usepackage{amsmath}
\usepackage{amsfonts}

% used for TeXing text within eps files
%\usepackage{psfrag}
% need this for including graphics (\includegraphics)
%\usepackage{graphicx}
% for neatly defining theorems and propositions
%\usepackage{amsthm}
% making logically defined graphics
%%%\usepackage{xypic}
 \usepackage[T2A]{fontenc}
 \usepackage[russian, english]{babel}

% there are many more packages, add them here as you need them

% define commands here

\begin{document}
\PMlinkescapeword{point}

This entry is best viewed in page \PMlinkescapetext{images mode}.  For the html version, \PMlinkname{click here}{TermsFromForeignLanguagesUsedInMathematics}.

Following are \PMlinkescapetext{terms} from foreign \PMlinkescapetext{languages} that appear in mathematical literature.  Each \PMlinkescapetext{chart} (TeX \PMlinkescapetext{object} tabular) contains \PMlinkescapetext{terms} from the foreign \PMlinkescapetext{language} indicated.  The foreign \PMlinkescapetext{languages} are ordered according to how many \PMlinkescapetext{terms} appear in its corresponding \PMlinkescapetext{chart}.  In each \PMlinkescapetext{chart}, the \PMlinkescapetext{terms} are listed in alphabetical \PMlinkescapetext{order}.

\section{Latin}
\begin{center}
\small
\begin{tabular}{|c|c|c|c|}
\hline
abbr. & \PMlinkescapetext{term} & literal \PMlinkescapetext{translation} & mathematical usage \\
\hline
& {\em a fortiori} & with stronger reason & used in logic to denote an \\ 
& & & \PMlinkescapetext{argument} to the effect that \\ 
& & & because one ascertained fact exists; \\ 
& & & therefore another which is \\
& & & included in it or analogous to is \\ 
& & & and is less improbable, unusual, \\
& & & or surprising must also exist \\
\hline
& {\em a priori} & from the former & already known/assumed \\
\hline
& {\em ad absurdum} & to absurdity & an assumption is made in hopes of \\
& & & obtaining a contradiction \\
& & & [reductio ad absurdum is also used] \\
\hline
& {\em ad infinitum} & to infinity & endlessly, infinitely \\
\hline
& {\em casus irreducibilis} & not-reducible case & roots real but not \\
& & & expressible via real \PMlinkname{radicals}{Radical5} \\
\hline
cf. & {\em confer} & compare & used to suggest that another \\
& & & work might also be consulted \\
& & & in \PMlinkescapetext{relation} to that \PMlinkescapetext{argument} \\
\hline
et al. & {\em et alii} & and others & used in multi-author references \\ 
& & & but it is customary to include \\
& & & all the authors in  the first \\
& & & citation and/or in the \\
& & & bibliography \\
\hline
e.g. & {\em exempli gratia} & for example's sake & for example \\
\hline
ibid. & {\em ibidem} & in the same \PMlinkescapetext{place} & relates to the \\
& & & immediately prior \PMlinkescapetext{source} \\
\hline
i.e. & {\em id est} & that is & that is \\
\hline
inf & {\em inferior}, {\em infimum} & lowest & limit inferior; greatest lower bound \\
\hline
& {\em inter alia} & among other things & among other things \\
\hline
loc. cit. & {\em loco citato} & in the \PMlinkescapetext{place} already mentioned & relates to \PMlinkescapetext{sources} before the \\
& & & immediately prior citation \\
& & & [probably less frequent \\
& & & than op. cit.] \\
\hline
lb & {\em logarithmus binaris} & binary logarithm & log. in base 2 \\
\hline
lg & {\em logarithmus generalis} & general logarithm & log. in base 10 \\
\hline
ln & {\em logarithmus naturalis} & natural logarithm & log. in base $e$ \\
\hline
& {\em mutatis mutandis} & once changing thing to be changed & repeat the \PMlinkescapetext{similar argument} \\
& & & for the related case\\
\hline
N.B. & {\em nota bene} & note well & the following is important \\
\hline
op. cit. & {\em opere citato} & in the work already mentioned & relates to \PMlinkescapetext{sources} before the \\
& & & immediately prior citation \\
& & & [probably more frequent \\
& & & than loc. cit.] \\
\hline
QED & {\em quod erat demonstrandum} & which was to be demonstrated & end of proof \\
\hline
QEF & {\em quod erat faciendum} & which was to be done & end of construction \\
\hline
& {\em regula falsi}  & rule of false position & Newton's method \\
\hline
& {\em sine qua non}  & without which it could not be & an essential condition \\
& & & or element; an indispensable thing \\
\hline
sup & {\em superior}, {\em supremum} & uppermost & limit superior, \PMlinkname{least upper bound}{LowestUpperBound} \\
\hline
viz & {\em videlicet} & that is to say, namely & a keynote abbreviation \\
\hline
\end{tabular}
\end{center}


\section{German}
\begin{center}
\begin{tabular}{|c|c|c|c|}
\hline
abbr. & \PMlinkescapetext{term} & literal \PMlinkescapetext{translation} & mathematical usage \\
\hline
& {\em Ansatz} & approach, attempt & assumed form for an expression \\
\hline
& {\em eigen} & \PMlinkescapetext{characteristic}, typical & eigenvalue; eigenvector \\
\hline
& {\em Gr\"osse, Gr\"o\ss e} & size, magnitude & Gr\"ossencharacter \\
\hline
& {\em im kleinen} & in the small & connected im kleinen \\
\hline
& {\em Nullstellensatz} & zero point \PMlinkescapetext{theorem} & zero point \PMlinkescapetext{theorem} \\
\hline
& {\em Stufe} & stair, \PMlinkescapetext{level} & stufe of a field\\
\hline
& {\em Urelement} & primeval element & set element which is not a set \\
\hline
 $V$, $K_4$ & {\em Vierergruppe} & four-group & Klein 4-group \\
\hline
$\mathbb{Z}$ & {\em Zahlen} & numbers & integers \\
\hline
$Z$ & {\em Zentrum} & \PMlinkname{center}{GroupCentre} & \PMlinkname{center (of a group)}{GroupCentre} \\
\hline
\end{tabular}
\end{center}

\section{French}
\begin{center}
\begin{tabular}{|c|c|c|c|}
\hline
abbr. & \PMlinkescapetext{term} & literal \PMlinkescapetext{translation} & mathematical usage \\
\hline
& {\em espace} & space & (topological) space [see Espace \'Etal\'e] \\
\hline
& {\em \'etale} & slack & \'etale fundamental group; \PMlinkname{\'etale morphism}{Etale}; \'etale site \\
\hline
& {\em \'etal\'e} & spread out, displayed & \PMlinkname{\'Etal\'e space}{EtaleSpace3} \\
\hline 
p.p. & {\em presque partout} & almost everywhere & \PMlinkname{almost everywhere}{AlmostSurely} \\
\hline
\end{tabular}
\end{center}

\section{Russian}
\begin{center}
\begin{tabular}{|c|c|c|c|}
\hline
abbr. & \PMlinkescapetext{term} & literal \PMlinkescapetext{translation} & mathematical usage \\
\hline
$\partial$ & italic ``\cyrd'' [may be pronounced ``doh''] & letter ``d'' & e.g. in $\displaystyle{\frac{\partial f}{\partial x}}$ [see partial derivative] \\
\hline
\end{tabular}
\end{center}
%%%%%
%%%%%
\end{document}
