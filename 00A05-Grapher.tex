\documentclass[12pt]{article}
\usepackage{pmmeta}
\pmcanonicalname{Grapher}
\pmcreated{2013-03-22 17:46:13}
\pmmodified{2013-03-22 17:46:13}
\pmowner{PrimeFan}{13766}
\pmmodifier{PrimeFan}{13766}
\pmtitle{Grapher}
\pmrecord{4}{40226}
\pmprivacy{1}
\pmauthor{PrimeFan}{13766}
\pmtype{Definition}
\pmcomment{trigger rebuild}
\pmclassification{msc}{00A05}
\pmclassification{msc}{01A07}

\endmetadata

% this is the default PlanetMath preamble.  as your knowledge
% of TeX increases, you will probably want to edit this, but
% it should be fine as is for beginners.

% almost certainly you want these
\usepackage{amssymb}
\usepackage{amsmath}
\usepackage{amsfonts}

% used for TeXing text within eps files
%\usepackage{psfrag}

% need this for including graphics (\includegraphics)
\usepackage{graphicx}

% for neatly defining theorems and propositions
%\usepackage{amsthm}
% making logically defined graphics
%%%\usepackage{xypic}

% there are many more packages, add them here as you need them

% define commands here

\begin{document}
{\em Grapher} is a software graphing calculator that comes bundled with the Apple Mac OS Xoperating system. It can graph 2-dimensional equations like $y = \sin x$, as well as 3-dimensional equations like $z = \frac{y^3}{x^2 + y^2}$ (Cartan's umbrella, illustrated below).

\begin{center}
\includegraphics{C:TempGrapherScreenShot}
\end{center}

Equations to be graphed can be entered in Cartesian form or in parametric form, polar coordinates, logarithmic, etc. The program automatically ``typesets'' the formulas as they are entered, for example, converting \verb=x^2= to $x^2$ or \verb=1/x= to $\frac{1}{x}$. The program can create animations, and one can also move a graph or rotate it (just by dragging it, for 3D graphs; for 2D graphs one must first select the Move Tool).

In a default installation, the program is in the Utilities subfolder of the Applications folder. Mac OS 9 had a similar program, called Graphing Calculator, which was also capable of graphing 2D and 3D equations as well as animations.

\begin{thebibliography}{2}
\bibitem{ml} Maria Langer, {\it Visual Quickstart Guide: Mac OS 9.1}. New York: Peachpit Press (2001): 107
\bibitem{dp} David Pogue \& Adam Goldstein, {\it The Missing Manual: Switching to the Mac, Tiger Edition}. Sebastopol: O'Reilly Media, Inc. (2005): 452 - 453
\end{thebibliography}
%%%%%
%%%%%
\end{document}
