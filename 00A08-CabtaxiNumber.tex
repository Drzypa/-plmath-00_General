\documentclass[12pt]{article}
\usepackage{pmmeta}
\pmcanonicalname{CabtaxiNumber}
\pmcreated{2013-03-22 17:56:08}
\pmmodified{2013-03-22 17:56:08}
\pmowner{PrimeFan}{13766}
\pmmodifier{PrimeFan}{13766}
\pmtitle{cabtaxi number}
\pmrecord{4}{40432}
\pmprivacy{1}
\pmauthor{PrimeFan}{13766}
\pmtype{Definition}
\pmcomment{trigger rebuild}
\pmclassification{msc}{00A08}

% this is the default PlanetMath preamble.  as your knowledge
% of TeX increases, you will probably want to edit this, but
% it should be fine as is for beginners.

% almost certainly you want these
\usepackage{amssymb}
\usepackage{amsmath}
\usepackage{amsfonts}

% used for TeXing text within eps files
%\usepackage{psfrag}
% need this for including graphics (\includegraphics)
%\usepackage{graphicx}
% for neatly defining theorems and propositions
%\usepackage{amsthm}
% making logically defined graphics
%%%\usepackage{xypic}

% there are many more packages, add them here as you need them

% define commands here

\begin{document}
A {\em cabtaxi number} for a given $n$ is the smallest positive number which can be written as $a^3 + b^3$ in $n$ different ways, with either $a$ or $b$ allowed to be negative integers. For example, 91 is the 2nd cabtaxi number since it can be expressed $(-5)^3 + 6^3 = 3^3 + 4^3 = 91$. The known cabtaxi numbers are 1, 91, 728, 2741256, 6017193, 1412774811, 11302198488, 137513849003496, 424910390480793000, listed in A047696 of Sloane's OEIS. Adding the restriction $a \geq b > 0$ gives the definition for the taxicab numbers.
%%%%%
%%%%%
\end{document}
